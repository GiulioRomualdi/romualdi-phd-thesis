\section{Dynamics-based whole-body QP control layer} \label{sec:dynamics_QP}
The Dynamics-based whole-body QP control layer considers the dynamic model of the system to ensure the tracking of the desired trajectories. The proposed controller computes the desired robot joint torque $\tau$, the generalized mixed representation acceleration ${}^{B[\mathcal{I}]}\dot{\nu}$, and a set of desired spatial contact forces expressed in the mixed representation ${}_{\mathcal{C}_j[\mathcal{I}]}\mathrm{f}_j$. Here $B[\mathcal{I}] = (o_B, [\mathcal{I}])$ is a frame placed on the base of the robot and oriented as the inertial frame $\mathcal{I}$. On the other hand, $\mathcal{C}_j[\mathcal{I}] = (p_j, \mathcal{I})$ is the frame associated with the admissible contact (e.g. the foot) where $j\in\mathbb{N}$ such that $1\le j \le n_c$, with $n_c$ is the number of admissible contacts. The control problem is formulated using the stack of tasks approach. Similar to the kinematics based whole-body control layer (Section~\ref{sec:ik_qp}), the objective is achieved by transcribing the problem as a constrained optimization problem. We define the low priority tasks as terms of the cost function. While the high priority tasks are treated as constraints.
\par
An approach similar to the one presented in this section has been also described in~\citep{nava16,Dean-Leon2019Whole-BodySkin,Englsberger2018,DelPrete2016ImplementingSensors,Henze2016,Ramuzat2022PassiveMulti-Contact,Mesesan2019}
\par
In the next section, we present the set of low and high priority tasks. For the sake of clarity, we simplify the notation of the generalized mixed robot velocity and acceleration by dropping the superscript $B[\mathcal{I}]$. Furthermore the contact wrenches are stacked into a vector $\mathrm{f} = \begin{bmatrix}
{}_{\mathcal{C}_1[\mathcal{I}]}\mathrm{f}_1^\top & {}_{\mathcal{C}_2[\mathcal{I}]}\mathrm{f}_2^\top & \hdots & {}_{\mathcal{C}_{n_c}[\mathcal{I}]}\mathrm{f}_{n_c}^\top 
\end{bmatrix}^\top \in \mathbb{R}^{6 n_c }$

\subsection{Low and high priority tasks\label{sec:tsid_tasks}}
What follows presents the tasks required to define the control objectives. We denote by $\Psi$ the equality task and by $\Phi$ the inequality task.

\subsubsection{Centroidal momentum task}
Given a frame $\bar{G} = (x_\text{CoM},[\mathcal{I}])$, the centroidal momentum rate of change ${}_{\bar{G}} \dot{h}$ balances the external spatial force applied on the robot~\eqref{eq:centroidal_momentum_dynamics_mixed}:
\begin{equation}
    \label{eq:tsid_centroidal_dynamics}
    {}_{\bar{G}} \dot{{h}} = \sum_{j = 1}^{n_c} \begin{bmatrix}
	I_3 & {0}_{3\times 3} \\
	\left({p}_j - {x}_\text{CoM}\right)\times & I_3 
	\end{bmatrix} {}_{\mathcal{C}_j[\mathcal{I}]}\mathrm{f}_j + m \bar{g},
\end{equation}
where $\bar{g} = \begin{bmatrix} 0&0&-g&0&0&0 \end{bmatrix}^\top$ is the 6D gravity acceleration vector.  Rearranging Equation~\eqref{eq:tsid_centroidal_dynamics}, we obtain
\begin{IEEEeqnarray}{LL}
\phantomsection \IEEEyesnumber \IEEEyessubnumber*
 {}_{\bar{G}} \dot{{h}} &= \begin{bmatrix}
	I_3 & {0}_{3\times 3} & \hdots & I_3 & {0}_{3\times 3} \\
	\left({p}_1 - {x}_\text{CoM}\right)\times & I_3 & \hdots & \left({p}_{n_c} - {x}_\text{CoM}\right)\times & I_3 
	\end{bmatrix} \mathrm{f} + m \bar{g} \\
	&= A_c \mathrm{f} + m\bar{g}.
\end{IEEEeqnarray}
To ask for a desired centroidal momentum trajectory, we specify the following task:
\begin{equation}
    \Phi_h = {}_{\bar{G}} \dot{{h}}^* - A_c \mathrm{f} - m\bar{g}.
\end{equation}
We set the desired linear centroidal momentum rate of change to guarantee the tracking of the desired center of mass trajectory $x_\text{CoM}^\text{ref}(t)$ as 
\begin{equation}
\label{eq:tsid_h_p_star}
   {}_{\bar{G}} \dot{{h}}^{p^*} = m\left[ \ddot{x}^\text{ref}_\text{CoM} +K^d_\text{CoM} (\dot{x}^\text{ref}_\text{CoM} - \dot{x}_\text{CoM}) + K^p_\text{CoM}({x}^\text{ref}_\text{CoM} - {x}_\text{CoM})   \right].
\end{equation}
Here $K^p_\text{CoM}$ and $K^d_\text{CoM}$ are positive define diagonal matrices. On the other hand,  the desired angular momentum rate of change ${}_{\bar{G}} \dot{{h}}^{\omega^*}$ is given by
\begin{equation}
\label{eq:tsid_h_omega_star}
      {}_{\bar{G}} \dot{{h}}^{\omega^*} = {}_{\bar{G}} \dot{{h}}^{\omega^\text{ref}} + K_{h^\omega} \left({}_{\bar{G}} {{h}}^{\omega^\text{ref}} - {}_{\bar{G}} {{h}}^{\omega}\right),
\end{equation}
where $ K_{h^\omega} > 0$. If a high-level whole-body planner is available, for instance \citep{carpentier2016versatile}, ${{h}}^{\omega^\text{ref}}$ is chosen to satisfy~\eqref{eq:cmm_intro}
\begin{equation}
    {}_{\bar{G}} h^{\omega^\text{ref}} = J_{\text{CMM}_\omega}^\text{pl} {\nu}^\text{pl},
\end{equation}
where the superscript \emph{pl} indicates that the quantity is computed by the high-level planner. 
Another common choice is to set the desired angular momentum equal to zero, i.e., ${{h}}^{\omega^\text{ref}} = 0_{3\times1}$
\par
Similar to the kinematics-based controller case, the centroidal momentum task is often divided into linear and angular centroidal momentum tasks. The linear task, denoted by $\Psi_\text{CoM}$ is given by
\begin{IEEEeqnarray}{ll}
\phantomsection \label{eq:tsid_com_task} \IEEEyesnumber \IEEEyessubnumber*
    \Psi_\text{CoM} &= \begin{bmatrix} I_3 & 0_{3\times3} \end{bmatrix} \Psi_h \\
    &= {}_{\bar{G}} \dot{h}^{p^*} - \begin{bmatrix} I_3 & 0_{3\times3} \end{bmatrix} A_c \mathrm{f} - mg \\
     &= {}_{\bar{G}} \dot{h}^{p^*} - A_{c_p} \mathrm{f} - mg
\end{IEEEeqnarray}
where ${}_{\bar{G}} \dot{h}^{p^*}$ is chosen as~\eqref{eq:tsid_h_p_star} and $ A_{c_p}$ is the matrix composed of the first three rows of $ A_{c}$. The angular momentum task $\Psi_{h^{\omega}}$ is 
\begin{IEEEeqnarray}{ll}
\phantomsection \IEEEyesnumber \IEEEyessubnumber*
\label{eq:tsid_acm_task}
    \Psi_{h^{\omega}} &= \begin{bmatrix}  0_{3\times3} & I_3  \end{bmatrix} \Psi_h \\
    &= {}_{\bar{G}} \dot{h}^{\omega^*} - \begin{bmatrix} 0_{3\times3} & I_3  \end{bmatrix} A_c \mathrm{f} \\
     &= {}_{\bar{G}} \dot{h}^{\omega^*} - A_{c_\omega} \mathrm{f}.
\end{IEEEeqnarray}
${}_{\bar{G}} \dot{h}^{\omega^*}$ is set equal to \eqref{eq:tsid_h_omega_star}. $A_{c_\omega}$ is the matrix composed of the last three rows of $A_{c}$.

\subsubsection{Cartesian task}
While walking, we often require some of the robot link frames to have a specific position and orientation with respect to the inertial frame. To accomplish this task, we recall that given a frame $L$, its acceleration expressed in mixed representation, denoted as ${}^{L[\mathcal{I}]}\dot{\mathrm{v}}_{\mathcal{I},L}$, is given by
\begin{equation}
    {}^{L[\mathcal{I}]}\dot{\mathrm{v}}_{\mathcal{I},L} = J_{L} \dot{\nu} + \dot{J}_{L} {\nu},
\end{equation}
where $J_{L}$ is the mixed velocity Jacobian of the link $L$~\eqref{eq:mixed-link-velocity} and $\dot{J}_{L}$ its derivative. 
To follow a desired Cartesian trajectory ${}^\mathcal{I} H_L^\text{ref} = (p_L^\text{ref}, {}^\mathcal{I} R_L^\text{ref}) \in \mathbb{R}^3 \times \SO(3)$, we specify the following task:
\begin{equation}
\label{eq:tsid_se3_task}
    \Psi_{L_{\SE(3)}} =  {}^{L[\mathcal{I}]}\dot{\mathrm{v}}_{\mathcal{I},L}^* - J_{L} \dot{\nu} - \dot{J}_{L} {\nu},
\end{equation}
where $ {}^{L[\mathcal{I}]}\dot{\mathrm{v}}^*_{\mathcal{I},L} =\begin{bmatrix}
\ddot{p}_L^{*^\top} & {}^\mathcal{I}\dot{\omega}^{*^\top}_{\mathcal{I},L}
\end{bmatrix} ^\top $ is chosen as~\footnote{An efficient implementation of the controller is available at \href{https://github.com/ami-iit/lie-group-controllers}{\texttt{https://github.com/ami-iit/lie-group-controllers}}.}
\begin{IEEEeqnarray}{c}
\phantomsection \label{eq:tsid_se3_task_velocity} \IEEEyesnumber \IEEEyessubnumber*
\ddot{p}_L^{*} = \ddot{p}_L^\text{ref} +  K_{L_p}^d(\dot{p}_L^\text{ref} - \dot{p}_L) +  K_{L_p}^p(p_L^\text{ref} - p_L) \label{eq:tsid_se3_task_velocity_R3} \\
{}^\mathcal{I}\dot{\omega}^{*}_{\mathcal{I},L} = {}^\mathcal{I}\dot{\omega}^\text{ref}_{\mathcal{I},L} + K_{L_\omega}^d  ( {}^\mathcal{I}\omega^\text{ref}_{\mathcal{I},L} -  {}^\mathcal{I}\omega_{\mathcal{I},L}) +  K_{L_\omega}^p \Log\left({}^\mathcal{I} R_L^\text{ref} \;\; {}^\mathcal{I} R_L^\top\right). \label{eq:tsid_se3_task_velocity_so3}
\end{IEEEeqnarray}
Here $K_{L_p}^p$, $K_{L_p}^d$, $K_{L_\omega}^p$ and $K_{L_\omega}^d$ are positive defined matrices.
\par
Starting from the definition of the $\SE(3)$ task~\eqref{eq:tsid_se3_task}, we introduce the positional and rotational tasks for the frame $L$, respectively, denoted as $\Psi_{L_{\mathbb{R}^3}}$ and $\Psi_{L_{\SO(3)}}$:
\begin{IEEEeqnarray}{ll}
\phantomsection \label{eq:tsid_r3_task}
 \IEEEyesnumber \IEEEyessubnumber*
\Psi_{L_{\mathbb{R}^3}} &= \begin{bmatrix}
    I_3 & 0_{3\times3} 
    \end{bmatrix}\Psi_{L_{\SE(3)}} \\
    &= \ddot{p}^*_L - \begin{bmatrix}
    I_3 & 0_{3\times3} 
    \end{bmatrix} \left(J_L \dot{\nu}  + \dot{J}_L \nu  \right) \\
     &= \ddot{p}^*_L -  J_{L_p} \dot{\nu}  - \dot{J}_{L_p} \nu,
\end{IEEEeqnarray}
\begin{IEEEeqnarray}{ll}
\phantomsection \label{eq:tsid_so3_task} \IEEEyesnumber \IEEEyessubnumber*
\Psi_{L_{\SO(3)}} &= \begin{bmatrix}
    0_{3\times3} & I_3 
    \end{bmatrix}\Psi_{L_{\SE(3)}} \\
    &= {}^\mathcal{I} \dot{\omega}^*_{\mathcal{I},L} - \begin{bmatrix}
    0_{3\times3} & I_3 
    \end{bmatrix} \left(J_L \dot{\nu}  + \dot{J}_L \nu  \right) \\
     &= {}^\mathcal{I} \dot{\omega}^*_{\mathcal{I},L} -  J_{L_\omega} \dot{\nu}  - \dot{J}_{L_\omega} \nu.
\end{IEEEeqnarray}
$\ddot{p}^*_L$ is given by \eqref{eq:tsid_se3_task_velocity_R3}, while ${}^\mathcal{I} \dot{\omega}^*_{\mathcal{I},L}$ is obtained from \eqref{eq:tsid_se3_task_velocity_so3}. $J_{L_p}$ and $J_{L_\omega}$ are, respectively, the matrices composed of the first and last three rows of $J_{L}$.


\subsubsection{Floating base dynamics task}
To whole-body control layer often considers the base~\eqref{eq:robot_dynamics_base} and joint dynamics~\eqref{eq:robot_dynamics_joints} as a set of high priority tasks. In this context, we define the base dynamics task as 
\begin{equation}
    \label{eq:tsid_base_dyn_task}
    \Psi_{\text{dyn}_\nu} =  h_{\nu} + M_{\nu}\dot{\nu} - J_{\mathcal{C}_\nu}^\top \mathrm{f}.
\end{equation}
On the other hand, the joint dynamics task is given by
\begin{equation}
    \label{eq:tsid_joint_dyn_task}
    \Psi_{\text{dyn}_s} =  h_{s} +  M_{s}(q) \dot{\nu} - \tau - J_{\mathcal{C}_s}^\top \mathrm{f},
\end{equation}
where the subscript $\nu$ refers to the first 6 rows of the matrix, while $s$ to the last $n$ rows, with $n$ the number of actuated degrees of freedom of the system. 

\subsubsection{Joint regularization task}
To prevent the controller from providing solutions with huge joint variations, we introduce a regularization tasks for the joint variables. The task is achieved by asking for a desired joint velocity that depends on the error between the desired and measured joint values, such as:
\begin{equation}
    \label{eq:tsid_s_task}
    \Psi_s = \ddot{s}^* - \begin{bmatrix}
    0_{n\times6} & I_n 
    \end{bmatrix} \dot{\nu},
\end{equation}
$\ddot{s}^*$ is equal to
\begin{equation}
    \ddot{s}^* = \ddot{s}^\text{ref} + k_s^d (\dot{s}^\text{ref} - \dot{s}) + k_s^p (s^\text{ref} - s).
\end{equation}
Here $s^\text{ref}$ is the desired joint configuration trajectory and is often set constant, i.e. $s^\text{ref}(t) = \bar{s}$.


\subsubsection{Local ZMP task}
Given a desired ZMP position expressed in the ${\mathcal{C}_j[\mathcal{I}]}$ frame, we want to define a task such that the desired contact force in $\mathcal{C}_j$ generates ${}^{\mathcal{C}_j[\mathcal{I}]} x_{\text{ZMP}_j}$. We now recall that, given a 6D force ${}_{\mathcal{C}_j[\mathcal{I}]} \mathrm{f}_j$, the ZMP, if exists, is given by~\citep{vukobratovic2004zero}
\begin{equation}
    \label{eq:tsid_local_zmp_definition}
   {}^{\mathcal{C}_j[\mathcal{I}]} x_{\text{ZMP}_j} = 
   \begin{bmatrix}
   -\frac{{}_{\mathcal{C}_j[\mathcal{I}]} \mu_{j_y}}{{}_{\mathcal{C}_j[\mathcal{I}]} f_{j_z}} \\
   \frac{{}_{\mathcal{C}_j[\mathcal{I}]} \mu_{j_x}}{{}_{\mathcal{C}_j[\mathcal{I}]} f_{j_z}}
   \end{bmatrix}.
\end{equation}
Assuming that ${}_{\mathcal{C}_j[\mathcal{I}]} f_z \ne 0$, i.e., the contact is active, we rearrange Equation~\eqref{eq:tsid_local_zmp_definition} as
\begin{equation}
    \label{eq:tsid_local_zmp_definition_rearranged}
  {}_{\mathcal{C}_j[\mathcal{I}]} f_{z_j}  {}^{\mathcal{C}_j[\mathcal{I}]} x_{\text{ZMP}_j} = 
   \begin{bmatrix}
   -{}_{\mathcal{C}_j[\mathcal{I}]} \mu_{j_y} \\
   {}_{\mathcal{C}_j[\mathcal{I}]} \mu_{j_x}
   \end{bmatrix}.
\end{equation}
Considering~\eqref{eq:tsid_local_zmp_definition_rearranged}, we define the local ZMP task for the contact $\mathcal{C}_j$ as
\begin{equation}
    \Psi_{\text{ZMP}_j} = \left({}^{\mathcal{C}_j[\mathcal{I}]} x_{\text{ZMP}_j}^\text{ref} e_3^\top - 
    \begin{bmatrix}
        0& 0&0&0&-1&0 \\
        0& 0&0&1&0&0
    \end{bmatrix} \right) {}_{\mathcal{C}_j[\mathcal{I}]} \mathrm{f}_j.
\end{equation}

\subsubsection{Global ZMP task}
In the case all the robot contact lies on the same plane, for example, the robot is walking on a planar surface, it is possible to ask for a desired global ZMP, denoted with ${}^\mathcal{I}x_\text{ZMP}$ (please note that the suffix $j$ is not present in this case since the ZMP is considered as a global quantity and not strictly dependent to the contact).
Given $n_c$ coplanar contacts, we define the global ZMP as~\citep[Section~3.2.3]{Kajita2014IntroductionRobotics}:
\begin{equation}
\label{eq:tsid_zmp_global_def}
    {}^\mathcal{I}x_\text{ZMP} = \sum_{j=1}^{n_c} \frac{{}_{\mathcal{C}_j[\mathcal{I}]}f_{j_z}}{\sum_{i=1}^{n_c} {}_{\mathcal{C}_i[\mathcal{I}]}f_{i_z}}\left( p_j + {}^{\mathcal{C}_j[\mathcal{I}]}x_{\text{ZMP}_j}  \right).
\end{equation}
By combining~\eqref{eq:tsid_local_zmp_definition} with \eqref{eq:tsid_zmp_global_def} we obtain
\begin{equation}
    {}^\mathcal{I}x_\text{ZMP} = \sum_{j=1}^{n_c} \frac{{}_{\mathcal{C}_j[\mathcal{I}]}f_{j_z}}{\sum_{i=1}^{n_c} {}_{\mathcal{C}_i[\mathcal{I}]}f_{i_z}}\left( p_j + 
    \begin{bmatrix}
   -\frac{{}_{\mathcal{C}_j[\mathcal{I}]} \mu_{j_y}}{{}_{\mathcal{C}_j[\mathcal{I}]} f_{j_z}} \\
   \frac{{}_{\mathcal{C}_j[\mathcal{I}]} \mu_{j_x}}{{}_{\mathcal{C}_j[\mathcal{I}]} f_{j_z}}
   \end{bmatrix}  \right).
\end{equation}
Assuming that ${}_{\mathcal{C}_j[\mathcal{I}]}f_{j_z} > 0$ we define the global ZMP task as 
\begin{equation}
\label{eq:tsid_zmp_global_task}
    \Psi_{\text{ZMP}} = \left({}^{\mathcal{I}} x_{\text{ZMP}}^\text{ref}   \begin{bmatrix}
        e_3^\top & \hdots e_3^\top 
    \end{bmatrix}- 
    \begin{bmatrix}
        \Gamma_1 & \hdots \Gamma_{n_c}
    \end{bmatrix} \right)\mathrm{f},
\end{equation}
where $\Gamma_i\in \mathbb{R}^{2\times6}$ writes as
\begin{equation}
 \Gamma_i = \begin{bmatrix}
 0 & 0 & p_{i_x} & 0 & -1 & 0 \\
 0 & 0 & p_{i_y} & 1 & 0 & 0
 \end{bmatrix}.
\end{equation}


\subsubsection{Generic variable regularization task}
In order to prevent the controller from providing solutions with huge torque $\tau$ or contact force ${}_{\mathcal{C}_j[\mathcal{I}]}\mathrm{f}_j$, we introduce a regularization task of the form
\begin{equation}
    \label{eq:tsid_generic_regularization}
    \Psi_u = u^\text{ref} - u,
\end{equation}
where $u$ is either the joint torque $\tau$ or the contact force ${}_{\mathcal{C}_j[\mathcal{I}]}\mathrm{f}_j$.

\subsubsection{Feasibile contact force task}
The feasibility of the contact wrench ${}_{\mathcal{C}_j[\mathcal{I}]}\mathrm{f}_j$ is guaranteed by a set of inequality of the form:
\begin{equation}
  \Phi_{\mathrm{f}_j}: \;\;  \label{eq:tsid_contact_wrench_feasibility_constraint}
    A_{\mathcal{C}_j[\mathcal{I}]} \; {}_{\mathcal{C}_j[\mathcal{I}]}\mathrm{f}_j \le b.
\end{equation}
where $A_{\mathcal{C}_j[\mathcal{I}]}$ is a matrix that depends on the position of the robot joints and on the base pose.
\par
More specifically, ${}_{\mathcal{C}_j[\mathcal{I}]}\mathrm{f}_j$ must belong to the associated friction cone, while the position of the local CoP is constrained within the support polygon. 

\subsubsection{Joint limits task}
The feasibility of the joint position $s$ is generally guaranteed by means of a set of inequalities of the form 
\begin{equation}
\label{eq:tsid_joint_limits}
    s^- \le s \le s^+,
\end{equation}
where $s^-$ and $s^+$ are respectively the lower and upper joints position limits. 
In the case of dynamics-based whole-body controller, the joint values $s$ cannot be arbitrary chosen. To overcome this issue, we substitute the joint position $s$ in~\eqref{eq:tsid_joint_limits} with its second-order discrete dynamics computed at time $t_0 + k \diff t$ by applying the forward Euler method, i.e. $s(t_0 + (k+1) \diff t) = s_{k+1} = s_k + \diff t \dot{s}_k + 0.5 \diff t^2 \ddot{s}_k$:
\begin{equation}
\label{eq:tsid_joint_limits_euler}
    s^- \le  s_k + \diff t \dot{s}_k + \frac{1}{2} \diff t^2 \ddot{s}_k \le s^+.
\end{equation}
Rearranging~\eqref{eq:tsid_joint_limits_euler} we obtain the final formulation of the joint limits task
\begin{equation}
   \Phi_s :  \;\; 2 \frac{s^-  -s - \diff t \dot{s} }{\diff t^2} \le   \dot{s} \le 2 \frac{s^+  -s - \diff t \dot{s}}{\diff t^2}.
\end{equation}
where we remove the time dependency $k$.
\subsubsection{Feasible joint torque}
To guarantee feasible contact torque, we define the following inequality task
\begin{equation}
    \label{eq:tsid_joint_limits_torque_task}
    \Phi_\tau:\;\; \tau^{-} \le \tau \le \tau^+,
\end{equation}
where $\tau^{-}$ and $\tau^+$ represent the maximum negative and positive torque that the joints can produce. 

\subsection{Quadratic programming problem \label{sec:tsid_optimal_control_problem}}
The control objective is achieved by transcribing the control problem as a constrained optimization problem considering the tasks presented in Section~\ref{sec:tsid_tasks}.
Similarly to what we discussed for the kinematics-based whole-body controller in Section~\ref{sec:ik_qp}, an equality task $\Psi$ can always be considered as a lower or high priority task, or, in another word, as a term of the cost function or as an equality constraint. Consequently, this makes the dynamics based whole-body problem modular\footnote{The \texttt{QPTSID} class implemented in our framework exploits this modularity to customize the control problem: \href{https://github.com/ami-iit/bipedal-locomotion-framework/tree/v0.6.0/src/TSID}{\texttt{https://github.com/ami-iit/bipedal-locomotion-framework/tree/v0.6.0/src/TSID}}}. 
\par
From now on, we consider the following set of tasks.
The tracking of the left and right foot poses are considered as high priority $\SE(3)$ tasks~\eqref{eq:tsid_se3_task} and they are denoted respectively as $\Psi_{L_{\SE(3)}}$ and $\Psi_{R_{\SE(3)}}$. We take into account the CoM tracking as high priority task~\eqref{eq:tsid_com_task} and the global ZMP tracking~\eqref{eq:tsid_zmp_global_task}. We also consider the base~\eqref{eq:tsid_base_dyn_task} and the joint dynamics~\eqref{eq:tsid_joint_dyn_task} as high priority tasks. 
The torso orientation is implemented as a low priority task $\SO(3)$ task~\eqref{eq:tsid_so3_task} and we denote it with $\Psi_{T_{\SO(3)}}$. Furthermore, the joint postural condition~\eqref{eq:tsid_s_task} is also added as a low priority task. To minimize the joint torques and forces, we also add two regularization tasks~\eqref{eq:tsid_generic_regularization} denoted $\Psi_\tau$ and $\Psi_\mathrm{f}$. We also ask for a joint torque $\tau$ such that the inequality joint limits task~\eqref{eq:tsid_joint_limits_torque_task} is satisfied. Finally, to guarantee feasible contact forces for the left and right feet, we add the task~\eqref{eq:tsid_contact_wrench_feasibility_constraint}, denoted respectively as $\Phi_{\mathrm{f}_L}$ and $\Phi_{\mathrm{f}_R}$.
\par
The above hierarchical control objectives can be cast into a whole-body optimization problem:
\begin{IEEEeqnarray}{CL}
\phantomsection \label{eq:tsid_optimization} \IEEEyesnumber \IEEEyessubnumber*
\;\minimize\limits_{\dot{\nu}, \; \tau, \;\mathrm{f}} \; & \Psi_{T_{\SO(3)}}^\top \Lambda_T \Psi_{T_{\SO(3)}} + \Psi_{s}^\top \Lambda_s \Psi_{s} + \Psi_{\mathrm{f}}^\top \Lambda_\mathrm{f} \Psi_{\mathrm{f}} + \Psi_{\tau}^\top \Lambda_\tau \Psi_{\tau} \label{eq:tsid_optimization_cost} \\
\st & \Psi_{L_{\SE(3)}} = 0  \label{eq:tsid_optimization_costraint_lf} \\
&  \Psi_{R_{\SE(3)}} = 0 \label{eq:tsid_optimization_costraint_rf} \\
& \Psi_{\text{CoM}} = 0 \label{eq:tsid_optimization_costraint_com} \\
& \Psi_{\text{dyn}_\nu} = 0 \label{eq:tsid_optimization_costraint_dyn_base} \\
& \Psi_{\text{dyn}_s} = 0 \label{eq:tsid_optimization_costraint_dyn_s} \\
& \Psi_{\text{ZMP}} = 0 \label{eq:tsid_optimization_costraint_zmp} \\
& \Phi_s \label{eq:tsid_optimization_costraint_s}  \\
& \Phi_\tau \label{eq:tsid_optimization_costraint_tau} \\
&\Phi_{\mathrm{f}_L} \\
&\Phi_{\mathrm{f}_R}
\end{IEEEeqnarray}
Since the decision variables are the robot acceleration $\dot{\nu}$, joint torques $\tau$ and the contact forces ${}_{\mathcal{C}_j[\mathcal{I}]}\mathrm{f}_j$, and the tasks depend linearly on them, we convert the optimization problem~\eqref{eq:tsid_optimization} into a quadratic programming problem (Section~\ref{sec:qp}) and we solve it via off-the-shelf solvers.




