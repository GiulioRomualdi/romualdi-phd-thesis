\section{Conclusions\label{sec:centroidal_mpc_conclusion}}
This chapter discusses the development of an online centroidal momentum non-linear MPC for humanoid robots. 
The controller aims to generate feasible contact locations and wrenches for locomotion.
Different from state-of-the-art architectures based on simplified models (e.g. LIPM), the proposed controller can be used to perform highly dynamic movements, such as jumping and running. Furthermore, the contact location adjustment is considered in the centroidal dynamics stabilization problem, so it is not required to design an ad-hoc block for this feature. We validate the controller with a simulation of one-leg and two-leg systems performing jumping and running tasks, respectively. 
The centroidal MPC is also embedded in the three-layer position-based control architecture and tested on the humanoid robot iCub v3 -- see Section~\ref{sec:iCub3}. The proposed strategy prevents the robot from falling while walking and pushed with external forces up to $\SI{40}{\newton}$ for 1 second applied to the robot arm.
\par
In future work, we may extend the MPC to consider the contact timing adjustment, thus increasing the robustness properties against unpredictable external disturbances.
Another interesting research direction is to substitute the Euler integrator required to transcribe the optimal control problem into an MPC~\ref{sec:mpc_formulation} with a multi-rate sampling technique~\citep{Elobaid2020Sampled-dataPlanning,Elobaid2019OnSampling}. Considering a multi-rate sampling technique, it would be possible to reduce the prediction model's discretization error. To improve the time performance, we may consider applying the convex relaxation of the angular momentum dynamics~\citep{Ponton2018,Ponton2016AGeneration} in the controller prediction model. As a consequence, it would be possible to solve the non-linear optimization problem~\eqref{eq:mpc_centroidal_contact_optimization} by using a convex programming solver and thus increasing the MPC frequency.  
Finally, to improve the overall time performance, we plan to warm start the non-linear optimization problem with the result of a human-like trajectory planner~\citep{Viceconte2022ADHERENT:Robots}.