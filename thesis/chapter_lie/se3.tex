\section{The Euclidean group\label{sec:se3}}

The Euclidean set $\SE(3)$ is defined as
\begin{equation}
\SE(3) :=  
\left\{ 
\begin{bmatrix} R & p \\ 0_{1\times3} & 1 \end{bmatrix} \in \mathbb{R}^{4 \times 4} \mid 
R \in \SO(3), p \in \mathbb{R}^3
\right\}.
\end{equation}
$\SE(3)$ is a group under the product operator. The identity element of the ($\SE(3), \cdot$) group is the $4 \times 4$ identity matrix $I_4$. Given two frames $A$ and $B$, an element of ${}^AH_B\in \SE(3)$ describes the position and orientation of the frame $B$ with respect to the other frame $A$ and it writes as
\begin{equation}
  \label{eq:homogeneousTransformation}
  {}^A H_B :=
     \begin{bmatrix}
     {}^A R_B & {}^A o_B \\
     0_{1\times3} & 1
  \end{bmatrix} .
\end{equation}
Given a 3D vector ${}^B p$ whose coordinates are written in $B$ and the same vector ${}^A p$ whose coordinates are written in $A$. We define the the \emph{homogeneous representation} of ${}^A p$ and ${}^B p$ as ${}^A \bar p := ({}^A p; 1) \in \mathbb{R}^4$ and likewise for
${}^B \bar p$. Then
\begin{equation} \label{eq:Abp_AHB_Bbp}
  {}^A \bar p = 
  {}^A H_B {}^B \bar p ,
\end{equation}
which is the matrix form of 
${}^A p = {}^A R_B {}^B p + {}^A o_B$.
In this context ${}^A H_B$ is often called \emph{homogeneous transformation}.

\subsection{6D spatial velocity\label{sec:se3_velocity}}
Given a smooth trajectory ${}^A H_B {}(t) \in \SE(3)$ we define the time derivative of the homogeneous transformation as ${}^A {\dot H}_B$. A more \emph{compact} representation of 
${}^A {\dot H}_B$ can be obtained
multiplying it by the inverse of ${}^A H_B$ on the left or on the right.
In both cases, the result element is the so called \emph{6D spatial velocity} and it is an element of $\se(3)$
Where the set $\se(3)$ is defined as
\begin{equation}
\se(3) :=  
\left\{ 
\begin{bmatrix} \omega \times & v \\ 0_{1\times3} & 0 \end{bmatrix} \in \mathbb{R}^{4 \times 4}  \mid \omega \times \in \so(3), v \in \mathbb{R}^3 
\right\}.
\end{equation} 
Hereafter the \emph{6D spatial velocity} is often denoted also a \emph{twist}.
We notice that an element of $\se(3)$ is uniquely identified by a 6D vector. Similar to what we discussed for the angular velocity (Section~\ref{sec:angular_velocity}), we now introduce the \emph{hat} operator. Given $\mathrm{v} = \begin{bmatrix}v^\top ,  \omega^\top  \end{bmatrix}^\top \in \mathbb{R}^6$. We define the \emph{hat} operator as 
\begin{equation}
    \mathrm{v} ^\wedge := \begin{bmatrix}
     v \\
     \omega
    \end{bmatrix}  ^\wedge =  \begin{bmatrix} \omega \times & v \\ 0_{1\times3} & 0 \end{bmatrix}.
\end{equation}
The inverse of the \emph{hat} map is denoted as \emph{vee} and it is defined as
\begin{equation}
    \begin{bmatrix} \omega \times & v \\ 0_{1\times3} & 0 \end{bmatrix}^\vee := \mathrm{v}.
\end{equation}
A more rigorous definition of the \emph{hat} and \emph{vee} operators is provided in Equation~\eqref{eq:hat_vee_operators}.
\par
Considering a smooth curve ${}^A H _B (t) \in \SE(3)$ representing a time-varying homogeneous transformation from the frame $B$ to the frame $A$. We define the \emph{right trivialized spatial velocity}, denoted with ${}^A \mathrm{v} ^\wedge _{A,B} \in \se(3)$ as
\begin{IEEEeqnarray}{LL}
\phantomsection  \label{eq:se3right} \IEEEyesnumber  \IEEEyessubnumber*
 {}^A \mathrm{v} ^\wedge _{A,B} := {}^{A} \dot{H}_{B} {}^{A} H_{B}^{-1}   
  & =  
  \begin{bmatrix}
    {}^A \dot{R}_B & {}^A{\dot o}_B \\
    0_{1\times3} & 0
  \end{bmatrix} 
  \begin{bmatrix} 
    {}^A {R}_B^\top & -{}^A {R}_B^\top {}^Ao_B \\
    0_{1\times3} & 1
  \end{bmatrix} \\
  & =
  \begin{bmatrix}
    {}^A {\dot R}_B {}^AR_B^\top  & {}^A {\dot o}_B - {}^A\dot{R}_B  {}^AR_B^\top  {}^A {o}_B\\
    0_{1\times3} & 0
  \end{bmatrix} .
\end{IEEEeqnarray}
Note that $ {}^A {\dot R}_B {}^AR_B^\top $ appearing on the right hand side of \eqref{eq:se3right} is skew symmetric and it is equal to~\eqref{eq:so3_right}. We now define ${}^B v_{A,B}$ and ${}^B \omega_{A,B} \in \mathbb{R}^3$ as
\begin{IEEEeqnarray}{LL}
\phantomsection  \IEEEyesnumber  \IEEEyessubnumber*
  {}^A v_{A,B} 
& := 
  {}^A {\dot o}_B - {}^A\dot{R}_B  {}^AR_B^\top  {}^A {o}_B
\\ 
\label{eq:AwAB}
  {}^A \omega^{\wedge}_{A,B} 
& := 
  {}^A {\dot R}_B {}^AR_B^\top.
\end{IEEEeqnarray}
The \emph{right trivialized} velocity of frame $B$ with respect to the frame $A$ is then given by
\begin{equation}
{}^A \spatial{v}_{A,B} := 
\begin{bmatrix}
{}^{A} v_{A,B} \\
{}^{A} \omega_{A,B}
\end{bmatrix} 
\in \mathbb{R}^6 .
\end{equation}
If the frame $A$ is the inertial frame and $B$ a frame rigidly attached to the body, the right trivialized spatial velocity is equivalent to the so-called \emph{inertial velocity}. The angular term of ${}^A \spatial{v}_{A,B}$, ${}^B \omega_{A,B}$, is the angular velocity of the frame $B$ with respect to the frame $A$, whose coordinates are written in $A$. The linear component, ${}^A v_{A,B}$ is the velocity, expressed in $A$, of a point belonging to the rigid body that is instantaneously coincident with the origin $o_A$. To give the reader a better understanding, ${}^A v_{A,B}$ is the velocity of a \emph{rigid extension} of the rigid body that extends up to the origin of the inertial frame $A$ -- see Figure~\ref{fig:spatial-velocity}.
\par
We define the \emph{left trivialized spatial velocity}, denoted with ${}^B \mathrm{v} ^\wedge _{A,B} \in \se(3)$, as
\begin{IEEEeqnarray}{LL}
\phantomsection  \label{eq:se3ex} \IEEEyesnumber  \IEEEyessubnumber*
  {}^B \mathrm{v} ^\wedge _ {A,B} := {}^{A} H_{B}^{-1} {}^{A} \dot{H}_{B}  
  & =  
  \begin{bmatrix} 
    {}^A {R}_B^T & -{}^A {R}_B^T {}^Ao_B \\
    0_{1\times3} & 1
  \end{bmatrix}
  \begin{bmatrix}
    {}^A \dot{R}_B & {}^A{\dot o}_B \\
    0_{1\times3} & 0
  \end{bmatrix} 
  \\
  & =
  \begin{bmatrix}
    {}^AR_B^T {}^A {\dot R}_B & {}^AR_B^T {}^A {\dot o}_B \\
    0_{1\times3} & 0
  \end{bmatrix} .
\end{IEEEeqnarray}

Note that ${}^A {R}_B^T {}^A \dot{R}_B$ appearing on the right hand side of \eqref{eq:se3ex} is skew symmetric and it is equal to~\eqref{eq:so3_left}.
Let us now define ${}^B v_{A,B}$ and ${}^B \omega_{A,B} \in \mathbb{R}^3$ so that
\begin{IEEEeqnarray}{LL}
\phantomsection \IEEEyesnumber  \IEEEyessubnumber*
  {}^B v_{A,B} 
& := 
  {}^A R_B^T {}^A {\dot o}_B , \label{eq:BvAB}
\\ 
  {}^B \omega^{\wedge}_{A,B} 
& := 
  {}^A R_B^T {}^A {\dot R}_B . \label{eq:BwAB}
\end{IEEEeqnarray}
The \emph{left trivialized} velocity of frame $B$ 
with respect to frame $A$ is
\begin{equation}
\label{eq:left_trivialized_velocity_definition}
{}^B \spatial{v}_{A,B} := 
\begin{bmatrix}
{}^{B} v_{A,B} \\
{}^{B} \omega_{A,B}
\end{bmatrix} 
\in \mathbb{R}^6 .
\end{equation}
If the frame $A$ is the inertial frame and $B$ a frame rigidly attached to the body, the right trivialized spatial velocity is equivalent to the so-called $\emph{body velocity}$. ${}^B \omega_{A,B}$ is the angular velocity of the frame $B$ with respect to the frame $A$, whose coordinates are written in $B$. The linear component, ${}^B v_{A,B}$ is the velocity of the origin of $B$, $o_B$ with respect the frame $A$ whose coordinates are written in $B$ -- see Figure~\ref{fig:spatial-velocity}.
\begin{figure}[tpb]
    \centering
	\includegraphics{chapter_lie/figures/spatial-velocity.tikz}
	\caption[Spatial velocity of a rigid-body.]{Spatial velocity of a rigid-body. ${}^A \omega_{A,B}$ and ${}^B \omega_{A,B}$ are the angular velocity of the rigid body $\mathcal{O}$ expressed in $A$ and $B$, respectively. ${}^A v_{A,B}$ is the linear velocity, expressed in $A$, of a point belonging to the \emph{rigid extension} of the rigid body $\mathcal{O}$ coincident with the origin $o_A$. ${}^B v_{A,B}$ is the linear velocity, expressed in $B$, of the point $o_B$.
	\label{fig:spatial-velocity}}
\end{figure}

\subsection{6D spatial force\label{sec:6d-spatial-force}}
Given a rigid body, we introduced the spatial force as a 6D vector containing a pure force a torque applied to the body~\footnote{A more rigorous approach is presented in Appendix~\ref{sec:co-tangent-space-and-lie-coalgebra}}~\citep[Chapter 2.5]{traversaro2017}. Similar to the spatial velocity, also the spatial force admits a left and right trivialization. Given a body with a frame, denoted with $B$, rigidly attached to it and a frame $A$ fixed in the space, we define \emph{right trivialized 6D-force}, denoted ${}_B \mathrm{f} \in \se(3)^*$ as 
\begin{equation}
    {}_B \mathrm{f} := 
    \begin{bmatrix}
       {}_B f \\
       {}_B \mu
    \end{bmatrix} \in \mathbb{R}^6.
\end{equation}
where $\se(3)^*$ is the dual space of $\se(3)$\footnote{A rigorous definition of the dual space is provided in Appendix~\ref{sec:co-tangent-space-and-lie-coalgebra}} and ${}_B f$ represents the force acting on the rigid body and ${}_B \mu$ is the pure torque about the origin $o_B$ -- see Figure~\ref{fig:spatial-force}.
\begin{figure}[tpb]
    \centering
	\includegraphics{chapter_lie/figures/spatial-force.tikz}
	\caption[Spatial force of a rigid-body.]{Spatial force of a rigid-body. ${}_A f$ and ${}_B f$ are the force acting on  the rigid body $\mathcal{O}$ expressed in $A$ and $B$, respectively. ${}_A \mu$ is the torque, expressed in $A$, of a point belonging to the \emph{rigid extension} of the rigid body $\mathcal{O}$ coincident with the origin $o_A$. ${}_B \mu$ is the torque, expressed in $B$, of the point $o_B$.
	\label{fig:spatial-force}}
\end{figure}
We define \emph{left trivialized spatial force}, ${}_A \mathrm{f}\in  \se(3)^*$ as 
\begin{equation}
    {}_A \mathrm{f} := 
    \begin{bmatrix}
       {}_A f \\
       {}_A \mu
    \end{bmatrix} \in \mathbb{R}^6.
\end{equation}
Here ${}_A f$ represents the force acting on the rigid body and ${}_A \mu$ is the pure torque about the origin $o_A$ -- Figure~\ref{fig:spatial-force}. To give the reader a better understanding, ${}_A \mu$ is the torque about a \emph{rigid extension} of the rigid body that extends up the origin of the inertial frame A -- see Figure~\ref{fig:spatial-force}.

\subsection{Exponential and Logarithmic map}
Given a small increment $\mathrm{v}^\wedge \in \se(3)$ such that
\begin{equation}
\mathrm{v}^\wedge = \begin{bmatrix}
   \theta \times & \rho \\
   0_{1\times3} & 0
\end{bmatrix} \in \se(3),
\quad 
\mathrm{v}= \begin{bmatrix}
   \rho \\
   \theta
\end{bmatrix} \in \mathbb{R}^6,
\end{equation} 
where $\rho\in\mathbb{R}^3$ and $\theta\in\mathbb{R}^3$ are, respectively, the linear and the angular terms of the increment.
The $\Exp$ map writes as~\citep{Sola2018ARobotics}:
\begin{equation}
    H = \Exp(\mathrm{v}) := \begin{bmatrix}
       \Exp(\theta) & V(\theta) \rho \\
       0_{3\times1} & 1
    \end{bmatrix},
\end{equation}
where $V(\theta)$ is given by
\begin{equation}
    V(\theta) =  I_3 + \frac{1 - \cos{\| \theta \|}}{\| \theta \| ^2} (\theta \times) + \frac{\| \theta \| - \sin{\| \theta \|}}{\| \theta \| ^3} (\theta \times) ^2.
\end{equation}

Given an element of the 3D rigid motion group $H \in \SE(3)$ such that
\begin{equation}
H = \begin{bmatrix}
   R & p \\
   0_{1\times3} & 1
\end{bmatrix} \in \SE(3),
\end{equation} 
the $\Log$ map is
\begin{equation}
    \mathrm{v}= \Log(H) := 
    \begin{bmatrix}
       V^{-1}\left(\Log(R)\right) p \\
       \Log(R)
    \end{bmatrix}.
\end{equation}


\subsection{The adjoint representation} 
Considering ${}^A H _B \in \SE(3)$ and a spatial velocity element ${}^B \mathrm{v}_{A,B} ^\wedge \in \se(3)$, the adjoint representation maps a left trivialized spatial velocity into a right trivialized one as: 
\begin{equation}
\label{eq:adjoint_action_se3}
    {}^A \mathrm{v}_{A,B} ^\wedge = \Ad_{{}^A H _B }\left({}^B \mathrm{v}_{A,B}^\wedge\right) = {}^A H _B {}^B \mathrm{v}_{A,B}^\wedge {}^A H _B^ {-1}.
\end{equation}
Since the adjoint representation is a linear mapping, there exists the inverse of $\Ad_{{}^A H _B}$, hereafter written as $\Ad_{{}^A H _B } ^{-1} = \Ad_{{}^A H _B ^{-1}}$. $\Ad_{{}^A H _B } ^{-1}$ maps a right trivialized spatial velocity into a left trivialized one, i.e.,
\begin{equation}
    {}^B \mathrm{v}_{A,B} ^\wedge = \Ad_{{}^A H _B }^{-1}\left({}^A \mathrm{v}_{A,B}^\wedge\right) =  \Ad_{{}^A H _B ^{-1} }\left({}^A \mathrm{v}_{A,B}^\wedge\right) = {}^A H _B ^{-1} {}^A \mathrm{v}_{A,B}^\wedge {}^A H _B.
\end{equation}
Given the adjoint representation $\Ad_{{}^A H _B }$, the associated adjoint matrix $\AdM_{{}^A H _B }$ is
\begin{equation}
\label{eq:adjoint_matrix_se3_def}
    \AdM_{{}^A H _B } =
    \begin{bmatrix}
    {}^A R_B & \left({}^A o_B \times\right) {}^A R_B  \\
     0_{3\times3} & {}^A R_B 
    \end{bmatrix}.
\end{equation}
The adjoint matrix $\AdM_{{}^A H _B }$ is identified by developing the adjoint representation in~\eqref{eq:adjoint_action_se3} as
\begin{IEEEeqnarray}{LL}
    \phantomsection \IEEEyesnumber  \IEEEyessubnumber*
     \AdM_{H} \mathrm{v} &= \left[\Ad_{H}\left(\mathrm{v}^\wedge\right) \right]^\vee = (H \mathrm{v} H^{-1})^\vee \\
     &= \begin{bmatrix}
     R (\omega\times) R ^\top &  -R (\omega\times) R ^\top o + R v \\
     0_{1\times3} & 0
     \end{bmatrix} ^\vee \\ 
      &= \begin{bmatrix}
     (R \omega)\times  &  (o\times) R \omega + R v \\
     0_{1\times3} & 0
     \end{bmatrix} ^\vee \\
      &= \begin{bmatrix}
     (o\times) R \omega + R v\\
     R \omega
     \end{bmatrix} = \begin{bmatrix}
    R & \left(o\times\right) R \\
     0_{3\times3} & R
    \end{bmatrix} 
    \begin{bmatrix}
   v \\ \omega
    \end{bmatrix}.
\end{IEEEeqnarray}
Where, to improve the readability all the prefix and postfix have been removed. 
Hereafter the Adjoint matrix $\AdM_{{}^A H _B }$ is denoted with ${}^A X _B $. The linear mapping between the ${}^B\mathrm{v}_{A,B}$ and ${}^A\mathrm{v}_{A,B}$ writes as
\begin{equation}
    \label{eq:left_to_right}
    {}^A\mathrm{v}_{A,B} = {}^A X _B {}^B\mathrm{v}_{A,B}. 
\end{equation}
\par
A rigorous definition of the Adjoint representation is provided in Equation~\eqref{eq:adjoint_action_def}.

\subsection{The co-adjoint representation}
Similarly to what we discussed for a 6D velocities, we can define a linear map to change the coordinates of a 6D force from a frame $B$ to another frame $A$. This coordinate transformation is indicated with ${}_A X ^ B$ and written as
\begin{equation}
    \label{eq:coadjoint_matrix_se3_def}
    {}_A\mathrm{f} = {}_A X ^ B {}_B\mathrm{f}
\end{equation}
Sometimes, ${}_A X ^ B$ is also denoted with $\AdM_{{}^A H_B}^*$, see Appendix~\ref{sec:adjoint_and_coadjoint_representation}.
\par
The matrix ${}_A X ^ B$ is induced by the velocity transformation \eqref{eq:adjoint_matrix_se3_def} and it is related to ${}^B X _ A$ as
\begin{equation}
   \AdM_{{}^A H_B}^* =  {}_A X ^ B = {}^B X _ A ^{\top} = \AdM_{{}^A H ^{-1}_B}^\top.
\end{equation}
In particular
\begin{equation}
    \AdM_{{}^A H_B}^* =
    \begin{bmatrix}
    {}^A R_B &  0_{3\times3} \\
     \left({}^A o_B \times\right) {}^A R_B & {}^A R_B 
    \end{bmatrix}.
\end{equation}


\subsection{The adjoint representation of $\se(3)$}
Considering ${}^B u(t)^\wedge \in \se(3)$ and ${}^A u(t)^\wedge \in \se(3)$ such that
\begin{equation}
    \label{eq:adjoint-motion-equation-matrix_se3_initial}
    {}^A u(t) = {}^A X _B(t) {}^B u(t).
\end{equation}
The time derivative of  
we may evaluate the time of Equation~\eqref{eq:adjoint-motion-equation-matrix_se3_initial} is given by
\begin{IEEEeqnarray}{LL}
    \phantomsection \label{eq:adjoint-motion-equation-matrix_se3} \IEEEyesnumber  \IEEEyessubnumber*
    \frac{\diff}{\diff t} {}^A u &= \frac{\diff}{\diff t} \left({}^A X_B {}^B u \right) \\
    &= {}^A X_B \left(\frac{\diff }{\diff t} {}^B u + {}^B \mathrm{v}_{A,B} \times {}^B u  \right),
\end{IEEEeqnarray}
where ${}^B \mathrm{v}_{A,B} \times$ is defined as
\begin{equation}
    {}^B \mathrm{v}_{A,B} \times := 
    \begin{bmatrix}
     {}^B \omega_{A,B} \times & {}^B v_{A,B} \times \\
     0_{3\times3} & {}^B \omega_{A,B}  \times
    \end{bmatrix}.
\end{equation}
The ${}^B \mathrm{v}_{A,B} \times$ matrix is often denoted as $\adM_{{}^B \mathrm{v}_{A,B} ^ \wedge}$, a rigorous explanation of this choice is given by Equation~\eqref{eq:adjoint-motion-equation-matrix}.


\subsection{The co-adjoint representation of $\se(3)$}
Similarly to what we discussed for a 6D velocities, we can compute the time derivative of the co-adjoint linear map~\eqref{eq:coadjoint_matrix_se3_def} as:
\begin{IEEEeqnarray}{LL}
    \phantomsection \label{eq:co-adjoint-motion-equation-matrix_se3} \IEEEyesnumber  \IEEEyessubnumber*
    \frac{\diff}{\diff t} {}_A \mathrm{f} &= \frac{\diff}{\diff t} \left({}_A X^B {}_B \mathrm{f} \right) \\
    &= {}_A X^B \left(\frac{\diff }{\diff t} {}_B \mathrm{f} + {}^B \mathrm{v}_{A,B} \times^* {}_B \mathrm{f}  \right),
\end{IEEEeqnarray}
where ${}^B \mathrm{v}_{A,B} \times^*$ is defined as
\begin{equation}
    {}^B \mathrm{v}_{A,B} \times^* := 
    \begin{bmatrix}
     {}^B \omega_{A,B} \times & 0_{3\times3} \\
     {}^B v_{A,B} \times & {}^B \omega_{A,B}  \times.
    \end{bmatrix}
\end{equation}
The ${}^B \mathrm{v}_{A,B} \times^*$ matrix is often denoted as $\adM^*_{{}^B \mathrm{v}_{A,B} ^ \wedge}$, a rigorous explanation of this choice is given by Equation~\eqref{eq:co-adjoint-motion-equation-matrix}.

\subsection{Mixed spatial velocity\label{sec:mixed_spatial_velocity}} 
In some applications, it may be helpful to define the 6D velocity as
\begin{equation}
    \begin{bmatrix}
       {}^A\dot{o}_B \\
       {}^A\omega_{A,B}
    \end{bmatrix}.
\end{equation} 
This representation is often called \emph{hybrid}~\citep{Murray1994} or \emph{mixed}~\citep[Chapter~2.3]{traversaro2017} velocity representation. To avoid confusion with hybrid systems theory, we will call it as \emph{mixed} velocity of frame $B$ with respect to frame $A$. 
The mixed representation is particularly helpful when designing and implementing controllers that aim to follow a desired Cartesian trajectory. 
To simplify the description of the algorithms and concepts presented in this thesis, we need a way to express this quantity coherently with the rest of the concepts introduced until know. Given two frame $A$ and $B$ with a given origin and orientation, hereafter indicated as $o_A$, $o_B$ and $[A]$ and $[B]$, we introduce the frame $B[A]:=(o_B, \; [A])$ as a frame having the origin on $o_B$ and oriented as $A$. We formally define the mixed velocity ${}^{B[A]}\mathrm{v}_{A,B}$ as 
\begin{equation}
\label{eq:mixed_velocity_definition}
    {}^{B[A]}\mathrm{v}_{A,B} := {}^{B[A]} X _B {}^B \mathrm{v} _{A,B} = 
    \begin{bmatrix}
       {}^A R _B & 0_{3\times3} \\
       0_{3\times3} & {}^A R _B
    \end{bmatrix}
    \begin{bmatrix}
       {}^A R _B ^\top {}^A \dot{o}_B\\
       {}^{B}\omega_{A,B}
    \end{bmatrix} = 
    \begin{bmatrix}
       {}^A\dot{o}_B \\
       {}^A\omega_{A,B}
    \end{bmatrix}.
\end{equation}
It is important to notice that ${}^{B[A]}\mathrm{v}_{A,B}$ is not isomorphic to any element of $\se(3)$, i.e., ${}^{B[A]}\mathrm{v}_{A,B} ^\wedge \notin \se(3)$. On the other hand, ${}^{B[A]}\mathrm{v}_{A,B}$ can be seen as the row concatenation of the left trivialized linear velocity whose coordinate are expressed in $A$ and the right trivialized angular velocity. In other words ${}^{B[A]}\mathrm{v}_{A,B}$ will belong to the Cartesian product of $\mathbb{R}^3$ and  $\so(3)$, i.e., ${}^{B[A]}\mathrm{v} ^\wedge _{A,B} \in \mathbb{R}^3 \times \so(3)$. 

\subsection{Mixed spatial force}
Smilar to the mixed spatial velocity, in some applications, it may be helpful to define the 6D force as
\begin{equation}
{}_{B[A]} \mathrm{f} = 
    \begin{bmatrix}
      {}_A f \\
      {}_{B[A]}\mu
    \end{bmatrix}.
\end{equation} 
This representation is often called \emph{mixed}~\citep[Chapter~2.3]{traversaro2017} force representation.
Given two frames $A$ and $B$ with a given origin and orientation, hereafter indicated as $o_A$, $o_B$ and $[A]$ and $[B]$, we introduce the frame $B[A]:=(o_B, \; [A])$ as a frame having the origin on $o_B$ and oriented as $A$. We formally define the mixed spatial force ${}_{B[A]}\mathrm{f}$ as 
\begin{equation}
\label{eq:mixed_force_definition}
    {}_{B[A]}\mathrm{f} := \AdM^*_{{}^{B[A]} H _B} {}_B \mathrm{f} =  {}_{B[A]} X ^B {}_B \mathrm{f} = 
    \begin{bmatrix}
       {}^A R _B & 0_{3\times3} \\
       0_{3\times3} & {}^A R _B
    \end{bmatrix}
    \begin{bmatrix}
       {}_B f\\
       {}_{B}\mu
    \end{bmatrix} = 
    \begin{bmatrix}
     {}_A f \\
     {}_{B[A]}\mu
    \end{bmatrix}.
\end{equation}
To give the reader a better comprehension, we may think that a mixed spatial force is a 6D vector whose linear component is the force acting on the rigid body whose entries are written with respect to the frame $A$, while the angular component is the torque about $o_B$ whose coordinate are written in $A$. 
\par
The mixed spatial force will play a crucial role in the design of the control systems presented in the following parts.



\section{Rigid body dynamics\label{sec:rigid-body-dynamics}}
Let a rigid body $\mathcal{O}$ lying in a uniform gravitational filed and given a frame $B$ rigidly attached to $\mathcal{O}$. Given an inertial frame $\lieGroup{I}$, the configuration of $\mathcal{O}$ is completely determined by the homogeneous transformation ${}^\lieGroup{I} H _ B \in \SE(3)$. We denote ${}^B \mathrm{v}_{\lieGroup{I} ,B} \in \se(3)$ the left trivialized spatial velocity of $\mathcal{O}$. Let $m$ the mass of the rigid body, $\mathbb{I}\in\mathbb{R}^{3\times3}$ the 3D inertial matrix of the body expressed in $B$ and ${}^B p_c$ the position of the center of mass (CoM) expressed in $B$, we introduce the 6D inertia matrix of the body as
\begin{equation}
\label{eq:6d_inertial_matrix_body}
    {}_B \mathbb{M} _ B = 
    \begin{bmatrix}
    m & -m  {}^B p_c \times \\
    m  {}^B p_c \times  & \mathbb{I}
    \end{bmatrix}.
\end{equation}
It is important to recall that $ {}_B \mathbb{M} _ B$ is a constant matrix. 
\par
We now introduce the \emph{left-trivialized Lagrangian of the rigid body}, denoted with  $\mathcal{L}({}^\lieGroup{I} H _ B, {}^B \mathrm{v}_{\lieGroup{I} ,B})$, as
\begin{IEEEeqnarray}{LL}
\phantomsection  \label{eq:se3_lagrangian} \IEEEyesnumber  \IEEEyessubnumber*
\mathcal{L}({}^\lieGroup{I} H _ B, {}^B \mathrm{v}_{\lieGroup{I} ,B}) &= \mathcal{K}({}^B \mathrm{v}_{\lieGroup{I} ,B}) - \mathcal{U}({}^\lieGroup{I} H _ B)\\
& \frac{1}{2} {}^B \mathrm{v}_{\lieGroup{I} ,B}^\top \; {}_B \mathbb{M} _ B  \; {}^B \mathrm{v}_{\lieGroup{I} ,B} + 
   \begin{bmatrix}
    g^\top & 0_{1\times3}
   \end{bmatrix} {}^B H _ \lieGroup{I}  
      \begin{bmatrix}
    m {}^B p_c\\ 
    m
   \end{bmatrix},
\end{IEEEeqnarray}
where $\mathcal{K}({}^B \mathrm{v}_{\lieGroup{I} ,B})$ is the \emph{kinetic energy}, while $\mathcal{U}({}^\lieGroup{I} H _ B)$ is the \emph{potential energy}. $g \in \mathbb{R}^3$ is the gravitational acceleration vector.
By applying \emph{Hamilton's Variational Principle}, it is possible to prove that given a time interval $[t_0,t_f]$, the trajectory ${}^B H _ \lieGroup{I}$ of the rigid body $\mathcal{O}$ is the one that minimizes the action~\citep[Theorem~2.1]{Traversaro2017ModellingDynamics}:
\begin{equation}
       \mathfrak{G} = \int_{t_0}^{t_f} \mathcal{L}({}^\lieGroup{I} H _ B, {}^B \mathrm{v}_{\lieGroup{I} ,B})\diff t,
\end{equation}
where $\mathcal{L}({}^\lieGroup{I} H _ B, {}^B \mathrm{v}_{\lieGroup{I} ,B})$ is given by~\eqref{eq:se3_lagrangian}.
\par
In particular the resulting equation of motions are the one that satisfies the following differential equation
\begin{equation}
\label{eq:eurel-poincare-equations-lie-se3}
     {}_B \mathbb{M} _ B {}^B \dot{\mathrm{v}} _{\lieGroup{I}, B } + {}^B \mathrm{v} _{\lieGroup{I}, B } \times^* {}_B \mathbb{M} _ B {}^B \mathrm{v} _{\lieGroup{I}, B }  = {}_B \mathbb{M} _ B \begin{bmatrix}
      {}^\lieGroup{I} R _ B ^\top g \\ 
      0_{3\times1}
     \end{bmatrix}.
\end{equation}
\par
The reader can find a brief introduction of \emph{Hamilton's Variational Principle} in Appendix~\ref{appendix:hamilton}.



     \par
In the case of a non-conservative spatial force acting on the body, Equation~\eqref{eq:eurel-poincare-equations-lie-se3} becomes 
\begin{equation}
\label{eq:eurel-poincare-equations-lie-se3-force}
     {}_B \mathbb{M} _ B {}^B \dot{\mathrm{v}} _{\lieGroup{I}, B } + {}^B \mathrm{v} _{\lieGroup{I}, B } \times^* {}_B \mathbb{M} _ B {}^B \mathrm{v} _{\lieGroup{I}, B }  = {}_B \mathbb{M} _ B \begin{bmatrix}
      {}^\lieGroup{I} R _ B ^\top g \\ 
      0_{3\times1}
     \end{bmatrix} + {}_B \mathrm{f}.
\end{equation}
Equation~\eqref{eq:eurel-poincare-equations-lie-se3-force} can be expressed also with respect to the inertial frame $\mathcal{I}$ \begin{equation}
\label{eq:eurel-poincare-equations-lie-se3-force-inertial}
     {}_\lieGroup{I} \mathbb{M} _ B {}^B \dot{\mathrm{v}} _{\lieGroup{I}, B } + {}^\lieGroup{I} \mathrm{v} _{\lieGroup{I}, B } \times^* {}_\lieGroup{I} \mathbb{M} _ B {}^\lieGroup{I} \mathrm{v} _{\lieGroup{I}, B }  = {}_\lieGroup{I} \mathbb{M} _ B \begin{bmatrix}
      g \\ 
      0_{3\times1}
     \end{bmatrix} + {}_\lieGroup{I} \mathrm{f}.
\end{equation}
Even if \eqref{eq:eurel-poincare-equations-lie-se3-force-inertial} seems similar to \eqref{eq:eurel-poincare-equations-lie-se3-force}, it is important to underline that $ {}_B \mathbb{M} _ B$ is a constant matrix depending only on the geometry of the rigid body, while $ {}_\lieGroup{I} \mathbb{M} _ B$ is a time-varying quantity that depends on the position of the rigid-body - i.e.,  ${}_\lieGroup{I} \mathbb{M} _ B =  {}_\lieGroup{I} X ^B {}_B \mathbb{M} _ B {}_\lieGroup{I} X ^B$.
\par
We finally introduce the 6D spatial momentum of the body $\mathcal{O}$ with respect to a frame $\lieGroup{I}$ expressed in $B$  and denoted with ${}_B h_{\mathcal{I}, B} $ as a 6D vector isomorphic to $\se(3)^*$, i.e., ${}_B h_{\mathcal{I}, B}  \in \mathbb{R}^6 \cong \se(3)^*$ and it writes as 
\begin{equation}
    \label{eq:spatial_momentum-rigid_body}
    {}_B h_{\mathcal{I}, B} := \AdM^*_{{}^B H _ \mathcal{I}} {}_B \mathbb{M} _ B {}^B \mathrm{v}_{\lieGroup{I}, B}.
\end{equation}
By means of the 6D spatial momentum, Equation~\eqref{eq:eurel-poincare-equations-lie-se3-force-inertial} can be rewritten as~\citep{Featherstone2014}.
\begin{equation}
    \label{eq:centroidal_spatial_momentum-rigid_body}
    {}_{\mathcal{I}} \dot{h}_{\mathcal{I}, B}  = {}_\lieGroup{I} X ^B {}_B \mathbb{M} _ B \begin{bmatrix} 
      {}^\lieGroup{I} R _ B ^\top g \\ 
      0_{3\times1}
     \end{bmatrix} + {}_\lieGroup{I} \mathrm{f}.
\end{equation}
Equation~\eqref{eq:centroidal_spatial_momentum-rigid_body} is also known as \emph{Newton-Euler equations} for a rigid body.