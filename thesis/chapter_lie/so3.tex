\section{The Rotation group \label{sec:so3}}
The set of rotation matrices $\SO(3)$ represents the set of $\mathbb{R}^{3 \times 3}$ orthogonal matrices with determinant equal to one, namely
\begin{equation}
\label{eq:so3_definition}
\SO(3) :=  \{\, R \in \mathbb{R}^{3 \times 3} \mid R^\top R = I_3 , \hspace{0.3em} \operatorname{det}(R) = 1 \,\}.
\end{equation}
It is worth noting that $\SO(3)$ is a \emph{group} under the product operator, i.e., ($\SO(3), \cdot$). Indeed, the product of two rotation matrices is still a rotation matrix. 
Its identity element is the $3\times3$ identity matrix $I_3$ and for each element $R\in \SO(3)$ there exists the inverse of $R$, i.e., $R^{-1} = R^\top \in \SO(3)$.
\par
Given two frames $A$ and $B$, an element ${}^A R _B \in \SO(3)$ denotes the coordinate transformation from frame $B$ to $A$. ${}^A R _B$ only depends on the relative orientation between the frame orientations $[A]$ and $[B]$.
\par
Given a 3D vector ${}^B p$ whose coordinates are expressed in the frame $B$, applying ${}^A R _B$ to ${}^B p$ results in the coordinate transformation from the frame $B$ to the frame $A$, namely ${}^A p = {}^A R _B {}^B p$ -- a generalization of this concept is described in Appendix~\ref{sec:action_lie}.

\subsection{Angular velocity\label{sec:angular_velocity}}
Given a smooth trajectory $R(t) \in \SO(3)$ with $t \in \mathbb{R}$ such that $R(0) = I_3$, we define the time derivative of the rotation matrix as $\dot{R}(t)$.
Given the $\SO(3)$ orthogonality condition, $R(t) R(t)^\top = I_3$, the time derivative writes as
\begin{equation}
    \label{eq:so3_derivative}
    \dot{R}(t) R(t)^\top + R(t) \dot{R}(t)^\top = 0_{3\times3}, 
\end{equation}
which can be rearranged as $\dot{R}(t) R(t)^\top = -\left(  \dot{R}(t) R(t)^\top \right)^\top$. It is worth noting that $\dot{R}(t) R(t)^\top$ is a skew-symmetric matrix, and thus \eqref{eq:so3_derivative} can be rewritten as
\begin{equation}
    \label{eq:so3_derivative_skew}
    \dot{R}(t) =  (\omega(t) \times)  R(t),
\end{equation}
where $\omega(t) \times$ belongs is a skew-symmetric matrix of the form
\begin{equation}
\label{eq:skew_simmetric_angular_velocity}
    \omega \times = \begin{bmatrix}
        0 & -\omega_z & \omega_y \\
         \omega_z & 0 & -\omega_x \\
         -\omega_y & \omega_x & 0
    \end{bmatrix}.
\end{equation}
If $R(t) = I_3$, Equation \eqref{eq:so3_derivative_skew} becomes $\dot{R}(t) = \omega\times$. 
Hereafter we denote the set of the 3D skew-symmetric matrices with $\so(3)$:
\begin{equation}
    \so(3) :=  \left\{\, S \in \mathbb{R}^{3 \times 3}  \mid S^T = -S \,\right\}.
\end{equation}
We notice that an element of $\so(3)$ is uniquely identified by a 3D vector -- see Equation~\eqref{eq:skew_simmetric_angular_velocity}. We now introduce \emph{hat} as 
\begin{equation}
    \omega ^\wedge := \omega \times.
\end{equation}
The inverse of the \emph{hat} map is denoted as \emph{vee} and it is defined as
\begin{equation}
    \left(\omega \times\right)^\vee := \omega.
\end{equation}
A more rigorous definition of the \emph{hat} and \emph{vee} operators is provided in Equation~\eqref{eq:hat_vee_operators}.


Given a smooth trajectory ${}^A R(t) _B \in \SO(3)$ representing a time-varying coordinate transformation from the frame $B$ to the frame $A$. 
\par
We define \emph{right trivialized angular velocity}, denoted by ${}^A \omega _{A,B} \times \in \so(3)$ as
\begin{equation}
    \label{eq:so3_right}
    {}^A \omega _{A,B} \times := {}^A \dot{R} _B  {}^A R _B ^\top.
\end{equation}
The \emph{right trivialized angular velocity} is equal to the angular velocity of the frame $A$ with respect to the frame $B$ whose coordinates are expressed in $A$.
\par
We define \emph{left trivialized angular velocity}, denoted with ${}^B \omega _{A,B} \times \in \so(3)$ as
\begin{equation}
    \label{eq:so3_left}
    {}^B \omega _{A,B} \times := {}^A R _B ^\top  {}^A \dot{R} _B. 
\end{equation}
The \emph{left trivialized angular velocity} is equal to the angular velocity of the frame $A$ with respect to the frame $B$ whose coordinates are expressed in $B$.
\par
It is worth recalling that the right and left trivialized angular velocities can also be computed by time differentiating the orthogonality conditions of the form ${}^A R _B  {}^A R _B ^\top = I_3$ and ${}^A R _B ^\top {}^A R _B = I_3$, respectively. 




\subsection{Exponential and Logarithmic map\label{sec:so3_exponential_log}}
Given a unit vector\footnote{Given a vector $v\in\mathbb{R}^n$ we call it unit vector if its norm is equal to 1. i.e., $v^\top v = 1$.} $u\in\mathbb{R}^3$ and an angle $\theta\in\mathbb{R}$ the associated rotation matrix is given by the exponential map as
\begin{equation}
    \label{eq:exp_so3}
    R = \Exp(\theta u) := I_3 + \sin(\theta) (u \times) + (1 - \cos(\theta)) (u \times) ^2,
\end{equation}
where \eqref{eq:exp_so3} is the well-known \emph{Rodrigues' Rotation Formula}~\citep{Murray1994}.
\par
Similarly, the logarithmic map is given by
\begin{equation}
    \label{eq:log_so3}
    \theta u = \Log(R) := \frac{\theta(R - R^\top) ^ \vee}{ 2 \sin(\theta)}; \quad \theta = \cos^{1} \left(\frac{\tr(R) - 1 }{2}\right).
\end{equation}
Where $\tr(R)$ represents the trace of the matrix $R$. It is worth noting that the $\Log$ map applied on a rotation matrix returns the axis times the angle of the axis-angle representation of the rotation. 
A detailed and more rigorous definition of the Exponential and Logarithmic operators is discussed in Appendix~\ref{sec:exp_and_log_maps}



\subsection{The adjoint representation}
Considering the left and right trivialized angular velocities ${}^B \omega _{A,B} \times \in \so(3)$ and ${}^A \omega_{A,B} \times \in \so(3)$ introduced in Equation~\eqref{eq:so3_left} and \eqref{eq:so3_right}, respectively. We introduce the adjoint representation as $\Ad_{{}^A R _B}\left({}^B \omega _{A,B} \times \right)$
\begin{equation}
   {}^A \omega  _{A,B} \times =  \Ad_{{}^A R _B}\left({}^B \omega _{A,B} \times \right) = {}^A R _B \left({}^B \omega  _{A,B} \times\right) {}^A R _B^{\top}.
\end{equation}
A more general definition of the adjoint representation is presented in Equation~\eqref{eq:adjoint_action_def},
In other words, the adjoint representation $\Ad_{{}^A R _B}\left({}^B \omega _{A,B}\times\right)$ maps a left trivialized angular velocity into a right trivialized one, i.e., a body frame velocity into an inertial frame velocity. 
As discussed in detail in Appendix~\ref{sec:adjoint_and_coadjoint_representation}, it is possible to show that the adjoint representation is a linear transformation, and as a consequence it can be uniquely represented by a matrix. We define such a matrix as \emph{adjoint matrix} and we denote it with $\AdM_{{}^A R _B}$:
\begin{equation}
    \AdM_{{}^A R _B} = {{}^A R _B}.
\end{equation}
Indeed, we can prove the identity $R( \omega\times) R^\top = (R \omega)\times$ by letting the right-hand side term act upon an arbitrary vector $v\in\mathbb{R}^3$
\begin{IEEEeqnarray}{LL}
\phantomsection \IEEEyesnumber  \IEEEyessubnumber*
    \left[(R\omega) \times\right] v &= (R\omega) \times v  \\
    &= (R\omega) \times (R R^\top v) \\
    &= R \left[ \omega \times (R^\top v) \right] \\
    &= R (\omega\times) R^\top v,
\end{IEEEeqnarray}

where we use the fact that for any rotation matrix $R$ and vectors $a$ and $b$ we have $(R a) \times (R b) = R(a \times b)$.





