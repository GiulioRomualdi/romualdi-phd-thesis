\section{The centroidal moment pivot\label{sec:cmp}}
Given a multi-body system that interacts with its surroundings. Consider a frame that is rigidly coupled to the contact surface $B:=(o_B, [B])$ and a frame that is positioned on the CoM. The origin of $G[B]:=(x_\text{CoM}, [B])$ is at the center of mass $x_\text{CoM}$, and is oriented as $B$. The Centroidal Moment Pivot (CMP)~\citep{Popovic2005} is defined as the point where a line parallel to the contact force, passing through the CoM, intersects the contact surface. More formally, we define the CMP, denoted as $x_\text{CMP}$, as
\begin{equation}
\label{eq:cmp_definition}
    x_\text{CMP} \in \mathbb{R}^3\quad \text{such that } ({}^B x_\text{CMP} - {}^B x_\text{CoM}) \times {}_B f = 0, \; e_3 ^ \top {}^B x_\text{CMP} = 0.
\end{equation}
Expanding Equation~\eqref{eq:cmp_definition}, the components ${}^B x_\text{CMP}$ write as
\begin{IEEEeqnarray}{RCL}
	\IEEEyesnumber \phantomsection \label{eq:cmp_definition_explicit}
	{}^B x_{\text{CMP}_x} &=& {}^B x_{\text{CoM}_x} - \frac{{}_B f_x}{{}_B f_z} x_{\text{CoM}_z} , \IEEEyessubnumber\\
	{}^B  x_{\text{CMP}_y} &=& {}^B x_{\text{CoM}_y} - \frac{{}_B f_y}{{}_B f_z} x_{\text{CoM}_z}, \IEEEyessubnumber \\
	{}^B  x_{\text{CMP}_z} &=& 0. \IEEEyessubnumber
\end{IEEEeqnarray}
Combining Equation~\eqref{eq:zmp_centroidal_relationship} with the CMP definition~\eqref{eq:cmp_definition_explicit} we can express the CMP in terms of ZMP location, rate of change of the centroidal angular momentum, and the ground reaction force as
\begin{IEEEeqnarray}{RCL}
	\IEEEyesnumber \phantomsection \label{eq:cmp_definition_explicit_zmp}
	{}^B x_{\text{CMP}_x} &=& {}^B x_{\text{ZMP}_x} + \frac{{}_{G[B]} \dot{h}^\omega_y}{{}_B f_z} , \IEEEyessubnumber\\
	{}^B  x_{\text{CMP}_y} &=& {}^B x_{\text{ZMP}_y} - \frac{{}_{G[B]} \dot{h}^\omega_x}{{}_B f_z}, \IEEEyessubnumber \\
	{}^B  x_{\text{CMP}_z} &=& 0. \IEEEyessubnumber
\end{IEEEeqnarray}
To give the reader a better comprehension, we can image the CMP as the point where the ground reaction
force would have to act to keep the horizontal component of the whole-body angular momentum constant. When the centroidal angular momentum is constant, i.e., ${}_{G[B]} \dot{h}^\omega = 0_{3\times 1}$ the CMP coincides with the ZMP -- Figure~\ref{fig:cmp-zmp}. Moreover, while by definition the ZMP cannot leave the contact domain $\Omega$, the CMP, in the case of ${}_{G[B]} \dot{h}^\omega_x \neq 0$  or ${}_{G[B]} \dot{h}^\omega_y \neq 0$, can.
Let us assume that the motion of the multi-body system is approximated by the LIPM, i.e., the hypothesis in Section~\ref{sec:lip} holds, then it is possible to prove that the position of the CMP coincides with the ZMP.
To prove this statement, it is worth noticing that when the hypotheses of the LIPM are satisfied the centroidal angular momentum is kept constant -- Section~\ref{sec:lip}, hence, substituting ${}_{G[B]} \dot{h}^\omega$ into Equation~\eqref{eq:cmp_definition_explicit_zmp}, we can conclude that if the system is approximated by the LIPM, then the CMP and ZMP coincide.
