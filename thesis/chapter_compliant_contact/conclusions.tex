\section{Conclusions \label{sec:conclusions_compliant}}
This chapter presents the development of a contact model to represent the interaction between the robot and the environment. Unlike state-of-the-art models, we consider the environment as a continuum of spring-damper systems. This allows us to compute the equivalent contact wrench by integrating the pressure distribution over the contact surface. As a result, rotational springs and dampers are not required to model the interaction between the robot and the environment. We also develop a whole-body controller that stabilizes the robot while walking in a compliant environment. Finally, an estimation algorithm is also introduced to compute the contact parameters in real-time. The proposed controller is then compared with the whole-body controller presented in Chapter~\ref{chapter:benchmarking_wbc}. We analyze the robustness properties of the architecture with respect to non-parametric uncertainty in the contact model. Finally, we study the performance of the contact parameter estimation in the case of an anisotropic environment. 
It is worth mentioning that the controller introduced in this chapter assumes that the flexibility is located in the surrounding environment, while the robot is considered rigid. In Chapter~\ref{chapter:flexible_joints} we will consider that the robot link deforms during the locomotion task, while the environment is considered rigid. We characterize the link flexibility by introducing equivalent passive joints where the link deflection is concentrated.