\chapter{Proof of Lemma~\ref{lemma:compliant_model}\label{appendix:proof_lemma_compliant}} 

Let us now consider a rigid body that makes a contact with a visco-elastic surface, and we assume that:
\begin{enumerate}
    \item there exists an inertial frame $\mathcal{I}$;
    \item there exists a frame $B$ rigidly attached to the body, and we denote $o_B$ the origin of the frame and $[B]$ its orientation;
    \item there exists a contact domain $\Omega \in \mathbb{R}^3$, we denote with ${}^B x$ a point in the contact surface expressed in the the frame $B$;
    \item the characteristics of the environment are isotropic;
    \item while in contact, the rigid body moves with a 6D velocity, denoted as ${}^{B[\mathcal{I}]} \mathrm{v}$ such that 
    \begin{equation}
    {}^{B[\mathcal{I}]} \mathrm{v} = 
    \begin{bmatrix}
       {}^\mathcal{I}\dot{o}_B \\
       {}^\mathcal{I}\omega_{\mathcal{I},B}
    \end{bmatrix} ;
    \end{equation} 
    \item  $\forall x \in \Omega$, there exists a continuous pure force distribution that depends on the point ${}^\mathcal{I} x$ and its velocity ${}^\mathcal{I} \dot{x}$ expressed in the inertial frame, as in \eqref{eq:contact_model_general}
    \begin{equation}
        \label{eq:contact_model_general_appendix}
        \rho\left({}^\mathcal{I} x, {}^\mathcal{I} \dot{x}\right) = k \left( {}^\mathcal{I} \bar{x} - {}^\mathcal{I} x\right) - b \; {}^\mathcal{I} \dot{x},
    \end{equation}
    where $\bar{x} \in \bar{\mathcal{X}}$.
\end{enumerate}
We now introduce $u$ and $v$ as the coordinates, in the body frame $B$, of the point belongs to the contact surface ${}^ B x$, such as 
\begin{equation}
    {}^ B x = \begin{bmatrix} u & v & 0 \end{bmatrix}^\top.
\end{equation}
The position of the contact point in the inertial frame, denoted as ${}^\mathcal{I} x$, is given by applying the homogeneous transformation ${}^\mathcal{I} H _ B$ to ${}^B x$ as
\begin{equation}
{}^ \mathcal{I} x = {}^\mathcal{I} H _ B {}^B x = o_B + \prescript{\mathcal{I}}{} R _ B\begin{bmatrix} u & v & 0 \end{bmatrix} ^\top,
\label{eq:contact_point_position}
\end{equation}
The contact point velocity ${}^ \mathcal{I} \dot{x}$ is given by time differentiating Equation~\eqref{eq:contact_point_position}:
\begin{equation}
{}^ \mathcal{I} \dot{x} = \dot{o}_B +  \left({}^{\mathcal{I}} \omega _ {\mathcal{I},B} \times\right) {}^\mathcal{I} R _ B \begin{bmatrix} u & v & 0 \end{bmatrix} ^\top,
\label{eq:contact_point_velocity}
\end{equation}
Using the hypothesis of rigid-body, $\bar{x}$ can be computed as:
\begin{equation}
{}^\mathcal{I} \bar{x} = \bar{o}_B + {}^\mathcal{I} \bar{R} _ B \begin{bmatrix} u & v & 0 \end{bmatrix} ^\top.
\label{eq:null_force_point}
\end{equation}
Here, $\bar{o}_B$ and ${}^\mathcal{I} \bar{R} _ B$ are the position and rotation of the body frame associated with a null force in the case of zero velocity.
\par
Combining \eqref{eq:contact_model_general_appendix} with \eqref{eq:contact_point_position}, \eqref{eq:contact_point_velocity} and \eqref{eq:null_force_point}, the force acting on a point lying on the contact surface becomes:
\begin{IEEEeqnarray}{LL}
\IEEEyesnumber \phantomsection
\label{eq:contact_model}
\rho &= k \left\{ \bar{o}_B - o_B + \left(\prescript{\mathcal{I}}{}{\bar{R}}_B - \prescript{\mathcal{I}}{}R _B\right) \begin{bmatrix} u & v & 0 \end{bmatrix}^\top \right\} \IEEEyessubnumber \label{eq:contact_model_spring}\\
&- b \left\{ \dot{o}_B + \left({}^{\mathcal{I}} \omega _ {\mathcal{I},B}\times\right) \prescript{\mathcal{I}}{}R _B \begin{bmatrix} u & v & 0 \end{bmatrix}^\top \right\}. \IEEEyessubnumber \label{eq:contact_model_damper}
\end{IEEEeqnarray}
where \eqref{eq:contact_model_spring} is the force generated by the spring and \eqref{eq:contact_model_damper} is the one produced by the damper. 
To facilitate the process of finding the solutions to the integrals~\eqref{eq:compliant-contact-contact-wrench}, let us recall that given
a double integral of a function $g(x, y)$, a variable change of the form \eqref{eq:contact_point_position}, \eqref{eq:contact_point_velocity} and \eqref{eq:null_force_point} yields
\begin{equation}
    \int\limits\!\!\!\! \int\limits g(x, y) \diff x \diff y = \int\limits\!\!\!\! \int\limits g\left(x(u,v), y(u,v)\right) |\det(J)| \diff u \diff v,
    \label{eq:integral_rule}
\end{equation}
where $J$ is the Jacobian of the variable transformation, i.e. 
\begin{equation}
    J = \begin{bmatrix}
       \dfrac{\partial x_1}{\partial u} & & \dfrac{\partial x_1}{\partial v} \\ 
       \\
      \dfrac{\partial x_2}{\partial u} & & \dfrac{\partial x_2}{\partial v} 
    \end{bmatrix}.
\end{equation}
Here, ${}_k$ operator extracts the $k$ element of a vector, that is, $x_k =  e_k^\top x$.
\par
Given the variable change defined in \eqref{eq:contact_point_position} and writing $\prescript{\mathcal{I}}{}R _B$ as the horizontal concatenation of three vectors $i$, $j$, $n$. It is straightforward to verify that $|\det(J)|$ is equal to 
\begin{equation}
    |\det(J)|  =  | i_1 j_2 - i_2 j_1 | = | n^\top e_3 | = | e_3 ^\top \prescript{\mathcal{I}}{}R _B e_3 | .
    \label{eq:jacobian_compliant}
\end{equation}

\section{Compliant contact force computation}
Equations \eqref{eq:integral_rule}, \eqref{eq:contact_model} and ~\eqref{eq:jacobian_compliant} can be used to evaluate the total force applied from the environment to a generic contact surface as

\begin{IEEEeqnarray}{LL}
\IEEEyesnumber \label{eq:integral_compliant_explict} \phantomsection
    {}_\mathcal{I} f &=  k | e_3 ^\top \prescript{\mathcal{I}}{}R _B e_3 | \int\limits\!\!\!\! \int\limits  \left\{ \bar{o}_B - o_B + \left(\prescript{\mathcal{I}}{}{\bar{R}}_B - \prescript{\mathcal{I}}{}R _B\right) \begin{bmatrix} u&v&0 \end{bmatrix}^\top \right\} \diff u \diff v \IEEEyessubnumber\\
    &-b| e_3 ^\top \prescript{\mathcal{I}}{}R _B e_3 |\int\limits\!\!\!\! \int\limits  \left\{ \dot{o}_B + \left({}^{\mathcal{I}} \omega _ {\mathcal{I},B}\times\right) \prescript{\mathcal{I}}{}R _B \begin{bmatrix} u&v&0\end{bmatrix}  ^\top \right\} \diff u \diff v \IEEEyessubnumber.
\end{IEEEeqnarray}

If $\Omega$ is represented by a rectangle with a length $l$ and a width $w$, the contact force ${}_\mathcal{I} f$ in ~\eqref{eq:integral_compliant_explict} writes as  
\begin{IEEEeqnarray}{LL}
\IEEEyesnumber \label{eq:integral_compliant_explict_rectangle} \phantomsection
    {}_\mathcal{I} f &=  k | e_3 ^\top \prescript{\mathcal{I}}{}R _B e_3 | \int\limits_{-l/2}^{l/2} \int\limits_{-w/2}^{w/2}  \bar{o}_B - o_B \diff u \diff v \label{eq:integral_compliant_explict_rectangle_spring_linear} \IEEEyessubnumber \\
    &+ k | e_3 ^\top \prescript{\mathcal{I}}{}R _B e_3 | \int\limits_{-l/2}^{l/2} \int\limits_{-w/2}^{w/2}  \left(\prescript{\mathcal{I}}{}{\bar{R}}_B - \prescript{\mathcal{I}}{}R _B\right) \begin{bmatrix} u&v&0 \end{bmatrix}^\top  \diff u \diff v \label{eq:integral_compliant_explict_rectangle_spring_angular} \IEEEyessubnumber\\
    &-b| e_3 ^\top \prescript{\mathcal{I}}{}R _B e_3 |\int\limits_{-l/2}^{l/2} \int\limits_{-w/2}^{w/2}    \dot{o}_B \diff u \diff v  \label{eq:integral_compliant_explict_rectangle_dumper_linear} \IEEEyessubnumber \\
    &- b| e_3 ^\top \prescript{\mathcal{I}}{}R _B e_3 |\int\limits_{-l/2}^{l/2} \int\limits_{-w/2}^{w/2}  \left({}^{\mathcal{I}} \omega _ {\mathcal{I},B}\times\right) \prescript{\mathcal{I}}{}R _B \begin{bmatrix} u&v&0\end{bmatrix}  ^\top  \diff u \diff v \label{eq:integral_compliant_explict_rectangle_dumper_angular} \IEEEyessubnumber.
\end{IEEEeqnarray}
In Equation~\eqref{eq:integral_compliant_explict_rectangle} we can recognize two common structures. The integrand functions in \eqref{eq:integral_compliant_explict_rectangle_spring_linear} and \eqref{eq:integral_compliant_explict_rectangle_dumper_linear} are constants. On the other hand, the integrands in \eqref{eq:integral_compliant_explict_rectangle_spring_angular} and \eqref{eq:integral_compliant_explict_rectangle_dumper_angular} are odd functions: 
\begin{IEEEeqnarray}{C}
\IEEEyesnumber \phantomsection
    \Gamma\sim \bar{o}_B - o_B \sim \dot{o}_B,   \IEEEyessubnumber \\
    \Xi(u,v) \sim \left(\prescript{\mathcal{I}}{}{\bar{R}}_B - \prescript{\mathcal{I}}{}R _B\right) \begin{bmatrix} u&v&0 \end{bmatrix}^\top  \sim \left({}^{\mathcal{I}} \omega _ {\mathcal{I},B}\times\right) \prescript{\mathcal{I}}{}R _B \begin{bmatrix} u&v&0\end{bmatrix}  ^\top.  \IEEEyessubnumber
\end{IEEEeqnarray}

The integral of the constant term $\Gamma$ in the rectangle contact domain is given by
\begin{equation}
   \int\limits_{-l/2}^{l/2} \int\limits_{-w/2}^{w/2}  \Gamma \diff u \diff v = \Gamma \int\limits_{-l/2}^{l/2} \int\limits_{-w/2}^{w/2}   \diff u \diff v  =  l w \Gamma,
\end{equation}
on the other hand, since the integration domain is symmetric, the integral of the odd term $\Xi(u,v)$ is equal to the zero vector.
To conclude, the equivalent contact force ${}_\mathcal{I} f$ writes as~\eqref{eq:contact_force_integral_rectangle}
\begin{equation}
    {}_\mathcal{I} f = k (\bar{o}_B - o_B) - b\dot{o}_B.
\end{equation}
\par

\section{Compliant contact torque computation}
The torque about the origin of $B$, $o_B$ produced by the force distribution $\rho$ is:
\begin{equation}
\label{eq:contact_torque}
\sigma_{o_B}(u,v) = \left(\prescript{\mathcal{I}}{}R _B \begin{bmatrix} u &
 v &
 0 \end{bmatrix} ^\top
  \right) \times \rho(u,v),
\end{equation}
Applying~\eqref{eq:integral_rule}, the integral of~\eqref{eq:contact_torque} on a rectangular contact surface leads to
\begin{IEEEeqnarray}{LL}
 \label{eq:integral_compliant_explict_torque} \phantomsection  \IEEEyesnumber \IEEEyessubnumber*
    {}_{B[\mathcal{I}]} \mu &=  k | e_3 ^\top \prescript{\mathcal{I}}{}R _B e_3 |   \int\limits_{-l/2}^{l/2}  \int\limits_{-w/2}^{w/2}  \left(\prescript{\mathcal{I}}{}R _B 
    \begin{bmatrix} u \\ v \\ 0 
    \end{bmatrix}
  \right) \times  \left(  \bar{o}_B - o_B \right)\diff u \diff v  \label{eq:integral_compliant_explict_torque_linear_spring} \IEEEyessubnumber\\
  &+  k | e_3 ^\top \prescript{\mathcal{I}}{}R _B e_3 | \int\limits_{-l/2}^{l/2}  \int\limits\limits_{-w/2}\limits^{w/2}  \left(\prescript{\mathcal{I}}{}R _B \begin{bmatrix} u \\ v \\ 0 \end{bmatrix}
  \right) \times   \left(\prescript{\mathcal{I}}{}{\bar{R}}_B - \prescript{\mathcal{I}}{}R _B\right) \begin{bmatrix} u \\ v \\ 0 \end{bmatrix} \diff u \diff v \label{eq:integral_compliant_explict_torque_angular_spring} \IEEEyessubnumber\\
    & -b | e_3 ^\top \prescript{\mathcal{I}}{}R _B e_3 | \int\limits_{-l/2}^{l/2}  \int\limits\limits_{-w/2}\limits^{w/2}  \left(\prescript{\mathcal{I}}{}R _B \begin{bmatrix} u \\ v \\ 0 \end{bmatrix}
  \right) \times   \dot{o}_B \diff u \diff v \label{eq:integral_compliant_explict_torque_linear_damper} \IEEEyessubnumber\\ 
   & -b | e_3 ^\top \prescript{\mathcal{I}}{}R _B e_3 | \int\limits_{-l/2}^{l/2}  \int\limits\limits_{-w/2}\limits^{w/2}  \left(\prescript{\mathcal{I}}{}R _B \begin{bmatrix} u \\ v \\ 0 \end{bmatrix}
  \right) \times \left({}^{\mathcal{I}} \omega _ {\mathcal{I},B}\times\right) \prescript{\mathcal{I}}{}R _B \begin{bmatrix} u\\v\\0\end{bmatrix}   \diff u \diff v \label{eq:integral_compliant_explict_torque_angular_damper} \IEEEyessubnumber.
\end{IEEEeqnarray}
In Equation~\eqref{eq:integral_compliant_explict_torque} we can recognize two common structures. The integral terms in \eqref{eq:integral_compliant_explict_torque_linear_spring} and \eqref{eq:integral_compliant_explict_torque_linear_damper} are linear odd functions on the integral variables $u$ and $v$. On the other hand the integrands in \eqref{eq:integral_compliant_explict_torque_angular_spring} and \eqref{eq:integral_compliant_explict_torque_angular_damper} can be rewritten as
\begin{equation}
    \label{eq:integrand_compliant}
    \left(\prescript{\mathcal{I}}{}R _B \begin{bmatrix} u \\ v \\ 0 \end{bmatrix}
  \right) \times \mathcal{A} \begin{bmatrix} u\\v\\0\end{bmatrix},
\end{equation}
where $\mathcal{A}$ is equal to $\mathcal{A} = \prescript{\mathcal{I}}{}{\bar{R}}_B - \prescript{\mathcal{I}}{}R _B$
in Equation~\eqref{eq:integral_compliant_explict_torque_angular_spring}, and $\mathcal{A} = \left({}^{\mathcal{I}} \omega _ {\mathcal{I},B}\times\right)\prescript{\mathcal{I}}{}R _B$ in Equation~\eqref{eq:integral_compliant_explict_torque_angular_damper}.
\par
We notice that since the integration domain is symmetric, the integral of the odd terms \eqref{eq:integral_compliant_explict_torque_linear_spring} and \eqref{eq:integral_compliant_explict_torque_linear_damper} is equal to zero. 
\par
We now aim to compute the following integral 
\begin{equation}
    \int\limits_{-l/2}^{l/2}  \int\limits\limits_{-w/2}\limits^{w/2} \left(\prescript{\mathcal{I}}{}R _B \begin{bmatrix} u \\ v \\ 0 \end{bmatrix}
  \right) \times \mathcal{A} \begin{bmatrix} u\\v\\0\end{bmatrix} \diff u \diff v.
\end{equation}
Let us rewrite the rotation matrix $\prescript{\mathcal{I}}{}R _B$ and the matrix $\mathcal{A}$ as the column concatenation of the following vectors
\begin{equation}
    \prescript{\mathcal{I}}{}R _B = \begin{bmatrix}
    i &j & k
    \end{bmatrix} \quad \quad 
    \mathcal{A} = \begin{bmatrix}
    a & b & c
    \end{bmatrix}.
\end{equation}
Then the integrand function~\eqref{eq:integrand_compliant} can be rewritten as 
\begin{IEEEeqnarray}{LLL}
\phantomsection  \IEEEyesnumber \IEEEyessubnumber*
    \left(\prescript{\mathcal{I}}{}R _B \begin{bmatrix} u \\ v \\ 0 \end{bmatrix}
  \right) \times \mathcal{A} \begin{bmatrix} u\\v\\0\end{bmatrix} &=& (u i + v j) \times (u a + v b) \\ 
&=&  u ^2 (i \times a) + v^2(j\times b) \label{eq:integrand_compliant_revwritten_not_odd}\\
&&+ u v [(i \times b) + (j \times a)] \label{eq:integrand_compliant_revwritten_odd}.
\end{IEEEeqnarray}
The term~\eqref{eq:integrand_compliant_revwritten_odd} is an odd function, and as a consequence the integral in the symmetric domain is equal to zero.  
On the other hand, the integral \eqref{eq:integrand_compliant_revwritten_not_odd} is given by
\begin{IEEEeqnarray}{lr}
\phantomsection  \IEEEyesnumber \IEEEyessubnumber*
    \int\limits_{-l/2}^{l/2}  \int\limits\limits_{-w/2}\limits^{w/2} \left(\prescript{\mathcal{I}}{}R _B \begin{bmatrix} u \\ v \\ 0 \end{bmatrix}
  \right) \times \mathcal{A} \begin{bmatrix} u\\v\\0\end{bmatrix} \diff u \diff v &\;\;= \\
  \int\limits_{-l/2}^{l/2}  \int\limits\limits_{-w/2}\limits^{w/2} u ^2 (i \times a) + v^2(j\times b) \diff u \diff v &= \\
  \frac{l w}{12} \{l^2 (i \times a) + w^2 (j \times b)\} &= \\
  \frac{l w}{12} \left\{l^2 \left(\prescript{\mathcal{I}}{}R _B e_1\right) \times \left(\mathcal{A} e_1\right) + w^2 \left(\prescript{\mathcal{I}}{}R _B e_2\right) \times \left(\mathcal{A} e_2\right)\right\}& \label{eq:integrand_compliant_revwritten_not_odd_final}.
\end{IEEEeqnarray}
Recalling that $\mathcal{A}$ is equal to $\mathcal{A} = \prescript{\mathcal{I}}{}{\bar{R}}_B - \prescript{\mathcal{I}}{}R _B$
in Equation~\eqref{eq:integral_compliant_explict_torque_angular_spring}, and $\mathcal{A} = \left({}^{\mathcal{I}} \omega _ {\mathcal{I},B}\times\right)\prescript{\mathcal{I}}{}R _B$ in Equation~\eqref{eq:integral_compliant_explict_torque_angular_damper}, the solution of the integral~\eqref{eq:integral_compliant_explict_torque} becomes
\begin{IEEEeqnarray}{LLL}
 \label{eq:integral_compliant_explict_torque_final} \phantomsection  \IEEEyesnumber \IEEEyessubnumber*
    {}_{B[\mathcal{I}]} \mu &=  \frac{klw}{12} | e_3 ^\top \prescript{\mathcal{I}}{}R _B e_3 | &
    \left\{ l^2 \left( \prescript{\mathcal{I}}{}R _B e_1 \right) \times \left[ \left( \prescript{\mathcal{I}}{}{\bar{R}} _B - \prescript{\mathcal{I}}{}R _B  \right)e_1\right] + \right.
  \IEEEyessubnumber\\
  & & \left. w^2 \left( \prescript{\mathcal{I}}{}R _B e_2 \right) \times \left[ \left( \prescript{\mathcal{I}}{}{\bar{R}} _B - \prescript{\mathcal{I}}{}R _B  \right)e_2\right] \right\} + \\
   & -\frac{blw}{12} | e_3 ^\top \prescript{\mathcal{I}}{}R _B e_3 | &
   \left\{ l^2 \left( \prescript{\mathcal{I}}{}R _B e_1 \right) \times \left[ \left( \left(  {}^{\mathcal{I}} \omega _ {\mathcal{I},B}\times \right) \prescript{\mathcal{I}}{}R _B \right) e_1\right] + \right.
  \IEEEyessubnumber\\
  & & \left. w^2 \left( \prescript{\mathcal{I}}{}R _B e_2 \right) \times \left[ \left( \left(  {}^{\mathcal{I}} \omega _ {\mathcal{I},B}\times \right) \prescript{\mathcal{I}}{}R _B \right) e_2\right] \right\}
  \IEEEyessubnumber.
\end{IEEEeqnarray}
By applying some basic rules of the cross product, Equation~\eqref{eq:integral_compliant_explict_torque_final} leads to 
\eqref{eq:contact_torque_integral_rectangle}.