\section{System modeling\label{sec:flex_joint_system_modeling}}

TALOS's hip flexibility has a significant impact on its leg control and, as a result, its balance and locomotion~\citep{Ramuzat2021ComparisonTALOS}. In this section, we model the TALOS's hip flexibility by means of underactuated joints. The section also extends the humanoid robot model dynamics presented in Section~\ref{sec:multi-body-dynamics} to consider the robot's link visco-elasticity.

\subsection{Model of the hip flexibility}

Following the work of~\cite{Villa2022TorqueFlexibility}, we model the flexibility by introducing two passive virtual joints between the base link and each leg. The virtual joints simulate the motion caused by the visco-elastic deformation of the waist-leg connection, where the link cross section is reduced.
Given the $i$-th passive joint, we assume that it exerts a torque $\tau^f_i$ that depends on the
joint deflection $s^f_i$ and its velocity $\dot{s}^f_i$~\citep{Nakaoka2007Constraint-basedMechanisms} as
\begin{equation}
\label{eq:flexible_joint}
\tau^f_i = -k_i s^f_i - d_i \dot{s}^f_i.
\end{equation}
where $k_i$ and $d_i$ are, respectively, the stiffness and damping coefficients of the flexible
joint $i$. By assuming to model the link flexibility with $n_f$ joints, we can consider the robot to
have $n$ joints where $n = n_a + n_f$ with $n_a$ are the actuated joints.
\par
In the specific case of the TALOS robot, the authors of \citep{Villa2022TorqueFlexibility} notice that
the stiffness is due to the vertical linkage and, consequently, they model the flexibility along
the pitch and roll axis, only. In this chapter, the same consideration holds, so we introduce two
passive flexible joints for each leg. As a result $n_f = 4$ while $n_a = 32$ -- see Section~\ref{sec:talos}.
\par
Assuming that it is possible to estimate $\tau^f_i$, we approximate the flexible joint state by discretizing
Equation~\eqref{eq:flexible_joint}, resulting in
\begin{IEEEeqnarray}{C}
\phantomsection \label{eq:flexibility_discretized} \IEEEyesnumber \IEEEyessubnumber*
s^f_i[k] = \frac{d_i s^f_i[k-1] - \tau^f_i[k] \diff t}{k_i \diff t + d_i}, \label{eq:flexibility_discretized_position} \\
\dot{s}^f_i[k] = \frac{s^f_i[k] - s^f_i[k-1]}{\diff t}.
\end{IEEEeqnarray}
Here $\diff t$ is the sampling time. $s^f_i[k] = s^f_i(t_0 + k \diff t)$ and $\dot{s}^f_i[k] = \dot{s}^f_i(t_0 + k \diff t)$.


\subsection{Modeling of a floating base system with flexible joints}
This section extends the floating base system model presented in Section~\ref{sec:multi-body-dynamics} to consider underactuated flexible joints.
\par
Let us consider an inertial frame $\mathcal{I}$ and a floating base system making $n_c$ contact with the environment. We recall:
\begin{itemize}
    \item $B = (p_B, [B])$ is the frame rigidly attached to the robot base. ${}^\mathcal{I} H _B \in \SE(3)$  describes the position and orientation of $B$ with respect to the inertial frame $\mathcal{I}$.
    \item Hereafter the base velocity expressed in mixed representation ${}^{B[\mathcal{I}]}\mathrm{v}_{\mathcal{I}, B}$ such that  ${}^{B[\mathcal{I}]}\mathrm{v}_{\mathcal{I}, B}^\top = \begin{bmatrix}
    {}^{\mathcal{I}} \dot{p}_B^\top & {}^{B[\mathcal{I}]}\dot{\omega}_{\mathcal{I}, B}^\top.
    \end{bmatrix}$
    \item The actuated and flexible joint positions are indicated, respectively, with $s^a \in \mathbb{R}^{n_a}$ and $s^f \in \mathbb{R}^{n_f}$;
    \item the actuated and flexible joint torques are indicated, respectively, with $\tau^a \in \mathbb{R}^{n_a}$ and $\tau^f \in \mathbb{R}^{n_f}$.
\end{itemize}
We extend the robot configuration~\eqref{eq:robot_configuration_mixed} by introducing the flexible
joint value as $q = ({}^{\mathcal{I}} p_B, {}^{\mathcal{I}} R_{B}, s^a, s^f)$. $q$ is an element
of a Lie group $\lieGroup{Q} = \mathbb{R}^3 \times SO{(3)} \times \mathbb{R}^{n_a} \times
\mathbb{R}^{n_f}$. The associated lie Algebra writes as $\lieAlgebra{q} = \mathbb{R}^3 \times \so(3)
\times \mathbb{R}^{n_a} \times \mathbb{R}^{n_f}$. Here, we recall that $\lieAlgebra{q}$ is isomorphic to $\mathbb{R}^3 \times \mathbb{R}^3 \times \mathbb{R}^{n_a} \times \mathbb{R}^{n_f}$ -- see Appendix~\ref{sec:tangent_space_lie}.
\par
The \emph{velocity of the multi-body system} belongs to $\lieAlgebra{q}$ and we denote it with $\nu = \left({}^{\mathcal{I}} \dot{p}_B, {}^{B[\mathcal{I}]}{\omega}_{\mathcal{I}, B}, \dot{s}^a, \dot{s}^f\right)$.
\par
Slightly modifying~\eqref{eq:system}, we write the dynamics of a floating base system with flexible joints as follows 
\begin{equation}
\label{eq:system_flex}
{M}({q})\dot{{\nu}} + h({q}, {\nu}) =  \begin{bmatrix}
{0}_{6\times n_a} \\ I_{n_a} \\ 0_{n_f\times n_a}
\end{bmatrix}{\tau}^a + 
\begin{bmatrix}
{0}_{6\times n_f} \\ 0_{n_a\times n_f} \\ I_{n_f} 
\end{bmatrix}{\tau}^f +
             {J}_{\mathcal{C}}(q)^\top \mathrm{f},
\end{equation}
where
\begin{equation}
	{J}_{\mathcal{C}}({q}) = 
	\begin{bmatrix}{J}_{\mathcal{C}_1}({q}) \\ \vdots \\ {J}_{\mathcal{C}_{n_c}}({q})  \end{bmatrix}, \quad
	\mathrm{f} = \begin{bmatrix}
		{}_{\mathcal{C}_1[\mathcal{I}]}\mathrm{f}_1 \\
		\vdots\\
		{}_{\mathcal{C}_{n_c}[\mathcal{I}]}\mathrm{f}_{n_c}
	\end{bmatrix}.
\end{equation}
Recalling that $n=n_a + n_f$, $M\in \mathbb{R}^{(n+6) \times (n+6)}$ is the mass matrix,
$h \in \mathbb{R}^{(n+6)}$ accounts for Coriolis, the centrifugal effects and the gravity term.  ${}_{\mathcal{C}_k[\mathcal{I}]}\mathrm{f}_k \in \mathbb{R}^{6}$ denotes the $k$-th external wrench applied by the environment on the robot expressed in mixed representation.
The Jacobian $J_{\mathcal{C}_k}$ is the mixed velocity Jacobian of the contact $\mathcal{C}_k$.
\par
Following the same approach presented in Section~\ref{sec:multi-body-dynamics}, the
dynamics~\eqref{eq:system_flex} is expressed by separating the first $6$ rows, which refers to the underactuated floating base, from the last rows $n_a + n_f$, which refers to the actuated and flexible joints as:
\begin{IEEEeqnarray}{c}
\IEEEyesnumber \phantomsection
M_{\nu}(q) \dot{\nu} + h_{\nu} (q, \nu) =  J^\top_{{\mathcal{C}}_\nu}(q) \;  \mathrm{f},\label{eq:system_flex_base_projection} \IEEEyessubnumber \\
M_{s}(q) \dot{\nu} + h_{s} (q, \nu) = \begin{bmatrix}
I_{n_a} \\ 0_{n_f\times n_a}
\end{bmatrix}{\tau}^a +
\begin{bmatrix}
 0_{n_a\times n_f} \\ I_{n_f}
\end{bmatrix}{\tau}^f +  J^\top_{{\mathcal{C}}_s}(q)  \; \mathrm{f}. \label{eq:system_flex_joint_projection} \IEEEyessubnumber
\end{IEEEeqnarray}
The subscript $\nu$ refers to the first $6$ rows of the matrix, while $s$ refers to the last $n$
rows. Equation~\eqref{eq:system_flex_base_projection} is often denoted as the base projection of the floating base dynamics, while
Equation~\eqref{eq:system_flex_joint_projection} is the joint space projection.
