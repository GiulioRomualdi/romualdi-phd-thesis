\section{Conclusions \label{sec:conclusions_flexible_joint}}
This chapter presents the design of a whole-body QP control layer for a humanoid robot affected by link flexibility. We model the flexibility by introducing equivalent passive joints that simulate the motion caused by the link deformation.
We then considered the passive joints position and velocity as state of the floating base system dynamics. Thanks to this choice, we develop a whole-body controller that implicitly considers the joint flexibility in the stabilization problem. 
The chapter also details the design of an estimator that aims at computing the flexible joint state in real-time. 
\par
The proposed approach is validated in a simulated version of the TALOS humanoid robot, where its hip flexibility has a significant impact while performing locomotion tasks. Moreover, the architecture is then compared with a whole-body controller that considers all links of the robot rigid.
\par
As a future work, we plan to mitigate the discontinuity of the contact forces by performing a smother transition between contiguous support phases. We also plan to make a detailed comparison with other state-of-the-art controllers that
consider the flexibility of the robot link~\citep{Villa2022TorqueFlexibility}. In addition, we plan to validate the architecture on the real robot.


