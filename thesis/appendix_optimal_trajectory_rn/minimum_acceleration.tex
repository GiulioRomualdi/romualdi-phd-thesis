\section{Minimum acceleration trajectory in $\mathbb{R}^n$\label{appendix:min_acc}}
Given a set of points, also denoted as knots, associated to a scalar parameter $t$ representing the time, i.e., $(t_0, x_0)$, $(t_1, x_1)$, ..., $(t_N, x_N)$ such that $t_i \ge 0$ and $x_i  \in \mathbb{R}^n$ and given a desired initial and final velocity, $(t_0, \dot{x}_0)$ and $(t_N, \dot{x}_N)$. We aim to compute a trajectory $x : \mathbb{R}_+ \rightarrow \mathbb{R}^n$ passing to knots having the desired initial and final velocity such that the second order derivative of $x(t)$ with respect to the time, i.e., the acceleration, is minimized.
\par
More formally, we seek for a trajectory $x^*(t)$ such that is a stationary point of the action~\eqref{eq:hamilton_x_epsilon_definition_optimization} with a Lagrangian function defined as
\begin{equation}
    \label{eq:hamiltonin_lagrangian_minimum_acceleration}
    \mathcal{L}(\ddot{x}) = \ddot{x}^\top \ddot{x}.
\end{equation}
Considering the set of points $\{(t_i, x_i)\}$ and the Lagrangian function~\eqref{eq:hamiltonin_lagrangian_minimum_acceleration}, the action functional $\mathfrak{G}$~\eqref{eq:hamilton_x_epsilon_definition_optimization} writes as
\begin{equation}
    \label{eq:minimum_acceleration_general_sum}
    \mathfrak{G}_{\ddot{x}} = \sum^{N-1}_{i = 0}\int_{t_{i}}^{t_{i+1}} \ddot{x}^\top \ddot{x} \diff t.
\end{equation}
Since all the terms within the sum of~\eqref{eq:minimum_acceleration_general_sum} are positive, the minimization of $\mathfrak{G}_{\ddot{x}}$ is equivalent to the minimization of each integral term in~\eqref{eq:minimum_acceleration_general_sum}.
\par
Substituting~\eqref{eq:hamiltonin_lagrangian_minimum_acceleration} into~\eqref{eq:hamilton_dae}, we obtain
\begin{equation}
    \label{eq:hamilton_min_acc_ode}
    x^{(4)}(t) = 0,
\end{equation}
we can conclude that a trajectory $x(t)$ that satisfies~\eqref{eq:hamilton_min_acc_ode} minimizes
\begin{equation}
    \int_{t_{i}}^{t_{i+1}} \ddot{x}^\top \ddot{x} \diff t,
\end{equation}
and consequently it is a minimum acceleration trajectory in the interval $t_{i}$, $t_{i+1}$. One of the infinite solutions of~\eqref{eq:hamilton_min_acc_ode} is given by the family of the 3$^{\text{rd}}$-order polynomial functions
\begin{equation}
    \label{eq:hamilton_3rd_poly}
    x(t) = a_3 t^3 + a_2 t^2 + a_1 t + a_0.
\end{equation}
\par
The optimal trajectory that minimizes~\eqref{eq:minimum_acceleration_general_sum} is given by the concatenation of the 3$^{\text{rd}}$-order polynomial functions \eqref{eq:hamilton_3rd_poly}, where the coefficients $a_i$ are chosen to satisfy the boundary conditions of position and velocity
\begin{IEEEeqnarray}{cc}
\phantomsection   \IEEEyesnumber \IEEEyessubnumber*
  \quad x(t_i) = x_i \quad & \dot{x}(t_i) = \dot{x}_i \\
   \quad x(t_{i+1}) = x_{i+1} \quad & \dot{x}(t_{i+1}) = \dot{x}_{i+1}.
\end{IEEEeqnarray}
Given a subtrajectory 
\begin{equation}
\label{eq:min_acc_subtraj}
    s_i: x(t) = a_{i,3} (t-t_i)^3 + a_{i,2} (t-t_i)^2 + a_{i,1} (t-t_i) + a_{i,0},
\end{equation}
defined in the closed domain $[t_{i}, t_{i+1}]$, the coefficients $a_{i,j}$ are equal to:
\begin{IEEEeqnarray}{c}
\label{eq:min_acc_subtraj_coef} \phantomsection   \IEEEyesnumber \IEEEyessubnumber*
  a_{i,0} = x_i \\ a_{i,1} = \dot{x}_i \\
  a_{i,2} = -\frac{3x_i - 3 x_{i+1} + 2 \delta_i \dot{x}_i + \delta_i \dot{x}_{i+1} }{\delta_i^2} \\ a_{i,3} = \frac{2x_i - 2 x_{i+1} + \delta_i \dot{x}_i + \delta_i \dot{x}_{i+1} }{\delta_i^3},
\end{IEEEeqnarray}
where $\delta_i = t_{i+1} - t_i$. Given a knot $(t_i, x_i)$ such that $i \ne 0$ and $i \ne N$, we compute the velocity $\dot{x}_i$ by asking for continuous acceleration at the knots, i.e., 
\begin{equation}
    \label{eq:min_acc_constraint}
    \at{\frac{\diff ^2}{\diff t^2} s_{i-1}}{t=t_i}
 = \at{\frac{\diff ^2}{\diff t^2} s_i}{t=t_i}
\end{equation}
By substituting~\eqref{eq:min_acc_subtraj_coef} into~\eqref{eq:min_acc_subtraj} applying the constraint~\eqref{eq:min_acc_constraint} and imposing the continuity at $t_i$, we obtain the following linear equation
\begin{equation}
\label{eq:min_acc_constraint_explict}
\frac{3 x_{i-1} - 3 x_i + 2 \delta_{i-1} \dot{x}_{i} + \delta_{i-1} \dot{x}_{i-1}}{\delta_{i-1}^2} 
+ \frac{3 x_{i} - 3 x_{i+1} + 2 \delta_{i} \dot{x}_{i} + \delta_{i} \dot{x}_{i+1}}{\delta_{i}^2} = 0.
\end{equation}
We note that for $i=1$ and $i=N-1$ the terms $\dot{x}_{i-1}$ and $\dot{x}_{i+1}$ are, respectively, known. In fact, when $i=1$, $\dot{x}_{i-1} = \dot{x}_{0}$, whether $i=N-1$, $\dot{x}_{i+1} = \dot{x}_{N}$.
By stacking Equation \eqref{eq:min_acc_constraint_explict} for all the points $(t_i, x_i)$ such that $i \ne 0$ and $i \ne N$, we obtain a sparse linear system which can be easily solved online.
\par
Once the velocity of all the knots has been determined, the coefficients~\eqref{eq:min_acc_subtraj_coef} can be evaluated and, consequently, the minimum acceleration trajectory is completely defined. 
