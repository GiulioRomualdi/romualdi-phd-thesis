\section{Simplified model architecture \label{sec:simplified_model_architecture}}
In this section, we summarize the components constituting the simplified model blocks of the three-layer controller architecture -- Figure~\ref{fig:three-layer}. In detail, we present
\begin{itemize}
    \item how we consider the first and last steps in the DCM planner,
    \item the swing foot trajectory planner,
    \item the simplified model control layer.
\end{itemize}
These components share a lot of commonalities with other state-of-the-art
approaches. Nevertheless, this section presents how these three components
interconnect to define part of a walking architecture.


\subsection{The DCM trajectory planner\label{sec:dcm_traj_planner_simplfied}}
The DCM trajectory planner used in our architecture inherits from the formulation presented in Section~\ref{sec:dcm_traj_gen}. However, differently from~\cite{Englsberger2014}, we explicitly consider specific boundary conditions for the first and last steps.
\par
For the first double support phase, we ask for an initial DCM position ${\xi}^{\text{ios}_\text{DS}}_1$ that coincides with the projection of the measured CoM on the ground surface. In this particular context, we set the initial boundary conditions of the DCM position~\eqref{eq:3d_dcm_ds_boundary_ios} and velocity~\eqref{eq:3d_dcm_ds_boundary_ios_v} as:
\begin{IEEEeqnarray}{LL}
\phantomsection  \IEEEyesnumber  \IEEEyessubnumber*
  {\xi}^{\text{ios}_\text{DS}}_1 = x_\text{LIP} \\
  \dot{{\xi}}^{\text{ios}_\text{DS}}_1 = {0}_{2\times1}.
\end{IEEEeqnarray}
Figure~\ref{fig:dcm_first_step} presents the DCM trajectory generation for the first double support phase. 
\begin{figure}[t]
\centering
    \begin{subfigure}[b]{0.48\textwidth}
        \centering
        \includegraphics{chapter_simplified_benchmarking/figures/dcm_first_step.tikz}
        \caption{DCM trajectory at the first step}
        \label{fig:dcm_first_step}
    \end{subfigure}
    \hfill
    \begin{subfigure}[b]{0.48\textwidth}
        \centering
        \includegraphics{chapter_simplified_benchmarking/figures/dcm_last_step.tikz}
        \caption{DCM trajectory at the last step}
        \label{fig:dcm_last_step}
    \end{subfigure}
	\caption[DCM trajectory planner at the first and last step]{DCM trajectory planner at the first and last step. The SS trajectory is represented by the orange segments, while the DS trajectory is represented by the light blue curves (a) At the first step, the initial DCM position ${\xi}^{\text{ios}_\text{DS}}_1$ coincides with the projection of the CoM on the ground surface. (b) At the last step, the final DCM position ${\xi}^{\text{eos}_{\text{DS}}}_N$ coincides with the projection of the CoM on the ground surface.}
	\label{fig:dcm_first_last_step}
\end{figure} 
\par
Similarly, the desired position of the DCM at the end of the double support phase
${\xi}^{\text{eos}_{\text{DS}}}_N$ must coincide with the projection of the desired CoM on the ground surface, denoted by $x_\text{LIP}^*$. This is obtained as the middle point between the two feet.
In this particular case, we set the final boundary conditions for the DCM position~\eqref{eq:3d_dcm_ds_boundary_eos} and velocity~\eqref{eq:3d_dcm_ds_boundary_eos_v} as:
\begin{IEEEeqnarray}{LL}
\phantomsection  \IEEEyesnumber  \IEEEyessubnumber*
  {\xi}^{\text{eos}_\text{DS}}_N = x_\text{LIP}^* \\
  \dot{{\xi}}^{\text{eos}_\text{DS}}_N = {0}_{2\times1}.
\end{IEEEeqnarray}
In \cite{Englsberger2014}, the DCM trajectory for the last single support phase is calculated by assuming that the final position of the DCM coincides with the local ZMP in the last step -- Equation~\eqref{eq:dcm_eos_evaluation_final_condition}. In our test, we noticed that this choice may lead to an undesired overshoot of the DCM projection into the \emph{coronal plane}~\footnote{A \emph{coronal plane}, also known as the \emph{frontal plane}, is any vertical plane that divides the body into ventral and dorsal sections.}, and consequently to an unfeasible DCM trajectory. 
To mitigate this behavior, we ask that the final condition of the single support DCM trajectory ${\xi}_{N-1}^{\text{eos}}$ belongs to the \emph{affine combination} (see Section~\ref{sec:affine_convex_sets}) of the two last footprints, i.e., 
\begin{equation}
    {\xi}_{N-1}^{\text{eos}} = \alpha_\text{LS} x_{\text{ZMP}_N}  + (1 - \alpha_\text{LS}) x_{\text{ZMP}_{N-1}}.
\end{equation}
When $\alpha_\text{LS} = 0$ the DCM position coincides with the ZMP position of the stance foot. On the other hand, if $\alpha_\text{LS} = 1$ we obtain the same condition proposed by \cite{Englsberger2014}. In Figure~\ref{fig:dcm_final_step_alpha}, we present the DCM trajectory in the last step projected on \emph{frontal plane} for different values of $\alpha_\text{LS}$.  $\alpha_\text{LS} = 1$ leads to an undesired overshoot and, as a consequence, to an unfeasible DCM trajectory. In our experiments, we always set $\alpha_\text{LS} = 0.2$, which in our opinion is a good compromise between a responsive trajectory while avoiding undesired overshoots. 
\begin{figure}[t]
  \centering
    \includegraphics{chapter_simplified_benchmarking/figures/dcm_final_step_alpha.tikz}
    \caption[Final step DCM trajectory w.r.t. $\alpha_\text{LS}$]{Final step DCM trajectory w.r.t. $\alpha_\text{LS}$. The greater $\alpha_\text{LS}$, the higher the overshoot. \label{fig:dcm_final_step_alpha}}
\end{figure}
Figure~\ref{fig:dcm_last_step} presents the generation of the DCM trajectory for the last double support phase. 

\subsection{Swing Foot Trajectory \label{sec:swing_foot_trajectory}}
We now discuss the problem of generating the foot trajectory from the footsteps. More formally, given two consecutive footsteps poses, hereafter denoted with ${}^\mathcal{I} H _{F_0} \in \SE(3)$ and ${}^\mathcal{I} H _{F_1} \in \SE(3)$, we aim to find a trajectory that connects these two poses. A common approach to this problem is to seek a foot trajectory ${}^\mathcal{I} H _{F}(t)$ such that its acceleration or jerk\footnote{We denote with \emph{jerk} the rate at which an object's acceleration changes with respect to time} is minimized. Since all the whole-body control layer implementations presented in Part~\ref{part:wbc} assume that frame velocity is expressed in \emph{mixed representation} -- Section~\ref{sec:mixed_spatial_velocity}, we split the problem of finding an optimal Cartesian trajectory into two subproblems, namely: evaluating a positional trajectory $p_F(t) \in \mathbb{R}^3$ and a rotation trajectory ${}^\mathcal{I} R _ F (t) \in \SO(3)$.

\subsubsection{Swing foot position}
During the walking gait, the swing foot has to move from the initial position $p_F(t_0) = p_{F_0}$ to $p_F(t_N) = p_{F_N}$ to reach a desired maximum
height. The instant in which a foot reaches its maximum altitude is called the apex time, $t_\text{apex}$.
\par
We want to find a trajectory $p_F(t) \in \mathbb{R}^3$ such that:
\begin{itemize}
    \item $p_F(t_0) = p_{F_0}$, $p_F(t_N) = p_{F_N}$ and $p_F(t_\text{apex}) = p_{F_\text{apex}}$
    \item the integral of a suitable Lagrangian function $\mathcal{L}(t)$ is minimized, i.e., 
    \begin{equation}
        \label{eq:foot_position_minimization_general}
        \minimize \int_{T_{0}}^{T_{N}} \mathcal{L}(t) \diff t,
    \end{equation}
    where $\mathcal{L}(t) \ge 0$.
\end{itemize}
Depending on the choice of the objective function, we can design a minimum acceleration or a minimum jerk trajectory.
\paragraph{Minimum acceleration trajectory}
If we set the Lagrangian function $\mathcal{L}(t)$ as the squared-norm of the linear acceleration:
\begin{equation}
    \mathcal{L} = \ddot{p}_F^\top \ddot{p}_F,
\end{equation}
it is possible to show that a trajectory composed by the concatenation of the $3^\text{rd}$ order polynomial functions
\begin{equation}
    \label{eq:spline_3d_order}
    s_i: p_F (t) =  a _{i, 3} t ^ 3 + a _ {i,2} t ^ 2 + a _ {i,1} t + a _ {i,0},
\end{equation}
is a minimum acceleration trajectory. Here the coefficients $a_{i,j}$ are chosen to satisfy the position and velocities boundary conditions for each subtrajectory,
The complete foot position minimum acceleration trajectory is finally computed by attaching the 3rd-order polynomial functions~\eqref{eq:spline_3d_order} as 
\begin{equation}
    p_F (t) = \begin{cases}
      a _{0, 3} t ^ 3 + a _ {0,2} t ^ 2 + a _ {0,1} t + a _ {0,0} \quad  \text{if } t_0 \le t \le t_\text{apex} \\
      a _{1, 3} t ^ 3 + a _ {1,2} t ^ 2 + a _ {1,1} t + a _ {1,0} \quad  \text{if }  t_\text{apex} < t \le t_N
    \end{cases}
\end{equation}
\par
Appendix~\ref{appendix:min_acc} presents all the passages required to evaluate a minimum acceleration trajectory in $\mathbb{R}^n$.
\paragraph{Minimum jerk trajectory}
If we set the Lagrangian function $\mathcal{L}(t)$ as the squared-norm of the linear jerk:
\begin{equation}
    \mathcal{L} = \dddot{p}_F^\top \dddot{p}_F
\end{equation}
we notice that a trajectory composed by the concatenation of a $5^\text{th}$ order polynomial functions:
\begin{equation}
\label{eq:spline_5d_order}
    s_i :  p_F (t) =  a _ {i,5} t ^ 5 + a _ {i,4} t ^ 4 + a _ {i,3} t ^ 3 + a _ {i,2} t ^ 2 + a _ {i,1} t + a _ {i,0},
\end{equation}
minimizes the jerk. The coefficients $a_{i,j}$ are chosen to satisfy the boundary conditions of position, velocity and acceleration.
Finally, the complete foot position minimum acceleration trajectory is computed by attaching $5^\text{th}$-order polynomial functions as:
\begin{equation}
    p_F (t) = \begin{cases} 
      a _ {0,5} t ^ 5 + a _ {0,4} t ^ 4  + a _{0, 3} t ^ 3 + a _ {0,2} t ^ 2 + a _ {0,1} t + a _ {0,0} \quad  \text{if } t_0 \le t \le t_\text{apex} \\
      a _ {1,5} t ^ 5 + a _ {1,4} t ^ 4  + a _{1, 3} t ^ 3 + a _ {1,2} t ^ 2 + a _ {1,1} t + a _ {1,0} \quad  \text{if }  t_\text{apex} < t \le t_N
    \end{cases}
\end{equation}
Appendix~\ref{appendix:min_jerk} presents all the passages required to evaluate a minimum jerk trajectory in $\mathbb{R}^n$.
\subsubsection{Swing foot orientation}
During the walking gait, the swing foot rotates from the initial orientation ${}^\mathcal{I}R_F(t_0) = {}^\mathcal{I}R_{F_0}$ to ${}^\mathcal{I}R_F(t_N) = {}^\mathcal{I}R_{F_N}$. Similarly to what we discussed for the generation of position trajectory, we now aim to compute a trajectory ${}^\mathcal{I}R_F(t)$ such that 
\begin{itemize}
    \item  ${}^\mathcal{I}R_F(t_0) = {}^\mathcal{I}R_{F_0}$ and ${}^\mathcal{I}R_F(t_N) = {}^\mathcal{I}R_{F_N}$
    \item a suitable Lagrangian function $\mathcal{L}(t)$ is minimized.
\end{itemize}
We can now construct a minimum acceleration or a minimum jerk trajectory depending on the objective function, $\mathcal{L}(t)$.
\paragraph{Minimum acceleration trajectory}
If we set the Lagrangian function $\mathcal{L}(t)$ as the squared-norm of the right trivialized (inertial frame) angular acceleration ${}^\mathcal{I} \dot{\omega}_{\mathcal{I}, F}$:
\begin{equation}
    \label{eq:lagrangian_min_acc}
    \mathcal{L} = {}^\mathcal{I} \dot{\omega}_{\mathcal{I}, F}^\top {}^\mathcal{I} \dot{\omega}_{\mathcal{I}, F}
\end{equation}
and we ask for a zero initial and final angular velocity, 
it is possible to show that the  
\begin{equation}
    \label{eq:min_acc_rot}
    {}^\mathcal{I}R_F(t) = \exp{\left(s(t-t_0) \log\left( {}^\mathcal{I}R_F(t_N) {}^\mathcal{I}R_{F_0}^\top  \right)\right)} {}^\mathcal{I}R_{F_0},
\end{equation}
where $s(\tau)$ is given by
\begin{equation}
\label{eq:min_acc_rot_s}
    s(\tau) = \frac{3}{(t_N - t_0)^2} \tau^2 - \frac{3}{(t_N - t_0)^3} \tau^3,
\end{equation}
minimizes the integral of the Lagrangian function~\eqref{eq:lagrangian_min_acc}~\citep[Section~9.2]{Lynch2017ModernControl}\citep{Zefran1999MetricsKinematics,Zefran1998OnMotions,Park1997SmoothRotations}
\par
In Appendix~\ref{appendix:optimal_planning_so3}, we present the theory behind the minimum acceleration trajectory in $\SO(3)$. 
\paragraph{Minimum jerk trajectory}
If we set the Lagrangian function $\mathcal{L}(t)$ as the squared norm of the right trivialized (inertial frame) angular jerk ${}^\mathcal{I} \ddot{\omega}_{\mathcal{I}, F}$:
\begin{equation}
    \label{eq:lagrangian_min_jerk}
    \mathcal{L} = {}^\mathcal{I} \ddot{\omega}_{\mathcal{I}, F}^\top {}^\mathcal{I} \ddot{\omega}_{\mathcal{I}, F}.
\end{equation}
and we ask for a zero initial and final angular velocity and acceleration, 
it is possible to show that  
\begin{equation}
    \label{eq:min_jerk_rot}
    {}^\mathcal{I}R_F(t) = \exp{\left(s(t-t_0) \log\left( {}^\mathcal{I}R_F(t_N) {}^\mathcal{I}R_{F_0}^\top  \right)\right)} {}^\mathcal{I}R_{F_0}
\end{equation}
where $s(\tau)$ is given by
\begin{equation}
\label{eq:min_jerk_rot_s}
    s(\tau) = \frac{6}{(t_N - t_0)^5} \tau^5 - \frac{15}{(t_N - t_0)^4} \tau^4 + \frac{10}{(t_N - t_0)^3} \tau^3,
\end{equation}
minimizes the integral of the Lagrangian function~\eqref{eq:lagrangian_min_jerk}~\citep{Liu2017Minimum-jerkActuator,Zefran1999MetricsKinematics}
\par
Differently from~\eqref{eq:min_acc_rot} and~\eqref{eq:min_acc_rot_s}, to prove~\eqref{eq:min_jerk_rot} and~\eqref{eq:min_jerk_rot_s} we need to exploit the mathematical tools of \emph{Differential Geometry}. Since this topic is outside the scope of the thesis, we avoid presenting the rigorous proof. The interested reader can refer to the extensive literature, among which it is worth mentioning~\citep{Zefran1999MetricsKinematics,Zefran1998OnMotions,Pressley2010ElementaryGeometry,Needham2021VisualForms,Selig2007CurvesSE3,Dubrovin1984ModernApplications}





\subsection{Simplified model control layer}

As introduced in Section~\ref{sec:dcm}, the simplified model~\eqref{eq:dcm_dynamics_lipm_com} shows that the CoM asymptotically converges to a constant DCM, while the DCM has an unstable first-order dynamics~\eqref{eq:dcm_dynamics_lipm_dcm}. In this section, we present control laws that aim at stabilizing the DCM dynamics by assuming the location of the ZMP $x_\text{ZMP}$ as the control input.
\par
This stabilization problem has been tackled by designing and comparing \emph{instantaneous} and \emph{predictive} (MPC) control laws.
Unlike the MPC, the instantaneous control law does not solve any optimization problem, and it uses only the current desired and actual position of the DCM to evaluate its output.
\par

\subsubsection{DCM instantaneous control}
\label{instantaneous-control}

Differently from \cite{Englsberger2015}, we propose an instantaneous control law that does not perform a cancellation of the unstable system dynamics, which, in practice, is never perfect and can lead to system instabilities. More precisely, we chose
\begin{equation}
\label{eq:reactive_dcm}
x_\text{ZMP}^{*} = \xi^{\text{ref}}- \frac{1}{\zeta} \dot{\xi}^{\text{ref}}+ K^{\xi}_{p} \left(\xi - \xi^\text{ref} \right) + K^{\xi}_{i} \int{\xi - \xi^\text{ref} \diff t}.
\end{equation}
where $K^{\xi}_{p}> I_2$ and $K^{\xi}_{i} > 0_{2\times2}$
\par
Applying the control input \eqref{eq:reactive_dcm} to the system \eqref{eq:dcm_dynamics_lipm_dcm_without_lip} leads to the following closed-loop dynamics:
\begin{equation}
\label{eq:reactive_cl}
\dot{\xi} - \dot{\xi}^{\text{ref}} = \zeta(I_2 - K^{\xi}_{p})  (\xi - \xi^{\text{ref}})
-\zeta K^{\xi}_{i}  \int{\xi - \xi^{\text{ref}} \diff t}.
\end{equation}
It is easy to show that the DCM error and its integral converge asymptotically to zero.
The above law~\eqref{eq:reactive_dcm} is very simple to implement and guarantees the tracking of the
desired DCM. 
The main limitation is that it may not ensure the feasibility of the gait, since the position of the ZMP may exit the support polygon and, consequently, the ZMP contact feasibility criterion~\citep{Vukobratovic1969} may not be guaranteed.
Furthermore, the above instantaneous control law does not take into account the desired DCM trajectory planned for the future.

\subsubsection{DCM predictive control}
\label{predictive-control}
In order to overcome these issues, we design a model predictive controller by following the work of ~\cite{Krause2012}. 
The control objective is achieved by framing the controller as a constrained optimization problem composed of three
main elements, namely: \emph{the prediction model}, an \emph{objective
function} and a set of \emph{inequality constraints}.

\paragraph{Prediction model} In the MPC framework, we consider the dynamics of DCM \eqref{eq:dcm_dynamics_lipm_dcm_without_lip} as a prediction model. Equation~\eqref{eq:dcm_dynamics_lipm_dcm_without_lip} is discretized supposing piecewise constant ZMP trajectories. By applying the \emph{Zero order hold} technique presented in Section~\ref{sec:zoh}, the discrete DCM dynamics writes as follows:
\begin{equation}
\label{eq:dcm_discrete}
\begin{split}
\xi_{k+1} &= F \xi_k + G x_{\text{ZMP}_{k}} \\
&=
\begin{bmatrix}
e^{\zeta T} & 0 \\
0 & e^{\zeta T}
\end{bmatrix} \xi_k {+}
\begin{bmatrix}
1 - e^{\zeta T} & 0\\
0 & 1 - e^{\zeta T}
\end{bmatrix}
x_{\text{ZMP}_{k}}.
\end{split}
\end{equation}
where $T$ is the fixed sampling time. 
\paragraph{Objective function}
The objective function is defined in terms of two main tasks, namely a tracking task and a control input regularization task. 
\par
To follow a desired DCM trajectory, we minimize the weighed norm of the error between the robot DCM and the desired trajectory as:
\begin{equation}
    \label{eq:mpc_simplified_cost_dcm}
    \Psi_\xi = \frac{1}{2} \norm{\xi-{\xi}^\text{ref}}_{\Lambda_\xi}^2,
\end{equation}
where $\Lambda_\xi$ is a positive definite diagonal matrix.
On the other hand, to reduce the rate of change of the ZMP, we minimize the ZMP derivative by considering the following task:
\begin{equation}
    \label{eq:mpc_simplified_cost_zmp}
    \Psi_\text{ZMP} = \frac{1}{2} \norm{\dot{x}_\text{ZMP}}_{\Lambda_\text{ZMP}}^2.
\end{equation}
With $\Lambda_\text{ZMP}$ a diagonal positive matrix.

\paragraph{Inequality constraint}
In order to ensure that the stance foot does not rotate around one of its edges, the desired ZMP must satisfy the contact stability criterion~\citep{Vukobratovic1969,vukobratovic2004zero}. This is verified by seeking for a ZMP inside foot print, if the robot is in single support, or inside the convex hull of the two feet, in the double support scenario. The $\mathcal{H}$-rep (Section~\ref{sec:convex_set_example}) is given by the following set of inequalities:
\begin{equation}
\label{eq:inequality_constraints_simplified}
A_{c} x_{\text{ZMP}} \preceq b_{c},
\end{equation}
where $A_{c}$ and $b_{c}$ are time-variant and their dimension depends on the type of support.

\paragraph{MPC formulation}
Combining the cost functions~\eqref{eq:mpc_simplified_cost_dcm} and \eqref{eq:mpc_simplified_cost_zmp}, with the prediction model \eqref{eq:dcm_discrete} and the inequality constraints~\eqref{eq:inequality_constraints_simplified}, we formulate the MPC as an optimization problem.
The MPC problem is transcribed using the Direct Single Shooting method with a constant sampling time $T$ -- Section~\ref{sec:shooting_methods}. The desired ZMP, $x^*_\text{ZMP}$ is generated by applying the receding horizon principle~\citep{Mayne90MPC} -- Section~\ref{sec:mpc}.
The MPC formulation is finally obtained by solving the following optimization problem:
\begin{IEEEeqnarray}{CL}
\phantomsection \label{eq:mpc_solution_simplified} \IEEEyesnumber  \IEEEyessubnumber*
\minimize\limits_{x_{\text{ZMP}_k}, \;k = [0\; N]} & \sum_{k = 0}^N \left(\Psi_\xi + \Psi_\text{ZMP}\right) \label{eq:mpc_cost_function} \\ 
\text{subject to}   &{\xi}_{k+1} = F {\xi}_k + G {x}_{\text{ZMP}_k} \label{eq:prediction_model} \\
 &  A_{c_k} {x}_{\text{ZMP}_k} \preceq {b}_{c_k} \label{eq:mpc_zmp_constraint}\\
 & {\xi}_0 = {\bar{\xi}} \\
 & {x}_{\text{ZMP}_{-1}} = {\bar{x}}_{\text{ZMP}}. 
\end{IEEEeqnarray}
where ${\bar{x}}_{\text{ZMP}}$ is the desired ZMP calculated in the previous control iteration and $\bar{{\xi}}$ is the estimated DCM position.
\par
Since the cost function is a quadratic positive function and the constraints are linear, the optimal control problem is quadratic and can be transcribed into a strictly convex quadratic programming
problem (QP) (Section~\ref{sec:qp}) of the form:
\begin{IEEEeqnarray*}{CLLL}
\;\minimize\limits_{w} \;& \IEEEeqnarraymulticol{3}{C}{\frac{1}{2} w^\top H_k w + g_k^\top w}\\
\; \text{subject to} \; & A_{c_k}^i w &\preceq &b_{c_k}^i
\end{IEEEeqnarray*}
Here, the optimization variables are stacked inside the vector 
\begin{equation}
w^\top = \begin{bmatrix}
x_{\text{ZMP}_{0}}^{\top} & \hdots & x_{\text{ZMP}_{N-1}}^{\top}
\end{bmatrix}.
\end{equation}
The Hessian matrix $H_k$ and the gradient vector $g_k$ derive from \eqref{eq:mpc_cost_function}.
The inequality constraint matrix $A_{c_k}^i$ and vector $b_{c_k}^i$ embed the ZMP constraint
\eqref{eq:mpc_zmp_constraint} and the prediction model \eqref{eq:prediction_model}.

\subsubsection{ZMP-CoM Controller}
\label{ZMP-CoM-Controller}
Independently of the chosen DCM controller, namely the controller in Section~\ref{instantaneous-control} or in~\ref{predictive-control}, one obtains the desired position and velocity of ZMP and CoM to stabilize. As a consequence, another control loop is needed after the DCM controller. Here, we choose the proposed control law \citep{Choi2007}, i.e.:
\begin{equation}
\label{eq:ZMP_controller}
\dot{x}^*_{\text{LIP}} = \dot{x}^\text{ref}_{\text{LIP}} - K_\text{ZMP}(x_\text{ZMP}^\text{ref} - x_\text{ZMP}) + K_\text{LIP} (x^\text{ref}_{\text{LIP}} - x_{\text{LIP}}),
\end{equation}
where $K_\text{LIP} > \zeta I_2$  and $0_2 < K_\text{ZMP} < \zeta I_2$.