\section{Background \label{sec:background-bachmarking}}

In several applications, we can assume flat terrain. In this scenario, a human, while walking, tends to keep the orientation of the body tangential to the path~\citep{Truong2010,Mombaur2010,flavigne2010reactive} to maximize the energy efficiency~\citep{handford2014sideways}. This consideration suggests designing a simple unicycle model to generate the desired footsteps locations and timings~\citep{Faragasso2013,cognetti2016real,8594277}. Authors of \citep{Faragasso2013} considers the unicycle model to plan footsteps in a corridor with turns and junctions using cameras. While \cite{cognetti2016real} exploits the unicycle model to perform \emph{evasive} robot maneuvers. However, in both cases, the footstep planners consider a constant velocity and a constant footsteps length. Attempts have been made to consider varying the length and velocity of the steps~\citep{8594277}.

\subsection{The unicycle model \label{sec:unicycle}}
\begin{figure}[!t]
  \centering
  \includegraphics{chapter_simplified_benchmarking/figures/unicycle.tikz}
  \caption{Unicycle model.\label{fig:unicycle}}
\end{figure}
Let us now consider a 2D unicycle -- Figure~\ref{fig:unicycle}, and assume that:
\begin{itemize}
    \item there exists an inertial frame $\mathcal{I}$;
    \item there exists a frame $U$ rigidly attached to the center of the unicycle, and we denote by $o_B$ the origin of the frame and by $[B]$ its orientation.
\end{itemize}
The unicycle model is described by the following dynamical system~\citep{Pucci2013}:
\begin{equation}
  \begin{cases}
    \label{eq:unicycle_dynamics}
    {\dot{o}}_{B} = R(\theta) e_1 v\\
    \dot{\theta} = \zeta
  \end{cases}
\end{equation}
where $v\in \mathbb{R}$ is the linear velocity of the robot expressed in the frame $B$. $\zeta\in \mathbb{R}$ is the unicycle angular velocity. ${o}_{B}$ is the position of the unicycle relative to the inertial frame $\mathcal{I}$. $\theta$ is the orientation of the robot. 
\par
The control objective is to asymptotically stabilize the point $F$, a fixed point with respect to the unicycle, to the desired point $F^*$. Thus, we define the error $\tilde{x}_F$ as
\begin{equation}
  \label{eq:unicycle_error}
  \tilde{x}_F:= x_F - x_F^*,
\end{equation}
where $x_F$ is given by
\begin{equation}
  \label{eq:point_unicycle_forward}
   x_F =
o_B +
R(\theta) \prescript{B}{}{x}_F,
\end{equation}
with $\prescript{U}{}{x}_F$ being a constant vector.  
The control objective becomes the asymptotic stabilization of $\tilde{x}_F$ to zero.
By time-differentiating Equation~\eqref{eq:point_unicycle_forward}, we obtain the following dynamical system
\begin{equation}
\label{eq:unicycle_output_dynamics}
    \dot{x}_F = \dot{o}_B + \zeta R(\theta) H \prescript{B}{}{x}_F.
\end{equation}
where $H$ is
\[
H= \begin{bmatrix}
  0 & -1 \\
  1 & 0
\end{bmatrix}.
\]
By substituting Equation~\eqref{eq:unicycle_dynamics} into~\eqref{eq:unicycle_output_dynamics}, we can rewrite the output dynamics as
\begin{equation}
    \dot{x}_F = B(\theta) u = \begin{bmatrix} R(\theta) e_1 & R(\theta) H  \prescript{U}{}{x}_F \end{bmatrix} u = R(\theta)
\begin{bmatrix}
  1 & -\prescript{U}{}{x}_{F_y}\\
  0 & \prescript{U}{}{x}_{F_x}
\end{bmatrix} u.
\end{equation}
Here, $u$ is the vector containing the controlled input $v$ and $\zeta$, i.e.,  $u = \begin{bmatrix} v & \zeta \end{bmatrix}^\top$. It can be seen that $\det\left(B(\theta)\right) = \prescript{B}{}{x}_{F_x}$, consequently, when the control point $F$ is not located on the axis of the wheels, its stabilization to an arbitrary reference trajectory $F^*$ can be achieved by using a simple feedback linearization law of the form
\begin{equation}
  \label{eq:unicycle_control_law}
 u  = B(\theta)^{-1} ({\dot{{x}}}_{F}^* - K ( x_F - x_F^* )),
\end{equation}
with $K$ a positive definite matrix, that is, $K \succ 0$.

\subsection{Footsteps trajectory planner
\label{sec:footsteps_trajectory}}
\begin{figure}[!t]
  \centering
  \includegraphics{chapter_simplified_benchmarking/figures/footsteps_unicycle.tikz}
  \caption{Footsteps planning from the unicycle trajectory.\label{fig:footsteps_unicycle}}
\end{figure}
To obtain the footsteps positions, the trajectory of the left foot and the right foot,
represented by the wheels of the unicycle, are sampled using a sampling time $\diff T$ as in~\citep{8594277}
\begin{equation}
    {{o}}_{\mathcal{L}_k} = {{o}}_{U_k} + R(\theta_k) \prescript{U}{}{{o}}_\mathcal{L},
\quad \quad
{{o}}_{\mathcal{R}_k} = {{o}}_{U_k} + R(\theta_k) \prescript{U}{}{{o}}_\mathcal{R},
\end{equation}

where $\mathcal{L}$ and $\mathcal{R}$ are, respectively, the frames attached to the left and right foot. The subscript $k$ denotes the time instant, i.e., $t = k \diff T$.
\par
Given the set of unicycle position (which is finite, thanks to discretization), we can select a particular position for the feet.
Each position in the set is associated with a time instant, $k$. This instant is considered the moment in which the corresponding foot is expected to touch the ground and is called the impact time, $t_{\text{imp}}$. Impact time can be used as decision variables to also select the position of feet.
\par
Using the impact time, the step duration can be defined:
\begin{equation}
    \Delta_t = |t^{lf}_{\text{imp}} - t^{rf}_{\text{imp}}|.
\end{equation}
Another interesting quantity is the distance of the feet during the double support phase
\begin{equation}
\Delta_x = \norm{\prescript{\mathcal{I}}{}{{o}}^{ds}_{\mathcal{L}_k} - \prescript{\mathcal{I}}{}{{o}}^{ds}_{\mathcal{R}_k}}.
\end{equation}
Both $\Delta_x$ and $\Delta_t$ are used within the cost function in order to reduce the possibility
to obtain long and fast steps
\begin{equation}
J = K_t \frac{1}{\Delta_t^2}  +  K_x \Delta_x^2,
\end{equation}
where $K_t>0$ and $K_x>0$ are used to allow the robot to perform long/short and fast/slow steps.
\par
However, considering only the cost function, too long/short or too fast/slow steps can be planned. To avoid this undesirable behavior, several constraints are added.
First of all, the bound on the step duration is necessary to avoid too fast or too slow steps and a division by zero in the cost function:
\begin{equation}
t_{\text{min}} \leq \Delta_t \leq t_{\text{max}}.
\end{equation}
Also, the distance between the feet during the double support phase has to be bound in order to
avoid undesired collision between feet and, on the other hand, step length that the robot, according to its mechanical structure cannot achieve:
\begin{equation}
d_{\text{min}} \le \Delta_x \leq d_{\text{max}}.
\end{equation}
Last but not least, an additional constraint has to be taken into account: the relative rotation
between the two feet, $|\Delta_\theta|$, has to be lower than $\theta_{\text{max}}$,
this is necessary to take into account the mechanical limitations of the humanoid robot.
\begin{equation}
|\Delta_\theta| \leq  \theta_{\text{max}}.
\end{equation}
\par

Finally, the control objective is achieved by defining the planning problem as a constrained optimization problem whose conditional variable is $t_{\text{imp}}$. In detail, we have:
\begin{IEEEeqnarray}{CCL}
\phantomsection \label{eq:path_optimization_problem}  \IEEEyesnumber  \IEEEyessubnumber*
\minimize\limits_{t_{\text{imp}}}  && K_t \frac{1}{\Delta_t^2}  +  K_x \Delta_x ^2 \\
\st& \quad &t_{\text{min}} \leq \Delta_t \leq t_{\text{max}}\\
    && d_{\text{min}} \le \Delta_x \leq d_{\text{max}} \\
   && |\Delta_\theta| \leq  \theta_{\text{max}}.
\end{IEEEeqnarray}


\subsection{DCM trajectory generator \label{sec:dcm_traj_gen}}
We now introduce the trajectory generation problem. Following the work of \cite{Englsberger2014}, we first assume instantaneous transitions between two consecutive single support phases (SS).
Later, we relax this assumption by considering the double support phases (DS).

\subsubsection{Generation of the single support trajectory \label{sec:dcm_ss}}
\begin{figure}[t]
  \centering
  \includegraphics{chapter_simplified_benchmarking/figures/dcm_and_zmp.tikz}
  \caption[DCM trajectory planning assuming single support phases only]{Planning of the DCM trajectory on a flat terrain considering only single support phases.
    The ZMP position is assumed constant during the
    single support transition. The final position of the DCM coincides
    with ZMP of the last footstep, ${\xi}_{4}^{\text{eos}} = {x}_{\text{ZMP}_5}$.\label{fig:dcm_and_zmp}}
\end{figure}
Let us consider a set of footsteps locations and timings generated by a proper footsteps planner, e.g., the unicycle footstep planner in Section~\ref{sec:footsteps_trajectory}. Let us also assume that the LIPM hypothesis is valid -- Section~\ref{sec:lip}. Then the DCM dynamics~\eqref{eq:3d_dcm_dynamics_eCMP} can be projected on the contact plane and rewritten as \eqref{eq:dcm_dynamics_lipm_dcm}:
\begin{equation}
    \label{eq:dcm_dynamics_lipm_dcm_without_lip}
    \dot{\xi} = \zeta\left(\xi - x_\text{ZMP}\right)
\end{equation}
where, for the sake of clarity, we removed the suffix LIP.
\par
For a constant ZMP, the projected DCM dynamics \eqref{eq:dcm_dynamics_lipm_dcm} admits a close form solution of the form:
\begin{equation}
  \label{eq:dcm_solution_ios}
  {\xi}(t) = {x}_\text{ZMP} + e ^{\zeta t} ({\xi}_0 - {x}_\text{ZMP}),
\end{equation}
where ${\xi}_0 = {\xi}(0)$ and the time $t$ has to belong to the step domain $t \in [0, \; t^\text{\text{step}}_i]$, where $t^\text{\text{step}}_i$ is the duration of the $i$-th step.
\par
Assuming that the final position of the DCM coincides with the ZMP at the last step, i.e., ${\xi}_{N-1}^{\text{eos}} = {x}_{\text{ZMP}_N}$, Equation~\eqref{eq:dcm_solution_ios} can be used to find the desired DCM position at the end of each step as:
\begin{IEEEeqnarray}{LL}
\phantomsection \label{eq:dcm_eos_evaluation}  \IEEEyesnumber  \IEEEyessubnumber*
    {\xi}_{N-1}^{\text{eos}} &= {x}_{\text{ZMP}_N} \label{eq:dcm_eos_evaluation_final_condition} \\
    {\xi}_{i-1}^{\text{eos}} &= {\xi}_{i}^{\text{ios}} = {x}_{\text{ZMP}_i} + e^{-\zeta t^\text{\text{step}}_i} ({\xi}_i^{\text{eos}} - {x}_{\text{ZMP}_i}),
\end{IEEEeqnarray}
where ${\xi}_{i}^{\text{ios}}$ and ${\xi}_{i}^{\text{eos}}$ are respectively the initial
and final positions of the desired DCM for the $i$-th step .
\par
By substitution of \eqref{eq:dcm_eos_evaluation} into \eqref{eq:dcm_solution_ios}, the reference DCM trajectory is calculated as follows:
\begin{equation}
  \label{eq:dcm_solution_eos}
  {\xi}_i(t) = {x}_{\text{ZMP}_i} + e ^{\zeta \left(t - t^\text{\text{step}}_i\right)} \left({\xi}^{\text{eos}}_i - {x}_{\text{ZMP}_i}\right).
\end{equation}
The DCM velocity trajectory can be easily evaluated by differentiation of \eqref{eq:dcm_solution_eos}
\begin{equation}
\dot{{\xi}}_i(t) =\zeta e ^{\zeta \left(t - t^\text{\text{step}}_i\right)} \left({\xi}^{\text{eos}}_i - {x}_{\text{ZMP}_i}\right).
\end{equation}
By applying, Equation~\eqref{eq:dcm_eos_evaluation} and  \eqref{eq:dcm_solution_eos} recursively, we compute the DCM trajectory. Figure~\ref{fig:dcm_and_zmp} presents an example of DCM trajectory generation in the case of five footsteps.

\subsubsection{Generation of the double support trajectory \label{sec:dcm_ds}}
The presented planning method is very powerful and allows one to generate the desired DCM
trajectory in real-time. Nevertheless, it has the limit of taking into account only single support phases. Indeed, considering only instantaneous transitions between two consecutive single support phases, the ZMP
reference is discontinuous~\citep{Englsberger2014}.
This makes the external forces  discontinuous too, and, in turn, also the desired joint torque may
suffer from discontinuity.
This can be unfeasible for a physical robot due to its limited actuator dynamics. These considerations motivate the development of a new DCM trajectory generator that handles the non-instantaneous transition between two single support phases \citep{Englsberger2014}.
\par
We notice that by rearranging Equation~\eqref{eq:dcm_dynamics_lipm_dcm_without_lip}, the ZMP position depends linearly on the DCM position and velocity
\begin{equation}
  \label{eq:dcm_zmp_relation}
  {x}_\text{ZMP} = {\xi} - \zeta \dot{{\xi}},
\end{equation}
as a consequence, we can conclude that a DCM reference trajectory with continuous derivative, i.e., $C^1$ continuity, guarantees a continuous ZMP trajectory. This consideration motivates the use of a third-order polynomial interpolation to smooth the edges of the DCM reference trajectory evaluated using the exponential technique 
\begin{equation}
\label{eq:3d_dcm_ds_base} 
    {\xi}^\text{DS}(t) = {a}_3 t^3 + {a}_2 t^2 + {a}_1 t + {a}_0,
\end{equation}
where the parameters ${a}_i$ must be chosen to satisfy the velocity and position boundary conditions, namely:
\begin{IEEEeqnarray}{LL}
\phantomsection \label{eq:3d_dcm_ds_boundary}  \IEEEyesnumber  \IEEEyessubnumber*
  {\xi}^{\text{ios}_\text{DS}}_i &= {x}_{\text{ZMP}_{i-1}} + e ^{\zeta \left(t^{\text{ios}_\text{DS}}_{i-1} - t^{\text{step}}_{i-1}\right)} ({\xi}^{\text{eos}}_{i-1} - {x}_{\text{ZMP}_{i-1}}) \label{eq:3d_dcm_ds_boundary_ios}  \IEEEyessubnumber \\
  \dot{{\xi}}^{\text{ios}_\text{DS}}_i &= \zeta e ^{\zeta \left(t^{\text{ios}_\text{DS}}_{i-1} - t^{\text{step}}_{i-1}\right)} ({\xi}^{\text{eos}}_{i-1} - {x}_{\text{ZMP}_{i-1}}) \label{eq:3d_dcm_ds_boundary_ios_v}  \IEEEyessubnumber \\
  {\xi}^{\text{eos}_{\text{DS}}}_i &= {x}_{\text{ZMP}_{i}} + e ^{\zeta \left(t^{\text{eos}_{\text{DS}}}_i - t^{\text{step}}_i\right)} \left({\xi}^{\text{eos}}_i - {x}_{\text{ZMP}_{i}}\right) \label{eq:3d_dcm_ds_boundary_eos}  \IEEEyessubnumber \\
  \dot{{\xi}}^{\text{eos}_{\text{DS}}}_i &= \zeta  e ^{\zeta \left(t^{\text{eos}_{\text{DS}}}_i - t^{\text{step}}_i\right)} ({\xi}^{\text{eos}}_i - {x}_{\text{ZMP}_{i}}). \label{eq:3d_dcm_ds_boundary_eos_v}  \IEEEyessubnumber
\end{IEEEeqnarray}
Equations~\eqref{eq:3d_dcm_ds_boundary_ios} and \eqref{eq:3d_dcm_ds_boundary_ios_v} denote, respectively, the desired position and velocity of DCM at the beginning of the double support phase. On the other hand, Equations~\eqref{eq:3d_dcm_ds_boundary_eos} and \eqref{eq:3d_dcm_ds_boundary_eos_v} represent
the desired position and velocity of DCM at the end of the double support phase. $t^{\text{ios}_\text{DS}}_i$ and
$t^{\text{eos}_{\text{DS}}}_i$ are, respectively, the initial and final time of the $i$-th double support phase.
\par
By combining the third-order DCM trajectory~\eqref{eq:3d_dcm_ds_base} with the boundary conditions~\eqref{eq:3d_dcm_ds_boundary}, we obtain the following:
\begin{equation}
  \label{eq:dcm_traj_constraints_vector}
  \begin{bmatrix}
    0 & 0 & 0 & 1 \\
    0 & 0 & 1 & 0 \\
    T_i^3 & T_i^2 & T_i & 1 \\
  3T_i^2 & 2T_i & 1 & 0
  \end{bmatrix}
  \begin{bmatrix}
    {a}_{3_i} ^\top\\
    {a}_{2_i} ^\top\\
    {a}_{1_i} ^\top\\
  {a}_{0_i} ^\top
  \end{bmatrix}
  =
  \begin{bmatrix}
    ({\xi}^{\text{ios}_\text{DS}}_i ) ^\top\\
    (\dot{{\xi}}^{\text{ios}_\text{DS}}_i) ^\top\\
    ({\xi}^{\text{eos}_{\text{DS}}}_i) ^\top\\
    (\dot{{\xi}}^{\text{eos}_{\text{DS}}}_i)^\top
  \end{bmatrix}
  \quad \forall i \in \left\{i \in \mathbb{N} | 1\leq i \leq N \right \}.
\end{equation}
$T_i$ is the duration of the $i$-th double support phase, i.e., $T_i = t^{\text{eos}_\text{DS}}_i - t^{\text{ios}_{\text{DS}}}_i$.
Equation \eqref{eq:dcm_traj_constraints_vector} is always solvable if and only if the $i$-th double
support duration $T_i \neq 0$. 
The position and velocity of DCM during the double support phase are finally obtained by combining the solution of~\eqref{eq:dcm_traj_constraints_vector} with~\eqref{eq:3d_dcm_ds_base}.
\par
\begin{figure}[t]
  \centering
  \includegraphics{chapter_simplified_benchmarking/figures/dcm_ds.tikz}
    \caption[DCM trajectory planning]{Planning of the DCM trajectory on a flat terrain.
    The SS trajectories, evaluated using the exponential
    function~\eqref{eq:dcm_solution_eos},
    are represented by orange segments. The DS trajectories, evaluated using the polynomial function~\eqref{eq:3d_dcm_ds_base},
    are represented by light blue curves. \label{fig:dcm_ds}}
\end{figure}
Figure~\ref{fig:dcm_ds} presents an example of DCM trajectory generation considering single and double support phases. While the robot is in single support, the DCM is generated by applying the exponential interpolation technique~\eqref{eq:dcm_solution_eos} (orange segment).
During the double support phase (light blue curves), the DCM trajectory is evaluated by means of the polynomial interpolation method~\eqref{eq:3d_dcm_ds_base}.

