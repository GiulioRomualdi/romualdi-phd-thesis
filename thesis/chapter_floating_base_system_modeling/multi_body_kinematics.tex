\section{Multi-body kinematics\label{sec:muti-body-kinematics}}
Once the floating base multi-body system has been described in the form of a connectivity graph, the position and orientation of the link $i$ can be reconstruct as follows:
\begin{equation}
    \label{eq:link-fk}
	{}^\mathcal{I} H_i = {}^\mathcal{I}H_0 \prod_{j \in \kappa(i)} {}^{\lambda(j)}H_{j},
\end{equation}
where in our case the link $0$ is the floating base of the robot, that is, $0 \equiv B$ and ${}^\mathcal{I}H_0 \equiv {}^\mathcal{I}H_B$ -- see Equation~\eqref{eq:base-pose}. The transformation matrix ${}^{\lambda(j)}H_{j}$ depends upon the position of the joint $j$, henceforth indicated with $s_j$.
To give the reader a better comprehension of Equation~\eqref{eq:link-fk}, we can imagine that ${}^\mathcal{I} H_i$ is given by the concatenation of $\SE(3)$ elements that express the link's frame to the parent. Equation~\eqref{eq:link-fk} can be obtained by exploring the connectivity graph from the link $i$ to the base link. For example, considering the tree in Figure~\ref{fig:graph-multi-body} and the link $7$, ${}^\mathcal{I} H_7$ is given by ${}^\mathcal{I} H_7 = {}^\mathcal{I} H_0 {}^0 H_6 (s_6) {}^6 H _7 (s_7)$. 
\par 
Following the geometric model in Figure~\ref{fig:geometric-model-rigid-bodies}, ${}^{\lambda(j)}H_{j}$ transformation can be seen as a composition of three transformations. ${}^{\lambda(j)}H_{j}$ writes
\begin{equation}
    {}^{\lambda(j)}H_{j} = {}^{\lambda(j)}H_{(\lambda(j), j)} {}^{(\lambda(j), j)}H_{(j, j)} {}^{(j,j)}H_{j},
\end{equation}
where ${}^{\lambda(j)}H_{(\lambda(j), j)}$ and ${}^{(j,j)}H_{j}$ give the position and orientation of the link frames with respect to the frame associated with the joint. ${}^{(\lambda(j), j)}H_{(j, j)}$ depends on the joint's position value. For a revolute joint characterized by the axis ${}_j a \in \lieGroup{S}^2$ and an angle $s_j \in \mathbb{R}$, ${}^{(\lambda(j), j)}H_{(j, j)}$ is given by
\begin{equation}
    {}^{(\lambda(j), j)}H_{(j, j)} = \Exp\left(\begin{bmatrix}
        0_{3 \times 1} \\ s_j \; {}_j a
    \end{bmatrix}\right) = \exp\left(\begin{bmatrix}
        (s_j \; {}_j a)\times & 0_{3 \times 1} \\
        0_{1 \times 3} & 0 
    \end{bmatrix}\right) 
\end{equation}
Recalling that the exponential operator for a skew-symmetric matrix given by the well-known Rodriguerz formula -- see Section~\ref{sec:so3_exponential_log}~\citep{Murray1994}, ${}^{(\lambda(j), j)}H_{(j, j)}$ writes as
\begin{equation}
\begin{split}
	{}^{(\lambda(j), j)}H_{(j, j)}(s_j) &= \begin{bmatrix}
	{}^{(\lambda(j), j)}R_{(j, j)}(s_j)  & 0_{3 \times 1} \\
	0_{1 \times 3} & 1
	\end{bmatrix} \\
	{}^{(\lambda(j), j)}R_{(j, j)}(s_j) &= I_3 + \cos(s_j)({}_j a \times) + \sin(s_j)({}_j a \times)^2.
\end{split}
\end{equation}
We now define the \emph{forward kinematics} function of a floating base system described as a \emph{kinematic tree} composed of $n_j + 1$ rigid bodes connected with $n_j$ joints as:
\begin{equation}
    \fk : \SE(3) \times \mathbb{R}^{n_j} \longrightarrow \SE(3); \quad {}^\mathcal{I} H_i = \fk\left({}^\mathcal{I}H_B, {s}\right) = {}^\mathcal{I}H_0 \prod_{j \in \kappa(i)} {}^{\lambda(j)}H_{j}.
\end{equation}
\par
We define the center of mass (CoM) of the multi-body system, denoted by $x_\text{CoM}$, as the weighted average of all the links CoM positions:
 \begin{equation}\label{eq:com_definition}
 x_\text{CoM} := {}^\mathcal{I} H_B \sum_i \frac{m_i}{m}{}^{B} H _i  ~ {}^i p _{\text{CoM}},
 \end{equation}
where ${}^i p _ {\text{CoM}} \in \mathbb{R}^3$ is the (constant) CoM position of link $i$ expressed in $i$ coordinates. $m, m_i \in \mathbb{R}$ are respectively the robot total mass and the $i$-th link mass. 
\par
Since every link pose can be computed as a function of the base pose ${}^\mathrm{I} H _B $ and the joint position values $s$, we define $q$ as the minimum set of coordinates as 
\begin{equation}
q_{\SE(3)} := ({}^\mathcal{I}H_B, s) \in \SE(3) \times  \mathbb{R}^n = \lieGroup{Q}_{\SE(3)}.
\end{equation}
$\lieGroup{Q}_{\SE(3)}$ is a composition of Lie groups, as a consequence, it is the Lie group itself. Consequently, there exists a Lie algebra $\lieAlgebra{q}_{\SE(3)} = \se(3) \times \mathbb{R}^n$ associated with $\lieGroup{Q}_{\SE(3)}$. The element of the Lie algebra is the spatial base velocity and the joint velocity.
Since in the rest of the thesis we often use the mixed spatial velocity representation, we often indicate $q$ as 
\begin{equation}
\label{eq:robot_configuration_mixed}
q_{\mathbb{R}^3 \times \SO(3)} := ({}^A o _B , {}^\mathcal{I}R_B, s) \in \mathbb{R}^3 \times \SO(3) \times  \mathbb{R}^n = \lieGroup{Q} _{\mathbb{R}^3 \times \SO(3)}.
\end{equation}
Similar to before $\lieGroup{Q} _{\mathbb{R}^3 \times \SO(3)}$ is a composition of Lie groups, so it admits a Lie algebra $\lieAlgebra{q} _{\mathbb{R}^3 \times \SO(3)} \in \mathbb{R}^3 \times \so(3) \times  \mathbb{R}^n$. The elements of the Lie algebra are the spatial base velocity in mixed representation and the joints velocity.
\par
Given a link $L$, we now aim to write a relation between its spatial velocity and the elements of $\lieAlgebra{q}_{\SE(3)}$. The following results are obtained by describing the link velocity by means of the left trivialized spatial velocity; however, similar results hold also if the velocity is expressed in right trivialization or mixed representation. Indeed, a spatial velocity in body frame, i.e., left trivialized, can be converted in inertial frame and in mixed representation thanks to Equations~\eqref{eq:left_to_right} and~\eqref{eq:mixed_velocity_definition}, respectively.  Let us now define the velocity of a link $L$ as ${}^L \mathrm{v}_{\mathcal{I}, L}$. ${}^L \mathrm{v}_{\mathcal{I}, L}$ can be decomposed into two terms as
\begin{equation}\label{eq:velocity_inertial}
	{}^L\mathrm{v}_{\mathcal{I}, L} = {}^L X _B {}^B\mathrm{v}_{\mathcal{I}, B} + {}^L\mathrm{v}_{B, L},
\end{equation}
where ${}^B\mathrm{v}_{\mathcal{I}, B}$ is the left trivialized base velocity, while ${}^L\mathrm{v}_{B, L}$ is a function of the position of the joints $s\in \mathbb{R}^{n_j}$ and the velocity $\dot{s} \in \mathbb{R}^{n_j}$. In fact ${}^L\mathrm{v}_{B, L}$ is given by 
\begin{IEEEeqnarray}{RCL}
\phantomsection \IEEEyesnumber \label{eq:link_velocity_expanded} \IEEEyessubnumber*
	{}^L\mathrm{v}_{B, L} ^ \wedge &=& {}^{B} H _{L}^{-1} {}^B\dot{H}_{L}\\
	&=& \sum_{j \in \kappa(L)}\left({}^L{H}_{\lambda(j)}  \frac{\partial}{\partial s_j}\left({}^{\lambda(j)}H_{j}\right)  {}^{j}{H}_{L}\right) \dot{s}_j  \\
	&=& \sum_{j \in \kappa(L)}\left({}^L {H}_{j} {}^{\lambda(j)}{H}_{j}^{-1} \frac{\partial}{\partial s_j}\left({}^{\lambda(j)}H_{j}\right)  {}^{j}{H}_{L}\right) \dot{s}_j \label{eq:link_velocity_expanded_ad}\\
	&=& \sum_{j \in \kappa(L)}\left[  {}^L X_{j} \left({}^{\lambda(j)}H_{j}^{-1} \frac{\partial}{\partial s_j}\left({}^{\lambda(j)}H_{j}\right) \right) ^ \vee \right] ^\wedge \dot{s}_j \label{eq:link_velocity_expanded_adm},
\end{IEEEeqnarray}
where from \eqref{eq:link_velocity_expanded_ad} to \eqref{eq:link_velocity_expanded_adm}  we used the fact that the adjoint action is a linear transformation so it can be associated to the adjoint matrix, i.e. 
\begin{IEEEeqnarray}{ll}
\phantomsection \IEEEyesnumber \IEEEyessubnumber*
  {}^L {H}_{j} {}^{\lambda(j)}{H}_{j}^{-1} \frac{\partial}{\partial s_j}\left({}^{\lambda(j)}H_{j}\right)  {}^{j}{H}_{L} &= \Ad_{{}^L H_{\lambda(j)}} \left( {}^{\lambda(j)}{H}_{j}^{-1} \frac{\partial}{\partial s_j}\left(   {}^{\lambda(j)}H_{j}\right) \right) \\
  &= \left[\AdM_{{}^L H_{\lambda(j)}}  \left({}^{\lambda(j)}{H}_{j}^{-1} \frac{\partial}{\partial s_j}\left(   {}^{\lambda(j)}H_{j}\right) \right)^\vee \right] ^\wedge  \\ 
  &= \left[  {}^L X_{j} \left({}^{\lambda(j)}H_{j}^{-1} \frac{\partial}{\partial s_j}\left({}^{\lambda(j)}H_{j}\right) \right) ^ \vee \right] ^\wedge.
\end{IEEEeqnarray}

Let us now introduce the \emph{left-trivialized joint motion subspace} ${}^{j}\textbf{s} $ as 
\begin{equation}
\label{eq:left-trivialized_joint_motion_subspace_def}
    {}^{j}\textbf{s} = \left({}^{\lambda(j)}H_{j}^{-1} \frac{\partial}{\partial s_j}\left({}^{\lambda(j)}H_{j}\right) \right) ^ \vee,
\end{equation}
equation~\eqref{eq:link_velocity_expanded} becomes
\begin{equation}
    \label{eq:link_velocity_expanded_motion_subspace}
    {}^L\mathrm{v}_{B, L} = \sum_{j \in \kappa{L}} {}^LX _{j}\,{}^{j}\textbf{s}\, \dot{s}_j.
\end{equation}
We observe that, in the case of a revolute joint, ${}^{j}\textbf{s}$ is constant and it depends on the revolute axis~\citep{traversaro2017}.
\par
Since ${}^L\mathrm{v}_{B, L}$ is an affine function of joints velocity $\dot{s} \in \mathbb{R}^{n_j}$, Equation \eqref{eq:link_velocity_expanded_motion_subspace} can be written as 
\begin{equation}
	{}^L\mathrm{v}_{B, L} = {}^L J _{B, L}^{\dot{s}}(s)  \dot{s},
\end{equation}
In summary, the left trivialized link velocity is as follows:
\begin{equation}
\label{eq:left-tivialized-link-velocity}
	{}^L\mathrm{v}_{\mathcal{I}, L} = \begin{bmatrix}
	{}^LX_B & {}^LJ_{B, L}^{\dot{s}}
	\end{bmatrix} \begin{bmatrix}
	{}^B\mathrm{v}_{\mathcal{I}, B} \\
	\dot{s}
	\end{bmatrix} = {}^LJ^B_{\mathcal{I}, L} \, {}^B {\nu}.
\end{equation}
${}^LJ^B_{\mathcal{I}, L} \in \mathbb{R}^{6\times (6+n_j)}$ is the \emph{left-trivialized Jacobian} of the link $L$, ${}^B \nu \in \mathbb{R}^{6+n_j}$ is the left trivialized multi-body system velocity and ${}^B \nu ^\wedge \in \mathfrak{q}_{\SE(3)}$:
\begin{equation}
    \label{eq:right_generalized_robot_velocity}
    {}^B\nu =  \begin{bmatrix}
	{}^B\mathrm{v}_{\mathcal{I}, B} \\
	\dot{s}
	\end{bmatrix}.
\end{equation}
\par
Given a left trivialized link velocity ${}^L\mathrm{v}_{\mathcal{I}, L}$, we obtain the associated mixed representation as
\begin{IEEEeqnarray}{ll}
\phantomsection \IEEEyesnumber \label{eq:mixed-link-velocity} \IEEEyessubnumber*
	{}^{L[\mathcal{I}]}\mathrm{v}_{\mathcal{I}, L}& = 	{}^{L[\mathcal{I}]} X _ L
	\begin{bmatrix}
	{}^LX_B & {}^LJ_{B, L}^{\dot{s}}
	\end{bmatrix}
	\begin{bmatrix} 
	{}^{B} X _ {B[\mathcal{I}]} & 0_{6\times n_j} \\
	0_{n_j\times 6} & I_{n_j} 
	\end{bmatrix} 
	\begin{bmatrix}
	{}^{B[\mathcal{I}]}\mathrm{v}_{\mathcal{I}, B} \\
	\dot{s}
	\end{bmatrix} \\ &=
	{}^{L[\mathcal{I}]}J_{\mathcal{I}, L} ^{B[\mathcal{I}]} \, {}^{B[\mathcal{I}]}\nu.
\end{IEEEeqnarray}
where ${}^{L[\mathcal{I}]}J_{\mathcal{I}, L} ^{B[\mathcal{I}]} \in \mathbb{R}^{6\times (6+n_j)}$ is the \emph{mixed velocity Jacobian} of link $L$ and ${}^{B[\mathcal{I}]} \nu \in \mathbb{R}^{6+n_j}$ is the mixed multi-body system velocity
\begin{equation}
    \label{eq:mixed_generalized_robot_velocity}
    {}^{B[\mathcal{I}]} \nu = \begin{bmatrix}
	{}^{B[\mathcal{I}]}\mathrm{v}_{\mathcal{I}, B} \\
	\dot{s}
	\end{bmatrix}.
\end{equation}
\par
To simplify the notation, we will simply assume that we use the mixed representation, i.e. ${}^{L[\mathcal{I}]}J_{\mathcal{I}, L} ^{B[\mathcal{I}]}$ will be simply indicated $J_L$. Similarly, ${}^{B[\mathcal{I}]}\nu$ will become $\nu$.
