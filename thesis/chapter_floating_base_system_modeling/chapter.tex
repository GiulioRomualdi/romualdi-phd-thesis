
\chapter{Modeling of Floating Base Multi-Body Systems \label{chapter:floating_base_system_modeling}}  %

\ifpdf
    \graphicspath{{ChapterFloatingBaseSystemModeling/figures/Raster/}{ChapterFloatingBaseSystemModeling/figures/PDF/}{ChapterFloatingBaseSystemModeling/figures/}}
\else
    \graphicspath{{ChapterFloatingBaseSystemModeling/figures/Vector/}{ChapterFloatingBaseSystemModeling/figures/}}
\fi

In the previous chapter, we introduced the rotation and the rototranslation matrix Lie Group and the associated rigid body dynamics. We now exploit these concepts to describe the kinematics and the dynamics of a floating base system. 
What is presented in this chapter is crucial to fully understand the design of the controllers and the estimators presented in Part~\ref{part:wbc} and Part~\ref{part:simplified}.
The chapter is organized as follows. In Section~\ref{sec:floating-base-multi-body-system}, we describe a multi-body system in terms of interaction between links and joints. Such a description is often denoted \emph{system model}~\citep{Featherstone2014}. Section~\ref{sec:muti-body-kinematics} discusses the kinematics of the multi-body system. Here, we present the \emph{minimum set of coordinates} $q$ as an element of a Lie group $\lieGroup{Q}$. We also introduce the \emph{velocity of the multi-body system} $\nu$ as an element of the Lie algebra $\lieAlgebra{q}$. Section~\ref{sec:muti-body-kinematics} also contains the definition of the \emph{forward kinematics} function and the relation between the link velocity and the velocity of the multi-body system $\nu$. Section~\ref{sec:multi-body-dynamics} presents the dynamics equation of the multi-body system, since the multi-body configuration space is not a vector space, the dynamics is obtained by extending the Euler-Poincar\'e presented in Section~\ref{sec:rigid-body-dynamics} to case of multi-body systems~\citep{Marsden2010}.
Finally, Section~\ref{sec:centroidal-dynamics} discusses the centroidal dynamics~\citep{Orin2013} as the aggregate linear and angular momentum of each link of the robot referred to the center of mass of the robot.



\section{Floating base multi-body system modeling\label{sec:floating-base-multi-body-system}}
A \emph{rigid-body system} is an assembly of component parts, namely: $n_b$ \emph{rigid bodies} and $n_j$ \emph{joints} which are responsible for the kinematic constraints in the system. In this section, we present a formalism used to describe rigid-body systems in terms of graphs. In particular, we seek to describe the interaction between bodies and joints as an element of nodes and the arch of a graph. We call such a description \emph{system model} with respect to an inertial frame $\mathcal{I}$. 
\par
A system model is often expressed in the form of a \emph{connectivity graph} having the following properties:
\begin{itemize}
    \item Each node represents one rigid body in the dynamical system. We assume that at least one frame is rigidly attached to each body; 
    \item Each arc represents a joint;
    \item There exists one node, i.e. a rigid body, which is called \emph{base} of the multi-body system. The base is indicated by the frame $B$;
    \item The graph is undirected;
    \item The graph is connected, i.e., there exists at least one path between any two nodes. 
\end{itemize}
If the base of the multi-body system does not have an \emph{a priori} fixed pose with respect to an inertial frame, $\mathcal{I}$, the associated body is called \emph{floating base}. In this case, the pose (position and orientation) of the floating base $B$ with respect to the inertial frame $\mathcal{I}$ is given by the element of the $\SE(3)$ and it writes as:
\begin{equation}
\label{eq:base-pose}
{}^\mathcal{I} H _B = \begin{bmatrix}
{}^\mathcal{I} R _B & {}^\mathcal{I} p _B \\
{0}_{1 \times 3} & 1
\end{bmatrix}.
\end{equation}
A graph is a topological tree if there exists exactly one path between any two
nodes in the graph. If the connectivity graph of a rigid-body system is a topological tree, then we call the system itself a \emph{kinematic tree}. If the rigid-body system is represented by a kinematic tree, there exists a well-defined relation between the number of joints $n_j$, and the number of rigid bodies $n_b$, that is,
\begin{equation}
    n_b = n_j + 1.
\end{equation}
From now on, we consider only floating base multi-body systems characterized by a kinematic tree. 
\par
Once we describe the rigid-body system as a graph, we aim to number the nodes and arcs following the \emph{regular numbering} scheme~\citep{Featherstone2014}. We now present the scheme for the topological tree. The generic scheme that applies to the generic graph is described in ~\citep[Section 4.1.2]{Featherstone2014}.
Given a kinematic tree $G$, we denote the elements as follows:
\begin{itemize}
    \item The base is represented by the number $0$;
    \item The remaining $n_b - 1$ nodes are numbered in any order such that each node has a higher number than its parent in $G$ ;
    \item The arcs are identified with a number for $1$ to $n_j$ such that the arc $i$ connects between node $i$ and its parent.
\end{itemize}

\begin{figure}[tpb]
\centering
    \begin{subfigure}[b]{1\textwidth}
        \centering
        \includegraphics{chapter_floating_base_system_modeling/figures/tree-multi-body.tikz}
        \caption{Regular numbering applied to an unbranched kinematic tree}
        \label{fig:tree-multi-body}
    \end{subfigure}
    \vskip 0.5cm
    \begin{subfigure}[b]{1\textwidth}
        \centering
        \includegraphics{chapter_floating_base_system_modeling/figures/graph-multi-body.tikz}
        \caption{Regular numbering applied to a kinematic tree}
        \label{fig:graph-multi-body}
    \end{subfigure}
    \caption[Schematic representation of a multi-body structure.]{Schematic representation of a multi-body structure. The links are represented by the graph nodes, while joints are the graph's arcs. The links and joints are named by applying the Regular numbering principle.}
\end{figure}
If these principles are applied to an unbranched kinematic tree (one with no more than one child), the bodies and joints are numbered sequentially from the base, as shown in Figure~\ref{fig:tree-multi-body}. On the other hand, if the tree is composed of multiple branches, the numbering is not unique. Figure~\ref{fig:graph-multi-body} represents one of the possible examples.
\par
A connectivity graph is fully determined by a pair of arrays named $p$ and $s$, such that the element $i$ in $p$ is the vector of the predecessor of the joint $i$ while the element $i$ in $s$ contains the successor of the joint $i$. 
Taking as an example the unbranched kinematic tree in Figure~\ref{fig:tree-multi-body} the arrays $s$ and $p$ are given by
\begin{equation}
    p  = \begin{bmatrix}
    0 &1&\dots&n_j-2 & &n_j-1 
    \end{bmatrix}, \quad 
    s  = \begin{bmatrix}
    1 &2&\dots&n_j-1 & &n_j 
    \end{bmatrix}.
\end{equation}
On the other hand,  the tree in Figure~\ref{fig:graph-multi-body} is described by:
\begin{equation}
    p = \begin{bmatrix}
    0 & 1 & 0 & 3 & 4 & 0 & 0
    \end{bmatrix}, \quad 
    s = \begin{bmatrix}
    1 &2&3&4&5&6&7&8 
    \end{bmatrix}.
\end{equation}
\par
The \emph{parent array} $\lambda$ identifies the parent of each body. $\lambda(i)$ is the parent of the body $i$.
We finally introduce for any body $i$ three sets, namely: $\kappa(i)$,  $\mu(i)$ and $\nu(i)$. $\kappa(i)$ is the set of all joints on the path between body $i$ and base. If $i=0$, i.e. the body is the base, the set is empty. 
$\mu(i)$ is the set of the children of body $i$, and $\nu(i)$ is the set of bodies in the subtree starting at body $i$. $\kappa(i)$,  $\mu(i)$ and $\nu(i)$ are called the \emph{support}, \emph{child} and \emph{subtree} sets, respectively. 
Given the tree in Figure~\ref{fig:graph-multi-body} the \emph{support}, \emph{child} and \emph{subtree} sets are given by
\begin{IEEEeqnarray}{lll}
\phantomsection \IEEEyesnumber \IEEEyessubnumber*
\kappa(0) = \emptyset \quad & \mu(0) = \{1,3,6,8\}\quad & \nu(0) = \{0,1,2,3,4,5,6,7,8\} \\
\kappa(1) = \{1\}\quad  & \mu(1) = \{2\}\quad & \nu(1) = \{1,2\} \\
\kappa(2) = \{1, 2\} \quad & \mu(2) = \emptyset \quad& \nu(2) = \{2\} \\
\kappa(3) = \{3\} \quad & \mu(3) = \{4\}\quad & \nu(3) = \{3,4,5\} \\
\kappa(4) = \{3,4\} \quad & \mu(4) = \{5\} \quad& \nu(4) = \{4,5\} \\
\kappa(5) = \{3,4,5\} \quad & \mu(5) = \emptyset \quad& \nu(5) = \{5\} \\
\kappa(6) = \{6\}\quad  & \mu(6) = \{7\}\quad & \nu(6) = \{6,7\} \\
\kappa(7) = \{6,7\}\quad  & \mu(7) = \emptyset \quad& \nu(7) = \{7\} \\
\kappa(8) = \{8\}\quad  & \mu(8) = \emptyset \quad  & \nu(8) = \{8\}.
\end{IEEEeqnarray}
\par
When a joint connects two bodies, its relative motion is limited. The mobility allowed by the joint and the placement of the joint relative to each body are necessary for a thorough description of the restriction. A joint model explains the former. The geometry of the system determines the latter.

\paragraph{The joint model}
In this context, a joint may be thought of as a motion constraint between two Cartesian frames, one of which is embedded in the body $i$ and the other in its parent, $\lambda(i)$.
The kind of mobility that a joint allows is determined by its type of joint. A \emph{revolute} joint, for example, provides for pure rotation along a single axis, but a spherical joint allows for unrestricted rotation around a single point. Consequently, we should introduce the admissible motions for each type of joint. However, since in this thesis we assume that the floating base system is composed only of one-degree-of-freedom revolute joints, we analyze the joint characteristics only for this kind of joint. A revolute joint $j$ is completely characterized by an axis ${}_j a \in \lieGroup{S}^2$ and an angle $s_j$, where $\lieGroup{S}^2$ is the set of 3D vectors with norm equal to 1, i.e., $\lieGroup{S}^2 = \left\{ x\in \mathbb{R}^3 | x^\top x = 1 \right\}$ and $s_j \in \mathbb{R}$. 

\begin{figure}[t]
    \centering
	\includegraphics{chapter_floating_base_system_modeling/figures/rigid-bodies.tikz}
	\caption[Geometric model of a rigid-body system.]{Geometric model of a rigid-body system. The red frames are the \emph{link frames}. The green frames are the \emph{joint frames}. Each arrow represents a homogeneous transformation. A black arrow indicates a constant transformation, while a blue one a joint state-dependent transformation.}
	\label{fig:geometric-model-rigid-bodies}
\end{figure}


\paragraph{Geometric model}
The geometric model describes the location of each joint in each rigid body. In detail, each body $i$ is characterized by a set of frames. One, known as \emph{link frame}, identifies the position of the link. The others are placed in the joints, $j$. Figure~\ref{fig:geometric-model-rigid-bodies} represents the geometry model of a rigid-body system. Here, each rigid body has a frame embedded in it, which defines a local coordinate system for that body. In Figure~\ref{fig:geometric-model-rigid-bodies} these frames are labeled as $B_i$ where $i$ represents the body index. Figure~\ref{fig:geometric-model-rigid-bodies} also presents several frames with names of the form $B_{(i,j)}$ where $i$ refers to a body and $j$ to the associated joint. The position and the orientation of the frame $B_{(i,j)}$ with respect to $B_i$ are constant and are given by the transformation ${}^{(i,j)} H _i$. Finally, each joint describes a joint transform labeled ${}^{(i,k)} H _ {(j, k)}$ where $i$ refers to the parent body, $j$ to the child body, and $k$ to the joint. ${}^{(i,k)} H _ {(j, k)}$ depends only on the joint position. 
From now on we assume that, given two rigid bodies $i$ and $j$ connected by a joint $k$, the transformation ${}^{(i,k)} H _i$ and ${}^{(j,k)} H _j$ are chosen so that ${}^{(i,k)} H _ {(j, k)} = I_4$ only if the joint position state is equal to 0, i.e., ${}^{(i,k)} H _ {(j, k)}(s_j) =  {}^{(i,k)} H _ {(j, k)}(0) = I_4$.



\section{Multi-body kinematics\label{sec:muti-body-kinematics}}
Once the floating base multi-body system has been described in the form of a connectivity graph, the position and orientation of the link $i$ can be reconstruct as follows:
\begin{equation}
    \label{eq:link-fk}
	{}^\mathcal{I} H_i = {}^\mathcal{I}H_0 \prod_{j \in \kappa(i)} {}^{\lambda(j)}H_{j},
\end{equation}
where in our case the link $0$ is the floating base of the robot, that is, $0 \equiv B$ and ${}^\mathcal{I}H_0 \equiv {}^\mathcal{I}H_B$ -- see Equation~\eqref{eq:base-pose}. The transformation matrix ${}^{\lambda(j)}H_{j}$ depends upon the position of the joint $j$, henceforth indicated with $s_j$.
To give the reader a better comprehension of Equation~\eqref{eq:link-fk}, we can imagine that ${}^\mathcal{I} H_i$ is given by the concatenation of $\SE(3)$ elements that express the link's frame to the parent. Equation~\eqref{eq:link-fk} can be obtained by exploring the connectivity graph from the link $i$ to the base link. For example, considering the tree in Figure~\ref{fig:graph-multi-body} and the link $7$, ${}^\mathcal{I} H_7$ is given by ${}^\mathcal{I} H_7 = {}^\mathcal{I} H_0 {}^0 H_6 (s_6) {}^6 H _7 (s_7)$. 
\par 
Following the geometric model in Figure~\ref{fig:geometric-model-rigid-bodies}, ${}^{\lambda(j)}H_{j}$ transformation can be seen as a composition of three transformations. ${}^{\lambda(j)}H_{j}$ writes
\begin{equation}
    {}^{\lambda(j)}H_{j} = {}^{\lambda(j)}H_{(\lambda(j), j)} {}^{(\lambda(j), j)}H_{(j, j)} {}^{(j,j)}H_{j},
\end{equation}
where ${}^{\lambda(j)}H_{(\lambda(j), j)}$ and ${}^{(j,j)}H_{j}$ give the position and orientation of the link frames with respect to the frame associated with the joint. ${}^{(\lambda(j), j)}H_{(j, j)}$ depends on the joint's position value. For a revolute joint characterized by the axis ${}_j a \in \lieGroup{S}^2$ and an angle $s_j \in \mathbb{R}$, ${}^{(\lambda(j), j)}H_{(j, j)}$ is given by
\begin{equation}
    {}^{(\lambda(j), j)}H_{(j, j)} = \Exp\left(\begin{bmatrix}
        0_{3 \times 1} \\ s_j \; {}_j a
    \end{bmatrix}\right) = \exp\left(\begin{bmatrix}
        (s_j \; {}_j a)\times & 0_{3 \times 1} \\
        0_{1 \times 3} & 0 
    \end{bmatrix}\right) 
\end{equation}
Recalling that the exponential operator for a skew-symmetric matrix given by the well-known Rodriguerz formula -- see Section~\ref{sec:so3_exponential_log}~\citep{Murray1994}, ${}^{(\lambda(j), j)}H_{(j, j)}$ writes as
\begin{equation}
\begin{split}
	{}^{(\lambda(j), j)}H_{(j, j)}(s_j) &= \begin{bmatrix}
	{}^{(\lambda(j), j)}R_{(j, j)}(s_j)  & 0_{3 \times 1} \\
	0_{1 \times 3} & 1
	\end{bmatrix} \\
	{}^{(\lambda(j), j)}R_{(j, j)}(s_j) &= I_3 + \cos(s_j)({}_j a \times) + \sin(s_j)({}_j a \times)^2.
\end{split}
\end{equation}
We now define the \emph{forward kinematics} function of a floating base system described as a \emph{kinematic tree} composed of $n_j + 1$ rigid bodes connected with $n_j$ joints as:
\begin{equation}
    \fk : \SE(3) \times \mathbb{R}^{n_j} \longrightarrow \SE(3); \quad {}^\mathcal{I} H_i = \fk\left({}^\mathcal{I}H_B, {s}\right) = {}^\mathcal{I}H_0 \prod_{j \in \kappa(i)} {}^{\lambda(j)}H_{j}.
\end{equation}
\par
We define the center of mass (CoM) of the multi-body system, denoted by $x_\text{CoM}$, as the weighted average of all the links CoM positions:
 \begin{equation}\label{eq:com_definition}
 x_\text{CoM} := {}^\mathcal{I} H_B \sum_i \frac{m_i}{m}{}^{B} H _i  ~ {}^i p _{\text{CoM}},
 \end{equation}
where ${}^i p _ {\text{CoM}} \in \mathbb{R}^3$ is the (constant) CoM position of link $i$ expressed in $i$ coordinates. $m, m_i \in \mathbb{R}$ are respectively the robot total mass and the $i$-th link mass. 
\par
Since every link pose can be computed as a function of the base pose ${}^\mathrm{I} H _B $ and the joint position values $s$, we define $q$ as the minimum set of coordinates as 
\begin{equation}
q_{\SE(3)} := ({}^\mathcal{I}H_B, s) \in \SE(3) \times  \mathbb{R}^n = \lieGroup{Q}_{\SE(3)}.
\end{equation}
$\lieGroup{Q}_{\SE(3)}$ is a composition of Lie groups, as a consequence, it is the Lie group itself. Consequently, there exists a Lie algebra $\lieAlgebra{q}_{\SE(3)} = \se(3) \times \mathbb{R}^n$ associated with $\lieGroup{Q}_{\SE(3)}$. The element of the Lie algebra is the spatial base velocity and the joint velocity.
Since in the rest of the thesis we often use the mixed spatial velocity representation, we often indicate $q$ as 
\begin{equation}
\label{eq:robot_configuration_mixed}
q_{\mathbb{R}^3 \times \SO(3)} := ({}^A o _B , {}^\mathcal{I}R_B, s) \in \mathbb{R}^3 \times \SO(3) \times  \mathbb{R}^n = \lieGroup{Q} _{\mathbb{R}^3 \times \SO(3)}.
\end{equation}
Similar to before $\lieGroup{Q} _{\mathbb{R}^3 \times \SO(3)}$ is a composition of Lie groups, so it admits a Lie algebra $\lieAlgebra{q} _{\mathbb{R}^3 \times \SO(3)} \in \mathbb{R}^3 \times \so(3) \times  \mathbb{R}^n$. The elements of the Lie algebra are the spatial base velocity in mixed representation and the joints velocity.
\par
Given a link $L$, we now aim to write a relation between its spatial velocity and the elements of $\lieAlgebra{q}_{\SE(3)}$. The following results are obtained by describing the link velocity by means of the left trivialized spatial velocity; however, similar results hold also if the velocity is expressed in right trivialization or mixed representation. Indeed, a spatial velocity in body frame, i.e., left trivialized, can be converted in inertial frame and in mixed representation thanks to Equations~\eqref{eq:left_to_right} and~\eqref{eq:mixed_velocity_definition}, respectively.  Let us now define the velocity of a link $L$ as ${}^L \mathrm{v}_{\mathcal{I}, L}$. ${}^L \mathrm{v}_{\mathcal{I}, L}$ can be decomposed into two terms as
\begin{equation}\label{eq:velocity_inertial}
	{}^L\mathrm{v}_{\mathcal{I}, L} = {}^L X _B {}^B\mathrm{v}_{\mathcal{I}, B} + {}^L\mathrm{v}_{B, L},
\end{equation}
where ${}^B\mathrm{v}_{\mathcal{I}, B}$ is the left trivialized base velocity, while ${}^L\mathrm{v}_{B, L}$ is a function of the position of the joints $s\in \mathbb{R}^{n_j}$ and the velocity $\dot{s} \in \mathbb{R}^{n_j}$. In fact ${}^L\mathrm{v}_{B, L}$ is given by 
\begin{IEEEeqnarray}{RCL}
\phantomsection \IEEEyesnumber \label{eq:link_velocity_expanded} \IEEEyessubnumber*
	{}^L\mathrm{v}_{B, L} ^ \wedge &=& {}^{B} H _{L}^{-1} {}^B\dot{H}_{L}\\
	&=& \sum_{j \in \kappa(L)}\left({}^L{H}_{\lambda(j)}  \frac{\partial}{\partial s_j}\left({}^{\lambda(j)}H_{j}\right)  {}^{j}{H}_{L}\right) \dot{s}_j  \\
	&=& \sum_{j \in \kappa(L)}\left({}^L {H}_{j} {}^{\lambda(j)}{H}_{j}^{-1} \frac{\partial}{\partial s_j}\left({}^{\lambda(j)}H_{j}\right)  {}^{j}{H}_{L}\right) \dot{s}_j \label{eq:link_velocity_expanded_ad}\\
	&=& \sum_{j \in \kappa(L)}\left[  {}^L X_{j} \left({}^{\lambda(j)}H_{j}^{-1} \frac{\partial}{\partial s_j}\left({}^{\lambda(j)}H_{j}\right) \right) ^ \vee \right] ^\wedge \dot{s}_j \label{eq:link_velocity_expanded_adm},
\end{IEEEeqnarray}
where from \eqref{eq:link_velocity_expanded_ad} to \eqref{eq:link_velocity_expanded_adm}  we used the fact that the adjoint action is a linear transformation so it can be associated to the adjoint matrix, i.e. 
\begin{IEEEeqnarray}{ll}
\phantomsection \IEEEyesnumber \IEEEyessubnumber*
  {}^L {H}_{j} {}^{\lambda(j)}{H}_{j}^{-1} \frac{\partial}{\partial s_j}\left({}^{\lambda(j)}H_{j}\right)  {}^{j}{H}_{L} &= \Ad_{{}^L H_{\lambda(j)}} \left( {}^{\lambda(j)}{H}_{j}^{-1} \frac{\partial}{\partial s_j}\left(   {}^{\lambda(j)}H_{j}\right) \right) \\
  &= \left[\AdM_{{}^L H_{\lambda(j)}}  \left({}^{\lambda(j)}{H}_{j}^{-1} \frac{\partial}{\partial s_j}\left(   {}^{\lambda(j)}H_{j}\right) \right)^\vee \right] ^\wedge  \\ 
  &= \left[  {}^L X_{j} \left({}^{\lambda(j)}H_{j}^{-1} \frac{\partial}{\partial s_j}\left({}^{\lambda(j)}H_{j}\right) \right) ^ \vee \right] ^\wedge.
\end{IEEEeqnarray}

Let us now introduce the \emph{left-trivialized joint motion subspace} ${}^{j}\textbf{s} $ as 
\begin{equation}
\label{eq:left-trivialized_joint_motion_subspace_def}
    {}^{j}\textbf{s} = \left({}^{\lambda(j)}H_{j}^{-1} \frac{\partial}{\partial s_j}\left({}^{\lambda(j)}H_{j}\right) \right) ^ \vee,
\end{equation}
equation~\eqref{eq:link_velocity_expanded} becomes
\begin{equation}
    \label{eq:link_velocity_expanded_motion_subspace}
    {}^L\mathrm{v}_{B, L} = \sum_{j \in \kappa{L}} {}^LX _{j}\,{}^{j}\textbf{s}\, \dot{s}_j.
\end{equation}
We observe that, in the case of a revolute joint, ${}^{j}\textbf{s}$ is constant and it depends on the revolute axis~\citep{traversaro2017}.
\par
Since ${}^L\mathrm{v}_{B, L}$ is an affine function of joints velocity $\dot{s} \in \mathbb{R}^{n_j}$, Equation \eqref{eq:link_velocity_expanded_motion_subspace} can be written as 
\begin{equation}
	{}^L\mathrm{v}_{B, L} = {}^L J _{B, L}^{\dot{s}}(s)  \dot{s},
\end{equation}
In summary, the left trivialized link velocity is as follows:
\begin{equation}
\label{eq:left-tivialized-link-velocity}
	{}^L\mathrm{v}_{\mathcal{I}, L} = \begin{bmatrix}
	{}^LX_B & {}^LJ_{B, L}^{\dot{s}}
	\end{bmatrix} \begin{bmatrix}
	{}^B\mathrm{v}_{\mathcal{I}, B} \\
	\dot{s}
	\end{bmatrix} = {}^LJ^B_{\mathcal{I}, L} \, {}^B {\nu}.
\end{equation}
${}^LJ^B_{\mathcal{I}, L} \in \mathbb{R}^{6\times (6+n_j)}$ is the \emph{left-trivialized Jacobian} of the link $L$, ${}^B \nu \in \mathbb{R}^{6+n_j}$ is the left trivialized multi-body system velocity and ${}^B \nu ^\wedge \in \mathfrak{q}_{\SE(3)}$:
\begin{equation}
    \label{eq:right_generalized_robot_velocity}
    {}^B\nu =  \begin{bmatrix}
	{}^B\mathrm{v}_{\mathcal{I}, B} \\
	\dot{s}
	\end{bmatrix}.
\end{equation}
\par
Given a left trivialized link velocity ${}^L\mathrm{v}_{\mathcal{I}, L}$, we obtain the associated mixed representation as
\begin{IEEEeqnarray}{ll}
\phantomsection \IEEEyesnumber \label{eq:mixed-link-velocity} \IEEEyessubnumber*
	{}^{L[\mathcal{I}]}\mathrm{v}_{\mathcal{I}, L}& = 	{}^{L[\mathcal{I}]} X _ L
	\begin{bmatrix}
	{}^LX_B & {}^LJ_{B, L}^{\dot{s}}
	\end{bmatrix}
	\begin{bmatrix} 
	{}^{B} X _ {B[\mathcal{I}]} & 0_{6\times n_j} \\
	0_{n_j\times 6} & I_{n_j} 
	\end{bmatrix} 
	\begin{bmatrix}
	{}^{B[\mathcal{I}]}\mathrm{v}_{\mathcal{I}, B} \\
	\dot{s}
	\end{bmatrix} \\ &=
	{}^{L[\mathcal{I}]}J_{\mathcal{I}, L} ^{B[\mathcal{I}]} \, {}^{B[\mathcal{I}]}\nu.
\end{IEEEeqnarray}
where ${}^{L[\mathcal{I}]}J_{\mathcal{I}, L} ^{B[\mathcal{I}]} \in \mathbb{R}^{6\times (6+n_j)}$ is the \emph{mixed velocity Jacobian} of link $L$ and ${}^{B[\mathcal{I}]} \nu \in \mathbb{R}^{6+n_j}$ is the mixed multi-body system velocity
\begin{equation}
    \label{eq:mixed_generalized_robot_velocity}
    {}^{B[\mathcal{I}]} \nu = \begin{bmatrix}
	{}^{B[\mathcal{I}]}\mathrm{v}_{\mathcal{I}, B} \\
	\dot{s}
	\end{bmatrix}.
\end{equation}
\par
To simplify the notation, we will simply assume that we use the mixed representation, i.e. ${}^{L[\mathcal{I}]}J_{\mathcal{I}, L} ^{B[\mathcal{I}]}$ will be simply indicated $J_L$. Similarly, ${}^{B[\mathcal{I}]}\nu$ will become $\nu$.

\section{Multi-body dynamics}
\label{sec:multi-body-dynamics}
As mentioned in Section~\ref{sec:muti-body-kinematics}, the robot configuration space is characterized by the \emph{position} and the \emph{orientation} of the base frame $B$ with respect to the inertial frame $\mathcal{I}$, and the joints' position coordinates $s$. Hereafter, we indicate the robot configuration with the following triplet $q = ({}^{\mathcal{I}} p_B, {}^{\mathcal{I}} R_{B}, s)$. $q$ is an element of a Lie group, $\lieGroup{Q} = \mathbb{R}^3 \times \SO(3) \times \mathbb{R}^n$. The associated Lie algebra is denoted as $\lieAlgebra{q} = \mathbb{R}^3 \times \so(3) \times \mathbb{R}^n$, where $\lieAlgebra{q}$ is isomorphic to $\mathbb{R}^3 \times \mathbb{R}^3 \times \mathbb{R}^n$, i.e. $\lieAlgebra{q} \approx\mathbb{R}^3 \times \mathbb{R}^3 \times \mathbb{R}^n$. The \emph{velocity of the multi-body system} belongs to $\lieAlgebra{q}$. We define it as the following triplet $\nu = \left({}^{\mathcal{I}} \dot{p}_B, {}^{B[\mathcal{I}]}\dot{\omega}_{\mathcal{I}, B}, \dot{s}\right)$.
From here on, we assume that the multi-body system interacts with the environment, exchanging $n_c$ distinct 6D spatial forces -- see Section~\ref{sec:6d-spatial-force}. Since the configuration space is not a vector space, we cannot apply the classical Euler-Lagrange equations. This is solved by employing the Euler-Poincar\'e formalism \citep[Chapter 13.5]{Marsden2010}, obtaining as a final result:
\begin{equation}
\label{eq:system_initial}
M(q)\dot{\nu} + C(q, \nu)\nu + G(q) =  \begin{bmatrix}
0_{6\times n} \\ I_n
\end{bmatrix}\tau_s + \sum_{k = 1}^{n_c} J^\top_{\mathcal{C}_k} \; {}_{\mathcal{C}_k[\mathcal{I}]}\mathrm{f}_k.
\end{equation}
Here $M\in \mathbb{R}^{(n+6) \times (n+6)}$ is the mass matrix, $C \in \mathbb{R}^{(n+6) \times (n+6)}$ accounts for Coriolis and centrifugal effects, and $G \in \mathbb{R}^{n+6}$ contains the gravity term. $\tau_s \in \mathbb{R}^{n}$ is a vector representing the internal actuation torques, ${}_{\mathcal{C}_k[\mathcal{I}]}\mathrm{f}_k \in \mathbb{R}^{6}$ denotes the $k$-th external wrench applied by the environment on the robot expressed in mixed representation.
The Jacobian $J_{\mathcal{C}_k}$ is the mixed velocity Jacobian of the contact frame $\mathcal{C}_k$.
Equation~\eqref{eq:system_initial} is strictly related to the rigid body dynamics presented in Section~\ref{sec:rigid-body-dynamics}. In fact, we introduce the \emph{left-tirvialized Lagrangian of the multi-body system} $\mathcal{L}(q,\nu)$ as 
\begin{equation}
    \label{eq:lagrangian_multi-body}
    \mathcal{L}(q,\nu) = \mathcal{K}(\nu) - \mathcal{U}(q)
\end{equation}
where $\mathcal{K}$ and $\mathcal{U}$ represent the \emph{kinetic} and \emph{potential} energy of the multi-body system and they are described in ~\cite[Section~3.5]{Traversaro2017ModellingDynamics}.
It is possible to prove that Equation~\eqref{eq:system_initial} is the solution of \emph{Hamilton's Variational Principle} considering the Lagrangian function~\eqref{eq:lagrangian_multi-body}.

By stacking all the Jacobians and contact wrenches, we can rewrite Equation~\eqref{eq:system_initial} as follows:
\begin{equation}
\label{eq:system}
{M}({q})\dot{{\nu}} + {C}({q}, {\nu}){\nu} + {G}({q}) =  \begin{bmatrix}
{0}_{6\times n} \\ I_n
\end{bmatrix}{\tau}_s + {J}_{\mathcal{C}}^\top \mathrm{f},
\end{equation}
where
\begin{equation}
	{J}_{\mathcal{C}}({q}) = 
	\begin{bmatrix}{J}_{\mathcal{C}_1}({q}) \\ \vdots \\ {J}_{\mathcal{C}_{n_c}}({q})  \end{bmatrix}, \quad
	\mathrm{f} = \begin{bmatrix}
		{}_{\mathcal{C}_1[\mathcal{I}]}\mathrm{f}_1 \\
		\vdots\\
		{}_{\mathcal{C}_{n_c}[\mathcal{I}]}\mathrm{f}_{n_c}
	\end{bmatrix}.
\end{equation}
In the case of rigid contacts between the robot and environment, we consider a set of holonomic constraints represented as
\begin{equation}\label{eqn:constraintsAll}
{}^\mathcal{I} H _{\mathcal{C}_{i}} \equiv {}^\mathcal{I} \bar{H} _{\mathcal{C}_{i}}.
\end{equation}
In other words, we require that the position and orientation of the frame associated with each contact link to be constant and equal to ${}^\mathcal{I} \bar{H} _{\mathcal{C}_{i}}$.
By time differentiating Equation~\eqref{eqn:constraintsAll} and recalling that $\mathrm{v} = J_{C_i} \nu$~\eqref{eq:mixed-link-velocity}, we rewrite the rigid contact constraint in the \emph{Pfaffian form}~\citep[Section~8.7]{Lynch2017ModernControl} as
\begin{equation}
\label{eqn:constraintsAll_velocity}
    {J}_{\mathcal{C}_i}(q) {\nu} = 0.
\end{equation}
Here, we prescribe the spatial velocity associated with each link in contact equal to zero. Since the Jacobian matrix is a smooth mapping from $\lieGroup{Q}$ to $\mathbb{R}^{6\times (6 + n)}$, Equation~\eqref{eqn:constraintsAll_velocity} can be differentiated
\begin{equation}\label{eq:holonomic_constraint}
	{J}_{\mathcal{C}_i} \dot{{\nu}} + \dot{{J}}_{\mathcal{C}_i} {\nu} = 0,
\end{equation}
obtaining a dependency on $\dot{{\nu}}$. Equations \eqref{eq:system} and \eqref{eq:holonomic_constraint} together are the dynamical equations that describe the motion of a floating-base system that instantiates rigid contacts with the environment.
\par
The dynamics~\eqref{eq:system_initial} is often expressed by separating the first $6$ rows, referring to the non-actuated floating base, from the last $n$ rows referring to the actuated joints as:
\begin{IEEEeqnarray}{c}
\IEEEyesnumber \phantomsection
M_{\nu}(q) \dot{\nu} + h_{\nu} (q, \nu) = \sum_{k = 1}^{n_c} J^\top_{{\mathcal{C}_k}_\nu}(q) \;  {}_{\mathcal{C}_k[\mathcal{I}]}\mathrm{f}_k, \IEEEyessubnumber \label{eq:robot_dynamics_base}\\
M_{s}(q) \dot{\nu} + h_{s} (q, \nu) = \tau_s + \sum_{k = 1}^{n_c} J^\top_{{\mathcal{C}_k}_s}(q)  \; {}_{\mathcal{C}_k[\mathcal{I}]}\mathrm{f}_k, \IEEEyessubnumber \label{eq:robot_dynamics_joints}
\end{IEEEeqnarray}
where $h(q,\nu) = C(q,\nu) + G (q)$, the subscript $\nu$ refers to the first $6$ rows of the matrix, while $s$ refers to the last $n$ rows. 
\section{Centroidal dynamics~\label{sec:centroidal-dynamics}}
Let us introduce the \emph{articulated body momentum} as the sum of the 6D spatial momentum \eqref{eq:spatial_momentum-rigid_body} of each rigid body of the floating base system, i.e. 
\begin{equation}
    {}_B h = \sum _{j=1}^{n_c} \AdM^*_{{}^B H  _ j } {}_j \mathbb{M}_j {}^j \mathrm{v}_{\lieGroup{I}, j}.
\end{equation}
In some cases, it is convenient to introduce the \emph{articulated body centroidal momentum}. Let $\bar{G}$ be a frame whose origin is located on the CoM of the multi-body system, while the orientation is parallel to the inertial frame $\mathcal{I}$.
The \emph{articulated body centroidal momentum}, often shortened to \emph{centroidal momentum} and denoted with ${}_{\bar{G}} h \in \mathbb{R}^6$ is given by
\begin{IEEEeqnarray}{ll}
\phantomsection \IEEEyesnumber \label{eq:centroidal_momentum} \IEEEyessubnumber*
    {}_{\bar{G}} h  = \begin{bmatrix}
    {}_{\bar{G}} h ^p \\ {}_{\bar{G}} h ^\omega
    \end{bmatrix}:&= \AdM^*_{{}^{\bar{G}} H  _ B } {}_{B} h \\
    &= \AdM^*_{{}^{\bar{G}} H  _ B } \sum _j \AdM^*_{{}^B H  _ j } {}_j \mathbb{M}_j {}^j \mathrm{v}_{\lieGroup{I}, j} \\
    &=  {}_{\bar{G}} {X}^B \sum_j {}_{B}{X}^i {}_j \mathbb{M}_j {}^j \mathrm{v}_{\mathcal{I},j}.
\end{IEEEeqnarray}
It is worth recalling that the linear component of the centroidal momentum ${}_{\bar{G}} h ^p$ depends linearly on the CoM velocity
\begin{equation}
    {}_{\bar{G}} h ^p = m \dot{x}_\text{CoM}.
\end{equation}
\par
Seeing that the link velocity linearly depends on the velocity of the multi-body system $\nu$ \eqref{eq:mixed-link-velocity}, Equation~\eqref{eq:centroidal_momentum} can be factorized as follows
\begin{equation}\label{eq:cmm_intro}
	{}_{\bar{G}} h = J_\text{CMM} \nu
\end{equation}
where ${J}_\text{CMM} \in \mathbb{R}^{6\times n}$ is the \emph{Centroidal Momentum Matrix} (CMM) \citep{orin08}.

The centroidal momentum rate of change balances the external wrenches applied to the robot:
\begin{IEEEeqnarray}{ll}
\phantomsection \IEEEyesnumber \label{eq:centroidal_momentum_dynamics} \IEEEyessubnumber*
	{}_{\bar{G}} \dot{{h}} &= \sum_{j = 1}^{n_c} {}_{\bar{G}}{X}^{\mathcal{C}_j} {}_{\mathcal{C}_j}\mathrm{f}_j + m \bar{{g}} \\
	&= \sum_{j = 1}^{n_c} \begin{bmatrix}
	{}^{\mathcal{I}}{R}_{\mathcal{C}_j} & {0}_{3\times 3} \\
	\left[\left({}^{\mathcal{I}}{p}_j - {x}_\text{CoM}\right)\times\right] {}^{\mathcal{I}}{R}_{\mathcal{C}_j} & {}^{\mathcal{I}}{R}_{\mathcal{C}_j}
	\end{bmatrix} {}_{\mathcal{C}_j}\mathrm{f}_j + m \bar{{g}} 
\end{IEEEeqnarray}
The adjoint matrix ${}_{\bar{G}}{X}^{\mathcal{C}_j} \in \mathbb{R}^{6 \times 6}$ transforms the contact wrench from the application frame (located in ${}^{\mathcal{I}}{p}_j$ with orientation ${}^{\mathcal{I}}{R}_j$) to $\bar{G}$. Finally, $\bar{{g}} = \left[\begin{smallmatrix} 0 & 0 & -g & 0 & 0 & 0\end{smallmatrix}\right]^\top$ is the 6D gravity acceleration vector.
\par
If the external wrenches are expressed in mixed representation, the centroidal momentum dynamics~\eqref{eq:centroidal_momentum_dynamics} becomes
\begin{IEEEeqnarray}{ll}
\phantomsection \IEEEyesnumber \label{eq:centroidal_momentum_dynamics_mixed} \IEEEyessubnumber*
	{}_{\bar{G}} \dot{{h}} &= \sum_{j = 1}^{n_c} {}_{\bar{G}}{X}^{\mathcal{C}_j[\mathcal{I}]}\; {}_{\mathcal{C}_j[\mathcal{I}]}\mathrm{f} + m \bar{{g}} \\
	&= \sum_{j = 1}^{n_c} \begin{bmatrix}
	I_3 & {0}_{3\times 3} \\
	\left({}^{\mathcal{I}}{p}_j - {x}_\text{CoM}\right)\times & I_3 
	\end{bmatrix} {}_{\mathcal{C}_j[\mathcal{I}]}\mathrm{f}_j + m \bar{g}.
\end{IEEEeqnarray}
\par
As will become evident later on in the manuscript, the centroidal momentum dynamics will play a crucial role in the design of the controller presented in Chapter~\ref{chapter:Centroidal_mpc}.



