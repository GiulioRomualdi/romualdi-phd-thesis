\section{Quadratic Programming\label{sec:qp}}
The optimization problem \eqref{eq:optmization_problem_def_real} is called a \emph{quadratic program} (QP) if the objective is convex quadratic and the constrain functions are affine. We express a quadratic program problem in the following form:
\begin{IEEEeqnarray}{cc}
\phantomsection \label{eq:qp} \IEEEyesnumber \IEEEyessubnumber*
\; \; \minimize_x \; \; & \frac{1}{2} x^\top P x + q^\top x + r\\
\st & G x \preceq h.
\end{IEEEeqnarray}
$x\in\mathbb{R}^n$, $P \succ 0$ is a positive definite matrix, $q\in \mathbb{R}^n$, $G\in \mathbb{R}^{m \times n}$, $h\in \mathbb{R}^m$ and $r\in\mathbb{R}$.
\begin{figure}[t]
\centering
    \begin{subfigure}[b]{0.48\textwidth}
        \centering
        \includegraphics{chapter_optimization_introduction/figures/qp_active.tikz}
        \caption{Optimizer on boundary of $S$.}
        \label{fig:qp_active}
    \end{subfigure}
    \hfill
    \begin{subfigure}[b]{0.48\textwidth}
        \centering
        \includegraphics{chapter_optimization_introduction/figures/qp.tikz}
        \caption{Optimizer in interior of $S$.}
        \label{fig:qp}
    \end{subfigure}
	\caption[Solution of a QP problem]{Geometric interpretation of the QP solution. (a) The solution belongs to the boundary of $P$. One of the constraint is active (green line). (b) The solution belongs to the interior of $P$. In this case, $x^* = -P^{-1} q$.}
	\label{fig:qp_active_qp}
\end{figure}
If the set $S = \{x | G x \preceq h\}$ is not empty, the solution exists and is unique. 
Figure~\eqref{fig:qp_active_qp} illustrates the solution of a QP problem. In Figure~\eqref{fig:qp_active} the optimal solution belongs to the boundary of the feasible set $S$, while in Figure~\eqref{fig:qp} the solution belongs to the interior of $S$. In this case, the optimization problem~\eqref{eq:qp} is equivalent to an unconstrained optimization problem. Since the cost function is quadratic and hence convex, the optimal solution is given by setting the gradient of 
$\frac{1}{2} x^\top P x + q^\top x + r$ to zero:
\begin{equation}
    \nabla_x \left[\frac{1}{2} x^\top P x + q^\top x + r \right] = 0,
\end{equation}
whose solution is given by $x^* = -P^{-1} q$.

\paragraph{Dual of QP}
Consider~\eqref{eq:qp} the Lagrange function~\eqref{eq:lagrangian_function} is given by
\begin{equation}
    \label{eq:lagrangian_function_qp}
    \mathcal{L}(x, \lambda) = x^\top P x + q^\top x  - \lambda ^\top \left[ Gx -h \right],
\end{equation}
The dual cost~\eqref{eq:dual_function} is obtained by minimizing the Lagrange function~\eqref{eq:lagrangian_function_qp} with respect to $x$
\begin{equation}
\label{eq:dual_cost_qp}
    g(\lambda) = \min_x \left\{ x^\top P x + q^\top x  - \lambda ^\top \left[ Gx -h \right] \right\},
\end{equation}
For a given $\lambda$, the Lagrange function~\eqref{eq:lagrangian_function_qp} is convex, consequently the dual cost~\eqref{eq:dual_cost_qp} is obtained by setting the gradient of~\eqref{eq:lagrangian_function_qp} equal to zero. We then have
\begin{equation}
    \label{eq:x_qp_dual}
    x = -P^{-1} \left(q + G^\top \lambda\right).
\end{equation}
Finally, the dual problem~\eqref{eq:lagrangian_dual_problem} is given by substituting~\eqref{eq:x_qp_dual} into~\eqref{eq:dual_cost_qp} and computing the maximum for $\lambda \succeq 0_{m\times1}$ as
\begin{IEEEeqnarray}{cl}
\label{eq:lagrangian_dual_problem_qp} \IEEEyesnumber \IEEEyessubnumber*
\; \; \minimize\limits_{\lambda} \; \; & \frac{1}{2} \lambda^\top (G P^{-1} G^\top) \lambda + \lambda^\top(h - G P^{-1} q) + \frac{1}{2} q^\top H^{-1} q \\
\st & \lambda \succeq 0_{m\times1}.
\end{IEEEeqnarray}
The dual of a QP problem is a QP problem itself.

\paragraph{KKT of QP}
Given a QP problem~\eqref{eq:lagrangian_function_qp} the KKT conditions~\eqref{eq:kkt} become
\begin{IEEEeqnarray}{rl}
\phantomsection \label{eq:kkt_qp} \IEEEyesnumber \IEEEyessubnumber*
Px + q + G^\top \lambda &= 0 \\
\lambda^{* ^\top} (G x - h) &= 0 \\
\lambda^{*} &\succeq 0_{ m \times1}  \\
G x - h  &\preceq 0_{ m \times1}.
\end{IEEEeqnarray}
