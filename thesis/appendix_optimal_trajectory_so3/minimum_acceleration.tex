\section{Minimum acceleration trajectory in $\SO(3)$}
Given a fixed initial rotation $(t_0, R_0)$ and a final rotation $(t_f, R_f)$ and the associated right trivialized angular velocities, $(t_0, \omega_0)$ and $(t_f, \omega_f)$,  we want to compute a trajectory $R : \mathbb{R}_+ \rightarrow \SO(3)$ such that $R(t_0) = R_0$, $R(t_f) = R_f$,  $\omega(t_0) = \omega_0$, $\omega(t_f) = \omega_f$ such that right trivialized angular acceleration is minimized. More formally, we seek a trajectory $R(t)$ such that it is a stationary point of the action~\eqref{eq:hamilton_R_epsilon_definition_optimization} with a Lagrangian function defined as
\begin{equation}
    \label{eq:hamiltonin_lagrangian_minimum_acceleration_R}
    \mathcal{L}(\dot{\omega}) = \dot{\omega}^\top \dot{\omega}.
\end{equation}
\par
By substituting~\eqref{eq:hamiltonin_lagrangian_minimum_acceleration_R} into~\eqref{eq:hamilton_principle_so3_explicit} we have
\begin{IEEEeqnarray}{lcl}
\phantomsection \label{eq:hamilton_principle_so3_explicit_min_acc} \IEEEyesnumber \IEEEyessubnumber*
    \delta\mathfrak{G} &=& \int_{t_0}^{t_f}   \left\langle \frac{\partial \mathcal{L}(\dot{\omega})} {\partial \dot{\omega}}, \delta \dot{\omega} \right\rangle \diff t = \int_{t_0}^{t_f}   \left\langle \frac{\partial \mathcal{L}(\dot{\omega})} {\partial \dot{\omega}},\ddot{\eta} + \dot{\omega}\times\eta + \omega\times\dot{\eta}  \right\rangle \diff t \\
    &=& \int_{t_0}^{t_f}   \left\langle \frac{\partial \mathcal{L}(\dot{\omega})} {\partial \dot{\omega}}, \ddot{\eta} \right\rangle \diff t  \label{eq:hamilton_principle_so3_explicit_min_acc_ddnu}\\
    &&+ \int_{t_0}^{t_f}   \left\langle \frac{\partial \mathcal{L}(\dot{\omega})} {\partial \dot{\omega}}, \dot{\omega}\times\eta \right\rangle \diff t  \label{eq:hamilton_principle_so3_explicit_min_acc_domega_nu}\\
    &&+ \int_{t_0}^{t_f}   \left\langle \frac{\partial \mathcal{L}(\dot{\omega})} {\partial \dot{\omega}}, \omega\times\dot{\eta} \right\rangle \diff t \label{eq:hamilton_principle_so3_explicit_min_acc_omega_dnu}.
\end{IEEEeqnarray}
We now analyze each term of~\eqref{eq:hamilton_principle_so3_explicit_min_acc} explicitly.
By integrating~\eqref{eq:hamilton_principle_so3_explicit_min_acc_ddnu} by parts, we obtain the following:
\begin{IEEEeqnarray}{lcl}
\phantomsection \label{eq:hamilton_principle_so3_explicit_min_acc_ddnu_explicit} \IEEEyesnumber \IEEEyessubnumber*
\int_{t_0}^{t_f}   \left\langle \frac{\partial \mathcal{L}(\dot{\omega})} {\partial \dot{\omega}}, \ddot{\eta} \right\rangle \diff t &=& \at{\left\langle \frac{\partial \mathcal{L}(\dot{\omega})} {\partial \dot{\omega}}, \dot{\eta}\right \rangle}{t_0}^{t_f} \\
&& - \at{\left\langle \frac{\diff }{\diff t}\frac{\partial \mathcal{L}(\dot{\omega})} {\partial \dot{\omega}}, {\eta}\right \rangle}{t_0}^{t_f} \\
&&+ \int_{t_0}^{t_f}   \left\langle \frac{\diff^2 }{\diff t^2} \frac{\partial \mathcal{L}(\dot{\omega})} {\partial \dot{\omega}}, {\eta} \right\rangle \diff t.
\end{IEEEeqnarray}
Recalling the fact that the infinitesimal variations $\eta$ and $\dot{\eta}$ vanish at $t_0$ and $t_f$, Equation~\eqref{eq:hamilton_principle_so3_explicit_min_acc_ddnu_explicit} can be simplified as follows:
\begin{equation}
    \label{eq:hamilton_principle_so3_explicit_min_acc_ddnu_final} 
    \int_{t_0}^{t_f}   \left\langle \frac{\partial \mathcal{L}(\dot{\omega})} {\partial \dot{\omega}}, \ddot{\eta} \right\rangle \diff t = \int_{t_0}^{t_f}   \left\langle \frac{\diff^2 }{\diff t^2} \frac{\partial \mathcal{L}(\dot{\omega})} {\partial \dot{\omega}}, {\eta} \right\rangle \diff t .
\end{equation}
\par
Applying the properties \emph{scalar triple product}, \eqref{eq:hamilton_principle_so3_explicit_min_acc_domega_nu} writes as 
\begin{equation}
    \label{eq:hamilton_principle_so3_explicit_min_acc_domega_nu_final} 
    \int_{t_0}^{t_f}   \left\langle \frac{\partial \mathcal{L}(\dot{\omega})} {\partial \dot{\omega}}, \dot{\omega}\times\eta \right\rangle \diff t = \int_{t_0}^{t_f}   \left\langle \frac{\partial \mathcal{L}(\dot{\omega})} {\partial \dot{\omega}} \times\dot{\omega}, \eta \right\rangle \diff t .
\end{equation}
\par
Similarly, we rewrite \eqref{eq:hamilton_principle_so3_explicit_min_acc_omega_dnu} applying the properties of the scalar triple product and integrating it by parts:
\begin{IEEEeqnarray}{lcl}
\phantomsection  \IEEEyesnumber \IEEEyessubnumber*
\int_{t_0}^{t_f}   \left\langle \frac{\partial \mathcal{L}(\dot{\omega})} {\partial \dot{\omega}}, \omega \times \dot{\eta} \right\rangle \diff t &=& \int_{t_0}^{t_f}   \left\langle \frac{\partial \mathcal{L}(\dot{\omega})} {\partial \dot{\omega}} \times \omega,  \dot{\eta} \right\rangle \diff t \\
&=&- \int_{t_0}^{t_f}   \left\langle\frac{\diff }{\diff t} \left(\frac{\partial \mathcal{L}(\dot{\omega})} {\partial \dot{\omega}} \times \omega\right),  {\eta} \right\rangle \diff t \label{eq:hamilton_principle_so3_explicit_min_acc_ddnu_explicit_vanish},
\end{IEEEeqnarray}
where in~\eqref{eq:hamilton_principle_so3_explicit_min_acc_ddnu_explicit_vanish}, we use the fact that $\eta(t_0) = \eta(t_f) = 0$.
\par
By substituting \eqref{eq:hamilton_principle_so3_explicit_min_acc_ddnu_final}, \eqref{eq:hamilton_principle_so3_explicit_min_acc_domega_nu_final} and \eqref{eq:hamilton_principle_so3_explicit_min_acc_ddnu_explicit_vanish} into~\eqref{eq:hamilton_principle_so3_explicit_min_acc}, we obtain the final formulation of the infinitesimal variation of the action functional:
\begin{equation}
    \delta\mathfrak{G} = \int_{t_0}^{t_f} \left\langle 
  \frac{\diff^2}{\diff t^2}\frac{\partial \mathcal{L}(\dot{\omega})} {\partial \dot{\omega}}
    -\frac{\diff}{\diff t} \left(\frac{\partial \mathcal{L}(\dot{\omega})} {\partial \dot{\omega}} \times \omega\right)
    +\frac{\partial \mathcal{L}(\dot{\omega})} {\partial \dot{\omega}} \times \dot{\omega}, 
    \eta \right\rangle \diff t.
\end{equation}
\par
We now recall that Hamilton's principle must be valid for all possible variations $\eta$, and consequently $\delta \mathfrak{G} = 0$ implies that
\begin{equation}
    \label{eq:hamilton_pde_so3}
    \frac{\diff^2}{\diff t^2}\frac{\partial \mathcal{L}(\dot{\omega})} {\partial \dot{\omega}}
    -\frac{\diff}{\diff t} \left(\frac{\partial \mathcal{L}(\dot{\omega})} {\partial \dot{\omega}} \times \omega\right)
    +\frac{\partial \mathcal{L}(\dot{\omega})} {\partial \dot{\omega}} \times \dot{\omega} = 0.
\end{equation}
\par
Combining the Lagrangian definition~\eqref{eq:hamiltonin_lagrangian_minimum_acceleration_R} with the partial differential equation~\eqref{eq:hamilton_pde_so3} we obtain
\begin{equation}
\label{eq:min_acc_so3_final_ode}
\dddot{\omega}+ \omega \times \ddot{\omega} = 0.
\end{equation}
Finally, we can conclude that a trajectory $R(t)$ that satisfies \eqref{eq:min_acc_so3_final_ode} is a minimum acceleration trajectory.
\par
It is worth noting that~\eqref{eq:min_acc_so3_final_ode} does not admit an analytic solution for arbitrary boundary conditions.
However, in the case of zero initial and final velocity $\omega$, it is possible to show that  
\begin{IEEEeqnarray}{c}
\phantomsection \label{eq:hamilton_so3_min_acc_close_solution} \IEEEyesnumber \IEEEyessubnumber*
    R(t) = \exp{\left(s(t-t_0) \log\left(R_f R_0^\top  \right) \right)} R_{0} \\ s(\tau) = \frac{3}{(t_f - t_0)^2} \tau^2 - \frac{3}{(t_f - t_0)^3} \tau^3,
\end{IEEEeqnarray}
satisfies condition~\eqref{eq:min_acc_so3_final_ode}.
To prove the latest statement, we first compute the right trivialized angular velocity as
\begin{IEEEeqnarray}{ll}
\phantomsection \IEEEyesnumber \IEEEyessubnumber*
    \hskip-1.5cm\omega^\wedge &= R^\top \dot{R} \\
    \hskip-1.5cm&= \dot{s}R_{0} ^\top \exp{\left(-s\log\left(R_f R_0^\top  \right) \right)}
    \log\left(R_f R_0^\top  \right) 
    \exp{\left(s\log\left(R_f R_0^\top  \right) \right)} R_{0} \\
    \hskip-1.5cm&= \dot{s}R_{0} ^\top \exp{\left(-s\log\left(R_f R_0^\top  \right) \right)}
   R_{0} \log\left( R_0^\top R_f \right)  R_{0}^\top
    \exp{\left(s\log\left(R_f R_0^\top  \right) \right)} R_{0} \\
        \hskip-1.5cm&= \dot{s}\exp{\left(-s\log\left(R_0^\top R_f  \right) \right)}
 \log\left( R_0^\top R_f \right)  
    \exp{\left(s\log\left( R_0^\top R_f  \right) \right)}\\
          \hskip-1.5cm&= \dot{s}\exp{\left(-s\log\left(R_0^\top R_f  \right) \right)}
    \exp{\left(s\log\left( R_0^\top R_f  \right) \right)}  \log\left( R_0^\top R_f \right)   \\
\hskip-1.5cm&= \dot{s} \log\left( R_0^\top R_f \right).
\end{IEEEeqnarray}
Here, we exploit the exponential and the adjoint maps properties -- Equation~\eqref{eq:adjoint_action_def} and Section~\ref{sec:exp_and_log_maps}.
We notice that $\omega$ satisfies the constraint~\eqref{eq:min_acc_so3_final_ode}:
\begin{IEEEeqnarray}{lcl}
\phantomsection \IEEEyesnumber \IEEEyessubnumber*
\dddot{\omega}+ \omega \times \ddot{\omega} &=& \frac{\diff ^4}{\diff t ^4} s \Log\left( R_0^\top R_f \right)\\
&&+ \left(\frac{\diff}{\diff t} s \Log\left( R_0^\top R_f \right) \right)\times \left(\frac{\diff ^3}{\diff t ^3} s \Log\left( R_0^\top R_f \right)\right) \\ 
&=&\frac{\diff ^4}{\diff t ^4} s \Log\left( R_0^\top R_f \right)  \\
&=&0.
\end{IEEEeqnarray}
Furthermore, the trajectory~\eqref{eq:hamilton_so3_min_acc_close_solution} satisfies the boundary conditions: namely $R(t_0) = R_0$, $R(t_f) = R_f$, $\omega(t_0) = \omega(t_f) = 0$. 
To conclude, the trajectory~\eqref{eq:hamilton_so3_min_acc_close_solution} is a minimum acceleration trajectory in the interval $[t_0, t_f]$.