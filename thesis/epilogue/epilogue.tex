\chapter*{Epilogue}
\addcontentsline{toc}{chapter}{Epilogue}
\chaptermark{Epilogue}
\lhead{\leftmark}
This thesis presented an application of different model-based controllers for time-critical humanoid locomotion. Exploiting a three-layer model-based architecture, we investigated the specific implementation of each block when the considered models change. Part~\ref{part:wbc} presents the design of three whole-body controllers for humanoid robot locomotion. The proposed controllers employ similar control algorithms and the main distinguishing factors are the robot and the description of the environment.
In Chapter~\ref{chapter:benchmarking_wbc} we compared whole-body controllers for locomotion on rigid surfaces. In this chapter, we proposed a kinematics-based and a dynamics-based whole-body controllers. The experiments were carried out on the Humanoid Robot iCub v2.7. We showed that when the robot is position controlled, the architecture was able to achieve the highest walking velocity. On the other hand, we noticed that unstructured uncertainty and the low reliability of the low-level torque-control loop, possibly due to the lack of collocated joint torque sensing, prevented the torque-based architecture from reaching the same performance obtained in simulation.
\par
Motivated by curiosity, we began investigating the consequences of loosening the assumption of the rigid contact model hypothesized in Chapter~\ref{chapter:benchmarking_wbc}. Chapter~\ref{chapter:wbc_visco_elastic} proposes a contact model that describes the mechanical characteristics of a visco-elastic carpet. To keep the problem still treatable online, we model the system as a continuum spring-damper system rather than exploiting the finite element method (FEM). In fact, while providing enhanced modeling capabilities, FEM methods demand heavy computational time, which usually prohibits their use in time-critical feedback control applications.
The whole-body controller then considers the model to compute viable joint torques, allowing the robot to perform a locomotion task.
Results are shown only in simulation. The poor performance of the low-level controller along with too noisy contact force information prevents us from achieving acceptable results on iCub.
\par
Both the architectures presented in Chapters~\ref{chapter:benchmarking_wbc} and \ref{chapter:wbc_visco_elastic} assume that the robot links do not deform during this locomotion task. In Chapter~\ref{chapter:flexible_joints} we attempted to loosen this assumption by modeling the link flexibility with passive visco-elastic joints. We proposed a whole-body controller that considers the joint elasticity while computing the desired actuated joint torques. Furthermore, since in our case the deflection was not directly measurable, we proposed an observer aiming at estimating the flexible joint state, considering the measured contact force and the actuated joint state. Results are shown only in simulation.
\par
Part~\ref{part:simplified} analyzed the outer loops of the three-layer controller architecture.
Chapter~\ref{chapter:simplified_benchmarking} presented and compared several DCM based kinematic-based architectures. Keeping a fixed trajectory optimization layer, we designed two simplified model controllers: an instantaneous controller and an MPC controller.
We benchmarked the two strategies on the Humanoid Robot iCub v2.7. We noticed that the computed ZMP is smoother when the simplified model control uses the predictive law to generate it. However, although this smoother behavior did contribute to fewer robot vibrations, the overall robot performance became less reactive, and as a consequence, the robot may fall. On the other hand, when the robot is position-controlled and the simplified layer implements the proposed instantaneous controller, iCub was able to reach the desired walking speed of $\SI{0.3372}{\meter\per\second}$. That is, to the best of our knowledge, the highest walking velocity achieved by the iCub robot v2.7.
\par
Motivated by the results obtained from the simplified controller, we decided to investigate whether, by increasing the complexity of the model, we were able to improve the robustness of the control architecture in the case of unexpected interactions with the environment. Chapter~\ref{chapter:Centroidal_mpc} attempted to answer this question. We presented a Non-Linear Model Predictive Controller for humanoid robot locomotion with online step adjustment capabilities. Differently from bipedal walking architectures based on simplified models, the presented approach considers the reduced centroidal model, thus allowing us to consider the contact location adjustment directly in the dynamics stabilization problem, while keeping the control problem still treatable online. The approach was validated in the position-controlled Humanoid Robot iCub v3.
\par
In summary, satisfactory experiments on the real robot could be achieved only when the robot was in position control, keeping the torque control walking still an open question in the case of a robot with non-collocated torque sensing.
In all the architectures presented in Part~\ref{part:wbc} and Chapter~\ref{chapter:simplified_benchmarking}, the contacts are planned without taking into consideration the state of the robot. Chapter~\ref{chapter:Centroidal_mpc}, addressed this issue by letting the optimizer determine where to place the contact. This choice had the drawback of complexifying the overall architecture.
In the case of flat terrain, simplified models are still one of the best choices in terms of complexity and robustness capabilities. On the other hand, we believe that reduced model controllers are one of the best trade-offs between an offline predictive planner that considers the complete robot description and a simplified model controller that relays on a tailored model depending on the desired task.
\par
In the Prologue, we mentioned that the future of humanoid robotics is the safe interaction between robots and humans in a human-like environment. The road to achieving this goal is still long and full of pitfalls.
We hope that the controllers presented in this thesis may help humankind to take a step forward in this direction.